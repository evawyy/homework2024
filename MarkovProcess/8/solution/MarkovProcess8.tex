%!Mode:: "TeX:UTF-8"
%!TEX encoding = UTF-8 Unicode
%!TEX TS-program = xelatex
\documentclass{ctexart}
\newif\ifpreface
%\prefacetrue
\input{../../../global/all}
\begin{document}
\large
\setlength{\baselineskip}{1.2em}
\ifpreface
  \input{../../../global/preface}
\else
  %\maketitle
\fi
\newgeometry{left=2cm,right=2cm,top=2cm,bottom=2cm}
%from_here_to_type.pr
\begin{problem}\label{pro:1}
  Assume \((X_n:n \geq 0)\) is an irreducible Markov chain on \(E\).
  Prove that \((X_n:n \geq 0)\) is recurrent (or transient) \(\iff\)
  \(\forall i \in E\),
  \[
    \mathbb{P}\left(\bigcap_{n=1}^{\infty} \bigcup_{k=n}^{\infty} \{X_k=i\}\right)=1 (\text{or } 0).
  \]
\end{problem}
\begin{solution}
  Since \((X_n: n \geq 0)\) is irreducible, then
  \(\forall i \in E, \exists n_i\) s.t.\(0< \mathbb{P}(X_{n_i}=i))\leq 1\).
  And \(\forall \omega \in \bigcap_{n=1}^{\infty} \bigcup_{k=n}^{\infty} \{X_k =i\}\), then \(\exists k_i \geq n_i X_{k_i}(\omega)=i\).
  Besides, \((X_n:n \geq 1)\) is a Markov chain,
  let \(T_i(\omega):=\min\{n \geq 1: X_0(\omega)=X_n(\omega)=i\}\),
  then \(\mathbb{P}_i(T_i<\infty)=\mathbb{P}(\min\{n \geq 1: X_{n_i + n}(\omega)=i\}< \infty) \geq \mathbb{P}(\{\bigcap_{n=1}^{\infty} \bigcup_{k=n}^{\infty} \{X_k=i\} \})\).

  \(\forall \omega \in \{T_i < \infty\}\), then \(\exists m_i\) s.t. \(\forall 1 \leq j < m_i, X_j \neq i, X_{m_i}=i\),
  Then \(\forall n \geq 1\), let \(t_n=[\frac{n}{m_i}]+ 1\), \(Y_m=X_k, m \equiv k \mod m_i, 0 \leq k \leq m_i-1\), so
  \(Y_{t_nm_i}(\omega)=i\).
  Thus, \(\mathbb{P}_i(T_i < \infty) \leq \mathbb{P}(\bigcap_{n=1}^{\infty} Y_{t_nm_i}=i)\).
  Besides, \(\{\bigcup_{k=n}^{\infty} \{X_k=i\} \supset\{X_{t_nm_i}=i\}\supset\{Y_{t_nm_i}=i\}\).
  Then \(\bigcap_{n=1}^{\infty} \bigcup_{k=n}^{\infty} \{X_k=i\} \supset \bigcap_{n=1}^{\infty} \{Y_{t_nm_i}=i\}\),
  thus, \(\mathbb{P}_i(T_i < \infty) \leq \mathbb{P}(\bigcap_{n=1}^{\infty} Y_{t_nm_i}=i) \leq \mathbb{P}(\bigcap_{n=1}^{\infty} \bigcup_{k=n}^{\infty} X_k=i)\).

  Therefore, \(\mathbb{P}_i(T_i < \infty) = \mathbb{P}(\bigcap_{n=1}^{\infty} \bigcup_{k=n}^{\infty} X_k=i)\). Then, \((X_n: n \geq 0)\) is recurrent (or transient)
  \(\iff \) \(\mathbb{P}_i(T_i < \infty)=1 (\text{or} 0)\) \(\iff \mathbb{P}(\bigcap_{n=1}^{\infty} \bigcup_{k=n}^{\infty} X_k=i)=1 (\text{or 0})\).
\end{solution}

% \begin{problem}\label{pro:2}
%   Let \((X_n:n \geq 0)\) is a one dimension simple random walk, and \(P\) is it's transition matrix.
%   Let \(a \leq b \in \mathbb{Z}\) satisfies \(\mathbb{P}(a \leq X_0 \leq b)=1\).
%   Define \(\tau = \inf \{n \geq 0: X_n =a \text{or } b\}, Y_n=X_{n \AND \tau}\).
%   Prove: \((Y_n: n \geq 0)\) is Markov chain on \([a,b]\cap \mathbb{Z}\), and give its transition matrix and
%   the classification.
% \end{problem}
\begin{problem}\label{pro:3}
  Prove: \((X_n: n \geq 0)\) is Markov chain on \(E\), where \(E\) is finite. Then \(\exists x \in E\), \(x\) is recurrent.
\end{problem}
\begin{solution}
  Since \(\sum_{i \in E}p_{ki}^{*}=\sum_{i \in E}\sum_{n \geq 0}p_{ki}(n)=\sum_{n \geq 0}\sum_{i \in E}p_{ik}(n)=\infty\), and \(E\) is finite,
  then \(\exists i\), \(p_{ki}^{*}=\infty\). So by colloary of 3.3.6 on textbook, then \(i\) is not
  transient, so \(i\) is recurrent.
\end{solution}

\begin{problem}\label{pro:4}
  Assume \((X_n:n \geq 0)\) is Markov chain on \(\mathbb{Z}\). Prove it is transient \(\iff\)
  \(\forall \mu_0\) is primitive distribution, \(\lim_{n \to \infty} |X_n| \overset{\text{a.s.}}{=} \infty\).
\end{problem}
\begin{solution}
  Only need to prove that \(\forall k \in \mathbb{N},\liminf_{n \to \infty}|X_n|>k,a. s.\).
  Consider the event \(\liminf_{n \to \infty}|X_n|\leq k\), it means \(\forall n \in \mathbb{N},\exists t \geq n,X_t \in [-k,k]\).
  So we only need to prove \(\mathbb{P}\left(\bigcap_{n=1}^{\infty} \bigcup_{t=n}^{\infty} \{X_t \in [-k,k]\}\right)=0\).
  It is sufficient to prove that \(\mathbb{P}(\bigcup_{u \in [-k,k]}\bigcap_{n=1}^{\infty} \bigcup_{t=n}^{\infty} \{X_t=u\})=0\).
  Since \((X_n)\) is transient, it has been proved that \(\mathbb{P}(\bigcap_{n=1}^{\infty} \bigcup_{t=n}^{\infty} \{X_t=u\})=0\).
  So \(\mathbb{P}(\bigcup_{u \in [-k,k]}\bigcap_{n=1}^{\infty} \bigcup_{t=n}^{\infty} \{X_t=u\})=0 \leq \sum_{u \in [-k,k]}\mathbb{P}(\bigcap_{n=1}^{\infty} \bigcup_{t=n}^{\infty} \{X_t=u\})=0\).
\end{solution}
\begin{problem}\label{pro:5}
  Assume \(\{a_i: i \geq 1\} \subset (0,1)\). Consider \(E:=\mathbb{N}\), \(P\) is a transition matrix on \(E\),
  where \(p_{ij}=a_i \mathbbm{1}_{\{j=0\}} + (1-a_i)\mathbbm{1}_{\{j=i+1\}}\).
  Prove:
  \begin{enumerate}
    \item \(P\) is irreducible.
    \item \(P\) is recurrent \( \iff \) \(\sum_{i}a_i=\infty\).
    \item \(P\) is ergodic \(\iff \) \(\sum_{k=1}^{\infty} \prod_{i=1}^{k-1} (1-a_i)< \infty\).
  \end{enumerate}
\end{problem}
\begin{solution}
  \begin{enumerate}
    \item \(\forall i, j \in E\), if \(i < j\), then \(p_{ij}(j-i)=\prod_{k=i}^{j-1}(1-a_k)>0\). If \(i \geq j\), then \(p_{ij}(j + 1)=a_i \prod_{k=0}^{j-1} (1-a_k)>0\).
      So \(P\) is recurrent.
    \item Since \(P\) is irreducible, then we only need to concern the circumstance when \(X_0=0\). Then \(\{T_0 > n\} \overset{\text{a.s.}}{=}\{X_k=k,k=0,\cdots,n\}\).
      Then \(\mathbb{P}_0(T_0 = \infty)=\mathbb{P}_0(\bigcap_{n}\{T_0 > n\})=\lim_{n \to \infty}\mathbb{P}_0(X_k=k,k=0,\cdots,n)=\lim_{n \to \infty}\prod_{k=0}^{n-1} (1-a_k) =\prod_{k=0}^{\infty} (1-a_k) \).
      Then \(\mathbb{P}_0(T_0=\infty)=0 \iff \prod_{k=0}^{\infty} (1-a_k) =0\iff \sum_{k}a_k=\infty\).
    \item Since \(\mathbb{E}_0(T_0)=\sum_{n \in E}\mathbb{P}_0(T_0 >n)=\sum_{n=0}^{\infty} \prod_{k=0}^{n-1} (1-a_k)\), then \(P\) is ergodic
      \(\iff \) \(\mathbb{E}_0(T_0) < \infty\) \(\iff\) \(\sum_{n=0}^{\infty} \prod_{k=0}^{n-1} (1-a_k) <\infty\).
  \end{enumerate}
\end{solution}

% \begin{problem}\label{pro:5}
%   Assume \(P\) is a transition matrix on \(\mathbb{Z}^+\), which has a first line \(\{a_0,a_1,\cdots\}\),
%   \(\forall i \geq 1\), \(p_{i,i-1}=1\), and \(\forall j \neq i-1\), \(p_{i,j}=0\).
%   Discuss the irreducibility, recurrence, ergodicity and periodicity of \(0\).
% \end{problem}
% \begin{problem}\label{pro:6}
%   Assume \(P\) is a transition matrix on \(E\).
%   Prove: \(\forall i \in E\), \(\lim_{n \to \infty}p_{ii}(n)\) exists, and \(\lim_{n \to \infty}p_{ii}(n)=\frac{1}{F'_{ii}(1)}=\frac{1}{\mathbb{P}_i(T_i)}\).
% \end{problem}

\begin{problem}\label{pro:7}
  Assume \(P\) is a transition matrix on \(E\) and \(P\) is irreducible, \(j \in E\).
  Prove: \(P\) is recurrent \(\iff\) \(1\) is the minimum non negetive solution of
  \begin{equation}\label{equ:7}
    y_i = \sum_{k \neq j} p_{ik}y_k + p_{ij}, i \in E\setminus\{j\}
  \end{equation}
\end{problem}
\begin{solution}
  ``\(\implies\)'': If \(P\) is recurrent, then the bounded solution of \(y_i=\sum_{k \in E}p_{ik}y_k, i  \in E\setminus\{j\}\) is constant.
  Specially, \(1\) is the minimum non-negetive solution.
  If \(1\) is not the minimum non-negetive solution of Equation \eqref{equ:7}, let \(\{z_i:i \geq 0\}\) be the minimum
  non-negetive solution. Let \(x_i=z_i-1\), then \(\{x_i:i \geq 0\}\) is the bounded solution of
  \(y_i = \sum_{k \in E}p_{ik}y_k, i \in E\setminus\{j\}\), contradiction.

  ``\(\impliedby\)'': If \(P\) is transient, then the bounded solution of \(y_i=\sum_{k \in E}p_{ik}y_k, i  \in E\setminus\{j\}\) has non constant ones.
  Let it be \(\{z_i:i \in E\}\). W.L.O.G., we can assume \(z_j=0,|z_i| \leq 1, \forall i \in E, \exists i_0 \in E, z_{i_0} <0\).
  Let \(y_i = 1 + z_i, i \in E\), then \(\{y_i: i \in E\}\) is the bounded solution of Equation \eqref{equ:7}.
  But \(y_{i_0} <1, y_i \geq 0, i \in E\), and \(\{y_i: i \in E\}\) is smaller than \(1\), which is a contradiction.
\end{solution}

\begin{problem}\label{pro:8}
  Let \(\{a_k:k \geq 0\}\) satisfies \(\sum_{k \geq 0}a_k =1, a_k \geq 0, a_0 >0\), \(\mu :=\sum_{k=1}^{\infty} ka_k >1\).
  Define \(p_{ij}=
  \begin{cases}
    a_j         & , i =0                    \\
    a_{j-i + 1} & ,i \geq 1 \AND j \geq i-1 \\
    0           & ,\text{otherwise}
  \end{cases}
  \).
  Prove: \(P\) is transient.
\end{problem}
\begin{solution}
  First of all, we prove that \(P\) is irreducible: Since \(\sum_{k=1}^{\infty} ka_k > 1\), then \(\exists m : a_m >0\).
  And \(\forall i \geq 1\), \( p_{i-1,i}=a_0 >0\). Then \(\forall i,j \), if \(i < j\), then \(p_{ij}(j-i)=a_0^{j-i}>0\).
  If \(i \geq j\), let \(t \equiv i-j \pmod{m}, 1 \leq t \leq m\), then \(p_{ij}(t + 1)=a_0^{t}a_m >0\).

  Second, we prove that \(P\) is transient: Let \(f(z)=\sum_{k=0}^{\infty} a_kz^k, z \in [0,1]\),
  then \(f(0)=a_0 >0, f(1)=1\), \(f'' \geq 0\), \(f'(0)=a_1 \geq 0\).
  Therefore, \(\exists \mid c \in (0,1)\), s.t. \(c=f(c)\), then \(\forall i \geq 1\),
  \[
    c^i=\sum_{k=0}^{\infty} a_kc^{k + i-1}=\sum_{j=i-1}^{\infty} a_{j-i + 1}c^j=\sum_{j=0}^{\infty} p_{ij}c^j
  \]
  So \(P\) is transient.
\end{solution}

\end{document}
