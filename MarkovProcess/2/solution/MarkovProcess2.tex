%!Mode:: "TeX:UTF-8"
%!TEX encoding = UTF-8 Unicode
%!TEX TS-program = xelatex
\documentclass{ctexart}
\newif\ifpreface
%\prefacetrue
\input{../../../global/all}
\begin{document}
\large
\setlength{\baselineskip}{1.2em}
\ifpreface
  \input{../../../global/preface}
\else
  \newgeometry{left=2cm,right=2cm,top=2cm,bottom=2cm}
  \maketitle
\fi
%from_here_to_type

\begin{problem}\label{pro:1}
  Prove that if \((X_n:n \geq 0)\) is a simple random walk, then so is \((-X_n:n \geq 0)\).
\end{problem}
\begin{solution}
  Since \((X_n: n \geq 0)\) is a simple random walk, then \(\exists (\xi_i: i \geq 0)\) are i.i.d.
  r.v. \(X_0\) is a r.v. which is independent with \(\xi_1\) such that
  \(X_n= X_0 + \sum_{i=0}^n \xi_i\), \(\mathbb{P}(|\xi_1|=1)=1\).
  So let \(Y_n = -X_n\), \(Y_0=-X_0\) is r.v., \(\varepsilon_i = -\xi_i, i \geq 0\),then
  \((\varepsilon_i, i \geq 0)\) are i.i.d. and is independent with \(Y_0\), and \(\mathbb{P}(|\varepsilon_1|=1)=1\).
  So \((Y_n: n \geq 0)\) is a simple ramdom walk.
\end{solution}

\begin{problem}\label{pro:2}
  Let \((X_n:n \geq 0)\) be a \(d\)-dimentional random walk, and \(\mathbb{P}(|\xi_1| \geq 1)>0\),
  prove that \(\mathbb{P}(\sup_{n}|X_n|=\infty)=1\).
\end{problem}
\begin{solution}
  Since \(\mathbb{P}(|\xi_1| \geq 1)>0\), then \(\exists t \in \mathbb{R}^d\), such that \(\mathbb{P}(\xi_1=t) > 0\).
  Besides,
  \begin{equation}
    \begin{aligned}
        & \mathbb{P}(\sup_{n} |X_n|= \infty)                             \\
      = & \mathbb{P}(\bigcap_{k \in \mathbb{N}}\{\sup_n |X_n| \geq k\} ) \\
      = & \lim_{k \to \infty}\mathbb{P}(\sup_n |X_n| \geq k)             \\
      = & \inf_{k \in \mathbb{N}} \mathbb{P}(\sup_n |X_n| \geq k)
    \end{aligned}
  \end{equation}
  Then, to prove \(\mathbb{P}(\sup_n |X_n| = \infty) = 1\) is equal to prove
  \(\forall k \in \mathbb{N}, \mathbb{P}(\sup_n |X_n| \geq k) =1\).
  Let \(v > 4 \frac{k}{|t|}\) and let \(A_u = \{\omega \in \Omega: \xi_{uv+1}=t,\cdots,\xi_{uv+v}=t\)\},
  so \(\forall \omega \in A_u, |X_{uv+v}-X_{uv}|=|\sum_{m = uv+1}^{uv+v}\xi_m|=|vt|=v|t| \geq 4k\).
  Then \(2\max\{|X_{uv + v}|,|X_{uv}|\} \geq |X_{uv+v}| + |X_{uv}| \geq |\sum_{m = uv + 1}^{uv + v} \xi_m| \geq 4k\),
  so \(\max \{|X_{uv + v}|, |X_{uv}|\} \geq 2k >k\).
  Thus \(\sup_n |X_n| \geq k\).
  Besides, since \(\xi_i, i \in \mathbb{N}^+\) is i.i.d., then \(\mathbb{P}(A_u) = \mathbb{P}(\xi_1=1)^v\).
  And it is obvious that \(A_u, u \in \mathbb{N}^+\) is independent, \(\sum_{i=0}^{\infty}\mathbb{P}(A_i)=\infty\),
  by BC theorem, we can get that \(\mathbb{P}(\bigcap_{i=1}^{\infty}\bigcup_{j=i}^{\infty}A_j)=1\).
  Since \(\bigcap_{i=1}^{\infty}\bigcup_{j=i}^{\infty}A_j \subset \bigcup_{i=1}^{\infty}A_i \subset \{\sup_n |X_n| = \infty\}\),
  then \(\mathbb{P}(\{\sup_n |X_n| = \infty\})=1\).

\end{solution}

\begin{problem}\label{pro:3}
  Let \((X_n:n \geq 0)\) be a symmtry simple random walk with \(X_0=0\), for \(d=2\), prove that
  \[
    \mathbb{P}(S_{2n}=0)=\frac{1}{4^{2n}}\left(\frac{(2n)!}{(n!)^2}\right)^2
  \]
  For \(d=3\), prove that
  \[
    \mathbb{P}(S_{2n}=0)=\frac{1}{2^{2n}}\frac{(2n)!}{(n!)^2}\sum_{i + j + k = n} \left(\frac{1}{3^n}\frac{n!}{i!j!k!}\right)^2
  \]
\end{problem}
\begin{solution}
  \begin{enumerate}
    \item \(d = 2\),
      \begin{equation}
        \begin{aligned}
            & \mathbb{P}(S_{2n} = 0)                                                           \\
          = & (\frac{1}{4^{2n}})(\sum_{k=0}^n\binom{2n}{k}\binom{2n-k}{k}\binom{2n-2k}{n-k})   \\
          = & \frac{1}{4^{2n}}\sum_{k=0}^n \frac{(2n)!}{(k!)^2((n-k)!)^2}                      \\
          = & \frac{1}{4^{2n}} \frac{(2n)!}{(n!)^2}\sum_{k=0}^n\frac{(n!)^2}{(k!)^2((n-k)!)^2} \\
          = & \frac{1}{4^{2n}} \frac{(2n)!}{(n!)^2}\sum_{k=0}^n \binom{n}{k} \binom{n}{n-k}    \\
          = & \frac{1}{4^{2n}} \frac{(2n)!}{(n!)^2}\binom{2n}{n}                               \\
          = & \frac{1}{4^{2n}} (\frac{(2n)!}{(n!)^2})^2
        \end{aligned}
      \end{equation}
    \item \(d = 3\),
      \begin{equation}
        \begin{aligned}
            & \mathbb{P}(S_{2n}=0)                                                                                                       \\
          = & \frac{1}{6^{2n}}(\sum_{k+j=0}^n \binom{2n}{k} \binom{2n-k}{k} \binom{2n-2k}{j} \binom{2n-2k-j}{j} \binom{2n-2k-2j}{n-k-j}) \\
          = & \frac{1}{6^{2n}}(\sum_{j+k=0}^n \frac{(2n)!}{(k!)^2(j!)^2((n-k-j)!)^2})                                                    \\
          = & \frac{1}{6^{2n}} \frac{(2n)!}{(n!)^2} \sum_{j+k=0}^n \frac{(n!)^2}{(k!)^2(j!)^2((n-k-j)!)^2}                               \\
          = & \frac{1}{2^{2n}}\frac{(2n)!}{(n!)^2}\sum_{i + j + k = n} \left(\frac{1}{3^n}\frac{n!}{i!j!k!}\right)^2
        \end{aligned}
      \end{equation}

  \end{enumerate}

\end{solution}

\begin{problem}\label{pro:4}
  Assume \((S_n:n \geq 0)\) is a symmtry simple random walk with \(S_0 = i \in \mathbb{Z}\).
  Prove that \(\forall a \in \mathbb{Z}\), let \(\tau_a:=\min\{n \in \mathbb{N}:S_n=a \}\), then \(\mathbb{P}(\tau_a < \infty) = 1 \).
\end{problem}
\begin{solution}
  By the theorem 1.2.2 of textbook, it is obvious that \(P(\tau_a < \infty)=\lim_{b \to \infty}P_i(\tau_a<\tau_b)=\frac{b-i}{b-a}=1\).
\end{solution}

\end{document}
