%!Mode:: "TeX:UTF-8"
%!TEX encoding = UTF-8 Unicode
%!TEX TS-program = xelatex
\documentclass{ctexart}
\newif\ifpreface
%\prefacetrue
\input{../../../global/all}
\begin{document}
\large
\setlength{\baselineskip}{1.2em}
\ifpreface
\input{../../../global/preface}
\newgeometry{left=2cm,right=2cm,top=2cm,bottom=2cm}
\else
\newgeometry{left=2cm,right=2cm,top=2cm,bottom=2cm}
%\maketitle \fi
%from_here_to_type
%p59: 1, 2, 4; p61: 1, 2
\begin{problem}\label{pro:1}
  Find the solution of \(6x^3 + 27 x^2 + 17x + 20 \equiv 0 \pmod{ 30}\).
\end{problem}
\begin{solution}
  By observation, we know that \(x \equiv 2 \pmod{30}\) is one of the solution of \(6x^3 + 27 x^2 + 17 x + 20 \equiv 0 \pmod{30}\).
  Besides, \(30=2\times3\times5\), we consider all the solution of
  \[
    \begin{cases}
      6x^3 + 27x^2 + 17x + 20 & \equiv 0 \pmod{2} \\
      6x^3 + 27x^2 + 17x + 20 & \equiv 0 \pmod{3} \\
      6x^3 + 27x^2 + 17x + 20 & \equiv 0 \pmod{5}
    \end{cases}
  \]
  Since \(x \equiv 0 \pmod{2}, x \equiv 1 \pmod{2}\) are all the solution of \(6x^3 + 27x^2 + 17x + 20 \equiv 0 \pmod{2}\),
  \(x \equiv 2 \pmod{3}\) is all the solution of \(6x^3 + 27 x^2 + 17 x + 20 \equiv 0 \pmod{3}\),
  \(x \equiv 0 \pmod{5}, x \equiv 1 \pmod{5}, x \equiv 2 \pmod{5}\) are the solution of \(6x^3 + 27 x^2 + 17x + 20 \equiv 0 \pmod{5}\),
  and the solution of \(6x^3 + 27x^2 + 17x + 20 \equiv 20 \pmod{5}\) are at most \(3\),
  then \(x \equiv 0 \pmod{5}, x \equiv 1 \pmod{5}, x \equiv 2 \pmod{5}\) are all the solution of \(6x^3 + 27x^2 + 17x + 20 \equiv 0 \pmod{5}\).
  By Chinese Remainder Theorem, we get to know that \(x \equiv \sum_{i=1}^{3} M_i' M_i a_i \pmod{30}\) are all the
  solution of \(6x^3 + 27x^2 + 17x + 20 \equiv 0 \pmod{30}\), where \(M_1=15,M_2=10,M_3=6,M_1' = 1,M_2'= 1,M_3=1,a_1 \in \{0,1\},a_2
  =2, a_3 \in \{0,1,2\}\).
  Therefore, \(x \equiv 20 \pmod{30}, x \equiv 26 \pmod{30}, x \equiv 2 \pmod{30}, x \equiv 5 \pmod{30},
  x \equiv 17 \pmod{30}, x \equiv 11 \pmod{30}\) are all the solution of \(6x^3 + 27 x^2 + 17x + 20 \equiv 0 \pmod{30}\).
\end{solution}

\begin{problem}\label{pro:2}
  Find the solution of \(31x^4 + 57x^3 + 96x + 191 \equiv 0 \pmod{ 225}\).
\end{problem}
\begin{solution}
  Since \(225=3^2\times5^2\), firstly we consider \( \equiv 0 \pmod{15}\).
  \[
    \begin{cases}
      31x^4 + 57x^3 + 96x + 191 & \equiv 0 \pmod{3^2} \\
      31x^4 + 57x^3 + 96x + 191 & \equiv 0 \pmod{5^2}
    \end{cases}
  \]
  To find the solution of \(31x^4 + 57x^3 + 96x + 191 \equiv 0 \pmod{3^2}\), we consider \(31x^4 + 57x^3 + 96x + 191 \equiv 0 \pmod{3}\).
  By observation, we get to know the solution of \(31x^4 + 57x^3 + 96x + 191 \equiv 0 \pmod{3}\) are \(x \equiv 1,2 \pmod{3}\).
  Since \(f'(x)=124x^3 + 171x^2 + 96\), \(3 \nmid f'(1)=391, 3 \nmid f'(2)=1772\), then suppose \(f(1 + 3 k) \equiv 0 \pmod{3^2}
  f(2 + 3t) \equiv 0 \pmod{3^2}\),
  so \(f(1) + 3k f'(1) \equiv 0 \pmod{3^2}, f(2) + 3kf'(2) \equiv 0 \pmod{3^2}\), then \(125 + 391k \equiv 0 \mod 3,
  1335 + 5316t \equiv 0 \mod 9\), then \(k \equiv 1 \pmod{3},t \equiv 1 \pmod{3}\),
  then \(x \equiv 4, 5\mod 9\) are the solution of \(31x^4 + 57x^3 + 96x + 191 \equiv 0 \pmod{9}\).

  To find the solution of \(31x^4 + 57x^3 + 96x + 191 \equiv 0 \pmod{5^2}\), we consider \(31x^4 + 57x^3 + 96x + 191 \equiv 0 \pmod{5}\).
  By observation, we get to know the solution of \(31x^4 + 57x^3 + 96x + 191 \equiv 0 \pmod{3}\) are \(x \equiv 1,2 \pmod{5}\).
  Since \(f'(x)=124x^3 + 171x^2 + 96\), \(5 \nmid f'(1)=391, 5 \nmid f'(2)=1772\), then suppose \(f(1 + 5 k) \equiv 0 \pmod{5^2}
  f(2 + 5t) \equiv 0 \pmod{5^2}\),
  so \(f(1) + 5k f'(1) \equiv 0 \pmod{5^2}, f(2) + 5kf'(2) \equiv 0 \pmod{5^2}\), then \(75 + 391k \equiv 0 \mod 5,
  267 + 391t \equiv 0 \mod 5\), then \(k \equiv 0 \pmod{5},t \equiv 3 \pmod{5}\),
  then \(x \equiv 0, 17\mod 25\) are the solution of \(31x^4 + 57x^3 + 96x + 191 \equiv 0 \pmod{25}\).

  By Chinese Remainder Theorem, we get to know that \(x \equiv \sum_{i=1}^{2} M_i' M_i a_i \pmod{225}\),
  where \(M_1=25,M_2=9,M_1'=4,M_2'=14,a_1 \in \{4,5\}, a_2 \in \{0,17\}\), then \(x \equiv 76,67,50,167 \pmod{225}\).
\end{solution}
\begin{problem}\label{pro:3}
  Prove: \(5x^2 + 11y^2 \equiv 1 \pmod{ m}\) has solution for all \(m \in \mathbb{N}\).
\end{problem}
\begin{solution}
  First of all, we consider all of the intergal solution of \(5x^2 + 11 y^2 = z^2\).
  After calculating, we get \(x = 11s^2 - 22 st -5 t^2,y=-11s^2 -10st + 5t^2,c=20t^2 + 44s^2, s,t \in \mathbb{Z}\).
  Let \(t = 5, s= 3^{16} \prod_{p \in \mathbb{P},5 < p \leq m} p^{16}\), then \(x \equiv 11 -110-125 =-224 \equiv 0 \mod 32\),
  \(y \equiv -11-50 + 125 =64 \equiv 0 \mod 32\), then \(32^2 \mod z\).
  Let \(x_1=\frac{x}{32},y_1=\frac{y}{32},z_1=\frac{z}{32}\). If \(\gcd(m,z_1) \neq 1\), then \(\exists p \in \mathbb{P}\),
  \(p \mid \gcd(m,z_1)\). If \(p >5 \) or \(p =3\), then \(p \leq m\), then \(p \mid s\).
  Since \(p \mid z_1 \mid z\), then \(p \mid t\), then \(p=5\). Contradiction!
  If \(p=2\), then \(2 \mid \frac{z}{32}\), then \(16 \mid 5t^2 + 11s^2\).
  But \(5t^2 + 11s^2 \equiv 125 + 11 \equiv 8 \mod 16\), Contradiction.
  If \(p=5\), then \(5 \mid \frac{z}{32}\), then \(5 \mid z\).
  Since \(5 \mid 20t^2 \), then \(5 \mid 44s^2\), then \(5 \mid s^2\).
  But obviously, \(5 \mid s\). Contradiction!
  Thus \(\gcd(m,z_1)=1\). So \(\exists w\) such that \(w z_1 \equiv 1 \mod m\), then \(5(w x_1)^2 + 11 (w y_1)^2 = w^2z^2 \equiv 1 \mod m\).
\end{solution}

\begin{problem}\label{pro:4}
  If \(n \mid p-1, n > 1,(a,p)=1\), prove :
  \begin{enumerate}
    \item \(x^n \equiv a \pmod{ p}\) has solution \(\iff\) \(a^{\frac{p-1}{n}} \equiv 1 \pmod{ p}\).
    \item If \(x^n \equiv a \pmod{ p}\) has solution, then it has \(n\) solution.
  \end{enumerate}
\end{problem}
\begin{solution}
  \begin{enumerate}
    \item ``\(\implies\)'':Since \(\gcd(a,p)=1\), then \(\gcd(x,p)=1\). Then \(a^{\frac{p-1}{n}} \equiv x^{p-1} \equiv 1 \mod p\).
    \item ``\(\impliedby\)'':
  \end{enumerate}

\end{solution}

\begin{problem}\label{pro:5}
  \(n \in \mathbb{N}^+\), \(\gcd(a,m)=1\), \(x^n \equiv a \pmod{ m}\) has one solution \(x \equiv x_0 \pmod{ m}\).
  Prove all the solution of \(x^n \equiv a \pmod{ m}\) have the form of \(x \equiv yx_0 \pmod{ m}\),
  where \(y\) is the solution of \(y^n \equiv 1\pmod{ m}\).
\end{problem}
\begin{solution}
  \begin{enumerate}
    \item First, we prove that \(x \equiv yx_0 \pmod{m}\) is the solution of \(x^n \equiv a \pmod{m}\).
      Since \(x_0^n \equiv a \pmod{m}\), then \((yx_0)^n=y^nx_0^n \equiv x_0^n \equiv a \pmod{m}\).
    \item Second, we prove that the solution of \(x^n \equiv a \pmod{m}\) have the form of \(x \equiv yx_0 \pmod{m}\).
      Assume \(x\) is the solution of \(x^n \equiv a \pmod{m}\), then \(\gcd(x,m)=\gcd(x_0,m)=1\), then
      \(\exists b \) such that \(bx_0 \equiv 1 \pmod{m}\). So \((bx_0)^n \equiv 1 \pmod{m}\).
      Then \(b^n \equiv a^{-1} \pmod{m}\), then \((xb)^n \equiv 1 \pmod{m}\).
      Then \(x \equiv xbx_0 \equiv (xb)x_0 \pmod{m}\).
      So let \(y=xb\).
  \end{enumerate}
\end{solution}

\end{document}
