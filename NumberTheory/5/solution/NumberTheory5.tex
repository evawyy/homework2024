%!Mode:: "TeX:UTF-8"
%!TEX encoding = UTF-8 Unicode
%!TEX TS-program = xelatex
\documentclass{ctexart}
\newif\ifpreface
%\prefacetrue
\input{../../../global/all}
\newtheorem{theorem}{Theorem}
\begin{document}
\large
\setlength{\baselineskip}{1.2em}
\ifpreface
\input{../../../global/preface}
\newgeometry{left=2cm,right=2cm,top=2cm,bottom=2cm}
\else
\newgeometry{left=2cm,right=2cm,top=2cm,bottom=2cm}
%\maketitle \fi
%from_here_to_type
%p53.  1(ii);  2, 3.   p55. 1(ii), 3.
\begin{problem}\label{pro:1}
  Find all the solutions to \(1215x \equiv 560(\mod 2755)\).
\end{problem}
\begin{solution}
  To find all the solution of \(1215x \equiv 560 (\mod 2755)\) is equal to find all the solution of
  \(1215x-2755m=560\).
  Since \(\gcd(1215,2755,560)=5\), then it is equal to prove that \(243x-551m=112\).
  Since \(x=\frac{112 + 551m}{243}=\frac{112 + 65 m}{243} + 2m \in \mathbb{Z}\), then
  \(x_1=\frac{112 + 65m}{243} \in \mathbb{Z}\), then \(m=\frac{243x_1-112}{65}=3x_1-1+\frac{48x_1-47}{65} \in \mathbb{Z}\),
  then \(m_1=\frac{48x_1-47}{65} \in \mathbb{Z}\), then \(x_1=m_1 + 1 + \frac{17m_1 - 1}{48}\),
  then \(x_2=\frac{17m_1 - 1}{48} \in \mathbb{Z}\), then \(m_1=2x_2 + \frac{14x_2+1}{17}\),
  then \(m_2=\frac{14x_2-1}{17} \in \mathbb{Z}\), then \(14x_2-17m_2=1\),
  then \(x_2=\frac{17m_2 - 1}{14}=m_2 + \frac{3m_2 - 1}{14} \in \mathbb{Z}\), then \(x_3=\frac{3m_2 - 1}{14} \in \mathbb{Z}\),
  then \(m_2=\frac{14x_3 +1}{3}=\frac{2x_3+1}{3} + 4x_3 \in \mathbb{Z}\),
  then \(m_3=\frac{2x_3+1}{3} \in \mathbb{Z}\). Consider equation \(3m_3-2x_3=1\).
  obviously, \(x_3=1,m_3=1\) is a special solution of \(3m_3-2x_3=1\).
  So \(m_2=5, x_2=6,m_1=17,x_1=24,m=88,x=200\).
  Then the solution of \(243x-551m=112\) have the form of \(x=200 + 551t,m=88 + 243t,t \in \mathbb{Z}\),
  then the solution of \(1215x-2755m=560\) have the form of \(x=1000 + 2755t, m=440 + 1215t, t \in \mathbb{Z}\).
  Thus, the solution of \(1215x \equiv 560 (\mod 2755)\) have the form \(x=1000 + 2755t, t \in \mathbb{Z}\).
\end{solution}

\begin{problem}\label{pro:2}
  Find all the solution of \(x + 4y-29 \equiv 0 (\mod 143), 2x-9y + 84 \equiv 0 (\mod 143)\).
\end{problem}
\begin{solution}
  Since additive of remainder, then the solution of \(x + 4y-29 \equiv 0 (\mod 143), 2x-9y + 84 \equiv 0 (\mod 143)\),
  it is equal to find the solution of \(17y \equiv 142 (\mod 143), 17x + 75 \equiv 0 (\mod 143)\).
  So by the same method in Problem \ref{pro:1}, we can get \(x = 4 + 143 t_1, y = 42 + 143t_2, t_1,t_2 \in \mathbb{Z}\)
  satisfy the equation \(17 \equiv 142 ( \mod 143), 17x + 75 \equiv 0 (\mod 143)\).
  That is the solution of \(x + 4y - 29 \equiv 0 (\mod 143), 2x - 9y + 84 \equiv 0 ( \mod 143)\).
\end{solution}
\begin{problem}\label{pro:3}
  \(a,b,m \in \mathbb{Z}\), \(\gcd(a,m)=1\), then the solution of \(ax \equiv b (\mod m)\) have the
  form of \(x \equiv ba^{\phi(m)-1}(\mod m)\).
\end{problem}
\begin{solution}
  Obviously, the solution of \(x \equiv ba^{\phi(m)-1} (\mod m)\) satisfy \(ax \equiv ba^{\phi(m)} \equiv b (\mod m)\)
  by Fermet Theorem.

  Next, we will prove the solution of \(ax \equiv b(\mod m)\) have the form of \(x \equiv y (\mod m)\)
  when \(\gcd(a,m)=1\), where \(y \in \mathbb{N}\).
  To find all the solution of \(ax \equiv b (\mod m)\), it is equal to find all the \(x\) satisfy
  \(ax-mk=b, k \in \mathbb{Z}\). First we can consider \(ax-mk=1\). Since \(\gcd(a,m)=1 \mid b\), then
  the solution of \(ax-mk=1\) must exist.
  Assume \( x= x_0,k=k_0\) is the special solution of \(ax-mk=1\). Then \(x=x_0b,k=k_0b\) is
  the special solution of \(ax-mk=b\).
  Thus, the solution of \(ax-mk=b\) must have the form \(x =bx_0 + mt,k=k_{0} + mt,t \in \mathbb{Z}\).
\end{solution}
\begin{problem}\label{pro:4}
  Find \(x\) satisfy
  \begin{equation}\label{equ:4}
    \begin{cases}
      x & \equiv 1 \mod 2 \\
      x & \equiv 2 \mod 5 \\
      x & \equiv 3 \mod 7 \\
      x & \equiv 4 \mod 9
    \end{cases}
  \end{equation}
\end{problem}
\begin{solution}
  Let \(m_1=2,m_2=5,m_3=7,m_4=9,b_1=1,b_2=2,b_3=3,b_4=4\), then \(m=\prod_{i=1}^{4} m_i, M_i=\frac{m}{m_i},i=1,\cdots,4\),
  then \(M_i^{'}M_i \equiv 1 \mod m_i, i=1,\cdots,4\). So we can get \(M_1^{'} \equiv 1 \mod 2\),
  \(M_2^{'} \equiv 1 \mod 5\), \(M_3^{'} \equiv 4 \equiv 7\), \(M_4^{'} \equiv 4 \mod 9\).
  Then the solution of Equation \eqref{equ:4} is \(x \equiv \prod_{i=1}^{4} M_i^{'}M_ib_i \equiv m\),
  \(x \equiv 315 \times 1 \times 1 + 126 \times 1 \times 2 + 450 \times 3 \times 4 + 70 \times 4 \times 4 \equiv 7087 \equiv 157 \mod 630\).
\end{solution}
\begin{problem}\label{pro:5}
  \(b_i, m_i \in \mathbb{N}, i=1,\cdots,k\), satisfy \(\gcd(m_i,m_j) \mid b_i-b_j, i \neq j\).
  \(m_i^{'} = \prod_{p \in \mathbb{P}, \forall j < i, V_{p}(m_j) < V_p(m_i), \forall j \in \mathbb{N}^+, V_p(m_j)\leq V_p(m_i)}p^{V_p(m_i)}\),
  where \(\mathbb{P} \subset \mathbb{N}\) is the set of all prime, \(V_p(m)=\sup \{t:p^t \mid m\}\).
  Prove:
  \begin{equation}\label{equ:5.1}
    \begin{cases}
      x & \equiv b_1 \mod m_1 \\
        & \cdots              \\
      x & \equiv b_k \mod m_k
    \end{cases}
  \end{equation}
  and
  \begin{equation}\label{equ:5.2}
    \begin{cases}
      x & \equiv b_1 \mod m_1^{'} \\
        & \cdots                  \\
      x & \equiv b_k \mod m_k^{'}
    \end{cases}
  \end{equation}
  have same solutions.
\end{problem}
\begin{solution}
  By the definition of \(m_i^{'}\), we get \(m_i^{'} \mid m\), \(\gcd(m_i^{'},m_j^{'})=1, i \neq j\).
  Then Equation \eqref{equ:5.2} must have solution, so do Equation \eqref{equ:5.1}.
  And the solution of Equation \eqref{equ:5.1}
  must be the solution of Equation \eqref{equ:5.2}. So we only need to prove the solution of Equation \eqref{equ:5.2}
  is the solution of Equation \eqref{equ:5.1}. That means we only need to prove \(\forall i =1,\cdots,k\),
  \(x \equiv b_i \mod m_i^{'}\) must satisfy \(x \equiv b_i \mod m_i\).
  That is \(m \mid x-b_i\), so it is to prove \(p^{V_p(m_i)} \mid x-b_i, \forall p \in \mathbb{P}\).
  Let \(j = \min \{t: V_p(m_t)= \max \{V_p(m_s): s=1,\cdots,k\}\}\),
  then \(x \equiv b_j \mod p^{V_p(m_j)}\). Obviously, \(p^{V_p(m_j)} \mid m_j^{'}, m_j^{'} \mid m_j \), then \(x \equiv b_j \mod m_j^{'}\).
  Then \(x \equiv b_j \mod m_j\). And \(\gcd(m_i,m_j) \mid b_i-b_j\), then \(b_i \equiv b_j \mod \gcd(m_i,m_j)\).
  Then \(x \equiv b_i \mod \gcd(m_i,m_j)\), then \(x \equiv b_i \mod m_i\).
\end{solution}
\begin{theorem} 
  \(f(x) \in \mathbb{Z}[x], p\) is prime, if \(f(x_1) \equiv 0 \mod p\) and \(p \nmid f'(x_1)\), then
  \(\forall \alpha \in \mathbb{N}^+, \exists x_\alpha: x_\alpha \equiv x_1 \mod p\), 
  \(x \equiv x_\alpha \mod p^{\alpha}\) is one of the solution of
  \(f(x) \equiv 0 \mod p^{\alpha}\).
\end{theorem}
\begin{proof}
  \begin{enumerate}
    \item When \(\alpha = 1\), then by \(x \equiv x_1 \mod p\), then \(\)\(f(x) \equiv f(x_1) \mod p\).
    \item When \(a = \alpha -1\), we have \(x_{a} \equiv x_1 \mod p\) is one of the solution of \(f(x) \equiv 0 \mod p^{a}\).
      Next we will prove \(x_{\alpha} \equiv x_1 \mod p\) is one of the solution of \(f(x) \equiv 0 \mod p^{\alpha}\).


  \end{enumerate}
\end{proof}

\end{document}
