%!Mode:: "TeX:UTF-8"
%!TEX encoding = UTF-8 Unicode
%!TEX TS-program = xelatex
\documentclass{ctexart}
\newif\ifpreface
%\prefacetrue
\input{../../../global/all}
\begin{document}
\large
\setlength{\baselineskip}{1.2em}
\ifpreface
\input{../../../global/preface}
\newgeometry{left=2cm,right=2cm,top=2cm,bottom=2cm}
\else
\newgeometry{left=2cm,right=2cm,top=2cm,bottom=2cm}
%\maketitle \fi
%from_here_to_type

\begin{problem}\label{pro:1}
  When \(p\) is prime, \(p > 2, A \mid p^\alpha\), find all the solution of \(y^2 \equiv A \pmod{p^\alpha}\).
\end{problem}
\begin{solution}
  Since \(A \mid p^{\alpha}\), then it is equal to find the solution of \(y^2 \equiv 0 \pmod{p^\alpha}\).
  Next, we will prove that the solution of \(y^2 \equiv 0 \pmod{p^\alpha}\) are \(\{y \in \mathbb{Z}: V_p(y) \geq \frac{\alpha }{2}\}\).

  Let \(y = \prod_{r \in P}r^{V_r(y)}\), where \(P\) is all the prime, \(V_r(n)=\min \{k \in \mathbb{N}:r^k \mid n \},r \in P, n \in \mathbb{Z}\).
  If \(p^\alpha \mid y^2 = \prod_{r \in P}r^{2V_r(y)}\), then \(V_p(y) \geq 1\) and \(\alpha \mid 2 V_p(y)\).
  So \(\frac{\alpha }{2}\leq V_p(y)\).

  And obviously, \(\forall y:V_p(y) \geq \frac{\alpha }{2}\), then \(V_p(y^2) =2V_p(y) \geq \alpha\), then \(p^\alpha \mid y^2\).
\end{solution}
\begin{problem}\label{pro:2}
  Prove:
  \[
    ax^2 + bx + c \equiv 0 \pmod{m},\gcd(2a,m)=1
  \]
  has solution.
  \(\iff\)
  \[
    x^2 \equiv q \pmod{m},q=b^2 -4ac
  \]
  has solutions, which can infer the solution of \(ax^2 + bx + c \equiv 0 \pmod{m}\).
\end{problem}
\begin{solution}
  Since \(\gcd(2a,m)=1\), then \(2 \nmid m, a \nmid m\), then \(\gcd(4a,m)=1\).
  So the solution of \(ax^2 + bx + c \equiv 0 \pmod{m}\) \(\iff\) it is the solution of \((2ax + b)^2 + (4ac -b^2) \equiv 0 \pmod{m}\)
  \(\implies\) \(y^2 + 4ac-b^2 \equiv 0 \pmod{m}\), where \(y \equiv 2ax + b \pmod{m}\).
  Since \(\gcd(2a,m)=1\), then the solution of \(y^2 + 4ac-b^2 \equiv 0 \pmod{m}\) \(y\), we
  let \(x \equiv A(y-b) \pmod{m}\), where \(A(2a) \equiv 1 \pmod{m}\),
  \(x\) is the solution of \((2ax + b)^2 + (4ac -b^2) \equiv 0 \pmod{m}\).

\end{solution}
\begin{problem}\label{pro:3}
  Find out all the squared remainder and non squared remainder of \(37\).
\end{problem}
\begin{solution}
  By the Theorem 2 on page 65 of text book, we can get that \(\{k^2 + 37t:1 \leq k \leq 18, t \in \mathbb{Z}\}=\{k + 37 t: t \in \mathbb{Z}, k \in A\}\),
  where \(A:=\{1,4,9,16,25,36,12,27,7,26,10,33,21,11,3,34,30,28\}\) are squared remainder.
  And \(\{k + 37t: t \in \mathbb{Z}, k \in B\}\), where \(B=\mathbb{N}^+ \cap[0,36] \setminus A\) are non
  squared remainder.
\end{solution}
\begin{problem}\label{pro:4}
  \begin{enumerate}
    \item Use the conclusion in the formar chapters, prove: there must exist quadratic residue and
      non quadratic residue in the reduced residue system of \(p\).
    \item Assume \(x_1,x_2\) are quadratic residues, \(X_3\) is non quadratic residue:
      prove \(x_1x_2\) is quadratic residue, \(x_1x_3\) is non quadratic residue.
    \item Apply the conclusions above, prove that both the quadratic residue and the non quadratic residue
      in the reduced residue system of \(p\) have
      \(\frac{p-1}{2}\) elements.
  \end{enumerate}
\end{problem}
\begin{problem}\label{pro:5}
  Prove: the solution of \(x^2 \equiv a\pmod{p^\alpha},\gcd(\alpha,p)=1\) is \(x \equiv \pm PQ'\pmod{p^\alpha}\), where \[
    P=\frac{(z + \sqrt{\alpha})^\alpha + (z - \sqrt{\alpha})^\alpha}{2}, Q=\frac{(z + \sqrt{\alpha})^\alpha - (z-\sqrt{\alpha})^\alpha}{\sqrt{\alpha}},
  \]
  \[
    z^2\equiv \alpha\pmod{p}, QQ' \equiv 1 \pmod{p^\alpha}.
  \]
\end{problem}
\begin{problem}\label{pro:6}
  Prove the solution of \(x^2 + 1 \equiv 0 \pmod{p},p=4m + 1\) is \(x \equiv \pm 1 \cdot 2 \cdot \cdots \cdot (2m)\pmod{p}\).
\end{problem}

\end{document}
