%!Mode:: "TeX:UTF-8"
%!TEX encoding = UTF-8 Unicode
%!TEX TS-program = xelatex
\documentclass{ctexart}
\newif\ifpreface
%\prefacetrue
\input{../../../global/all}
\begin{document}
\large
\setlength{\baselineskip}{1.2em}
\ifpreface
\input{../../../global/preface}
\newgeometry{left=2cm,right=2cm,top=2cm,bottom=2cm}
\else
\newgeometry{left=2cm,right=2cm,top=2cm,bottom=2cm}
%\maketitle \fi
%from_here_to_type
\begin{problem}\label{pro:1}
  Assume \(A=\{a \in P \mid a \mid m\} = \{q_i \mid i=1,\cdots,s\}\), where \(P \subset \mathbb{N}\), \(\forall p \in P\), \(p\) is prime,
  \(s = |A|\).
  Prove: \(g\) is the primative root mod \(m\) \(\iff\)
  \(g\) is \(q_i\)-tic non-residue mod \(m\), \(\forall i=1,\cdots,s\).
\end{problem}
\begin{problem}\label{pro:2}
  Prove:
  \begin{enumerate}
    \item \(10\) is the primative root mod \(17,257\).
    \item The length of repetend of \(\frac{1}{17}\) is \(16\), the length of repetend of \(\frac{1}{257}\) is \(256\).
  \end{enumerate}
\end{problem}
\begin{problem}\label{pro:3}
  Apply index table to solve the equation \[
    x^{15 } \equiv 14 \pmod{41}.
  \]
\end{problem}
\begin{problem}\label{pro:4}
  Assume \(m >2\) has primative root, prove \(\forall g\), \(g\) is the primative root mod \(m\),
  the index of \(-1\) is \(\frac{1}{2}\phi(m)\).
\end{problem}
\begin{problem}\label{pro:5}
  Assume \(g_1,g_2\) are two primative root mod \(m\), prove:
  \begin{enumerate}
    \item \(\ind_{g_1}g \cdot \ind_{g}g_1 \equiv 1 \pmod{\phi(m)}\);
    \item \(\ind_g a \equiv \ind_g g_1 \cdot\ind_{g_1}a \pmod{\phi(m)}\)
  \end{enumerate}

\end{problem}

\end{document}
