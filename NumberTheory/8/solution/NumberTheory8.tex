%!Mode:: "TeX:UTF-8"
%!TEX encoding = UTF-8 Unicode
%!TEX TS-program = xelatex
\documentclass{ctexart}
\newif\ifpreface
%\prefacetrue
\input{../../../global/all}
\begin{document}
\large
\setlength{\baselineskip}{1.2em}
\ifpreface
\input{../../../global/preface}
\newgeometry{left=2cm,right=2cm,top=2cm,bottom=2cm}
\else
\newgeometry{left=2cm,right=2cm,top=2cm,bottom=2cm}
%\maketitle \fi
%from_here_to_type
\begin{problem}\label{pro:1}
  Use the method in the contexts of this section to judge whether these equations below have solutions.
  \begin{enumerate}
    \item \(x^2 \equiv 429 \pmod{563}\)
    \item \(x^2 \equiv 680 \pmod{769}\)
    \item \(x^2 \equiv 503 \pmod{1013}\)
  \end{enumerate}
  where \(503,563,796,1013\) are prime.
\end{problem}
\begin{solution}
  \begin{enumerate}
    \item \(\legendre{429}{563}=\legendre{3}{563}\legendre{11}{563}\legendre{13}{563}=(-1)^{\frac{(2 + 10 + 12)\times 562}{4}}\legendre{563}{3}\legendre{563}{11}\legendre{563}{13}=\legendre{2}{3}\legendre{2}{11}\legendre{4}{13}=(-1)(-1)^{\frac{11^2-1}{8}}(1)=1\).
    \item \(\legendre{680}{769}=\legendre{170}{769}=\legendre{2}{769}\legendre{5}{769}\legendre{17}{769}=(-1)^{\frac{769^2-1}{8} + \frac{(4 + 16)768}{4}}\legendre{769}{5}\legendre{769}{17}=\legendre{4}{5}\legendre{4}{17}=1\).
    \item \(\legendre{503}{1013}=(-1)^{\frac{502 \times 1012}{4}}\legendre{1013}{503}=\legendre{7}{503}=(-1)^{\frac{6 \times 502}{4}}\legendre{503}{7}=-\legendre{6}{7}=-\legendre{-1}{7}=-(-1)^3=1\).
  \end{enumerate}
\end{solution}

\begin{problem}\label{pro:2}
  Find out the expression of the prime with the quadratic residue \(-2\);
  Find out the expression of the prime with the non quadratic residue \(-2\);
\end{problem}
\begin{solution}
  Easy to get that \(\legendre{-2}{p}=\legendre{-1}{p}\legendre{2}{p}=(-1)^{\frac{p-1}{2}}(-1)^{\frac{p^2-1}{8}}\).
  So \(\legendre{-2}{p}=1 \iff 16 \mid (p-1)(p + 5) \iff 4 \mid \frac{p-1}{2} \frac{p + 5}{2}\). Since \(\frac{p + 5}{2}-\frac{p-1}{2}=3 \equiv 1 \pmod{2}\), we know
  that they can't all be even. So \(4 \mid \frac{p-1}{4} \OR 4 \mid \frac{p + 5}{2}\), so \(p \equiv 1,3 \pmod{8}\).
  So \(\legendre{-2}{p}=1 \iff p \equiv 1,3\pmod{8}\), and \(\legendre{-2}{p}=-1 \iff p \equiv 5,7 \pmod{8}\).
\end{solution}

\begin{problem}\label{pro:3}
  Assume \(n \in \mathbb{N}_+\), \(4n + 3,8n + 7\) are prime, prove:
  \[
    2^{4n + 3}\equiv 1 \pmod{8n + 7}
  \]
  Then prove \(23 \mid(2^{11}-1),47 \mid (2^{23}-1),503 \mid (2^{251}-1)\).
\end{problem}
\begin{solution}
  In fact we don't need \(4n + 3\) is prime. Since \(2^{\frac{8n + 7-1}{2}}\legendre{2}{8n + 7}=(-1)^{\frac{(8n + 7)^2-1}{8}}=(-1)^{8n^2 + 14n + 6}=1\),
  we easily get that \(2^{4n + 3} \equiv 1 \pmod{8n + 7}\). We let \(n = 2,5,62\), then \(23 \mid (2^{11}-1), 47 \mid (2^{23}-1),503 \mid(2^{251}-1)\).
\end{solution}

\begin{problem}\label{pro:4}
  Find out the expression of the prime with the quadratic residue \(\pm 3\);
  which prime has the non quadratic residue \(\pm 3\)?
\end{problem}
\begin{solution}
  Assume \(p > 3\). Easy to get that \(\legendre{3}{p}=(-1)^{\frac{p-1}{2}}\legendre{p}{3}\).
  And \(\legendre{-3}{p}=\legendre{p}{3}\). So \(\legendre{-3}{p}=1 \iff p \equiv 1 \pmod{3}, \legendre{-3}{p}=-1 \iff p \equiv 2 \pmod{3}\).
  And \(\legendre{3}{p}=1 \iff (p \equiv 1 \pmod{3} \AND p \equiv 1 \pmod{ 4}) \OR (p \equiv 2 \pmod{3} \AND p \equiv 3 \pmod{ 4}) \iff p \equiv 1,11 \pmod{ 12}\).
  Then \(\legendre{3}{p}=1 \iff p \equiv 1,11 \pmod{12}, \legendre{3}{p}=-1 \iff p \equiv 5,7 \pmod{12}\).
\end{solution}

\begin{problem}\label{pro:5}
  Find out the expression of the prime with the minimum non quadratic residue \( 3\).
\end{problem}
\begin{solution}
  Only need to solve \(\legendre{2}{p}=1 \AND \legendre{3}{p}=-1\). Easy to ge that \(\legendre{2}{p}=1 \iff p \equiv 1,7 \pmod{8}\).
  Since \(\legendre{3}{p}=1 \iff p \equiv 1, 11 \pmod{12}\). So finally we get that \(p \equiv 1,23 \pmod{24}\).
  So \(p \in \mathbb{P}\) with minimum non-quadratic \(3 \iff p \equiv 1,23 \pmod{24}\).
\end{solution}
\end{document}
