%!Mode:: "TeX:UTF-8"
%!TEX encoding = UTF-8 Unicode
%!TEX TS-program = xelatex
\documentclass{ctexart}
\newif\ifpreface
%\prefacetrue
\input{../../../global/all}
\begin{document}
\large
\setlength{\baselineskip}{1.2em}
\ifpreface
\input{../../../global/preface}
\newgeometry{left=2cm,right=2cm,top=2cm,bottom=2cm}
\else
\newgeometry{left=2cm,right=2cm,top=2cm,bottom=2cm}
%\maketitle \fi
%from_here_to_type
\begin{problem}\label{pro:1}
  Find the number of all the intergal solution of equations as follow:
  \begin{enumerate}
    \item \(x^2 \equiv 3766 \pmod{5987}\);
    \item \(x^2 \equiv 3149 \pmod{5987}\).
      Where \(5987\) is a prime.
  \end{enumerate}
\end{problem}

\begin{solution}
  \begin{enumerate}
    \item
      \begin{equation}
        \begin{aligned}
           & \legendre{3766}{5987}=\legendre{2^3}{5987}\legendre{471}{5987}=\legendre{2}{5987}(-1)^{\frac{5986}{2} \frac{470}{2}}\legendre{5987}{471} \\
           & =(-1)^{\frac{5987^2-1}{8}} (-1)\legendre{5987}{471}=\legendre{-136}{471}=\legendre{2}{471}\legendre{17}{471}\legendre{-1}{471}           \\
           & =(-1)^{\frac{471^2-1}{8}}(-1)^{\frac{471-1}{2} \frac{17-1}{2}}\legendre{471}{17}(-1)^{\frac{471-1}{2}}                                   \\
           & =-\legendre{-5}{17}=-(-1)^{16}\legendre{5}{17}=-\legendre{17}{5}=-\legendre{2}{5}=-(-1)^{\frac{5^2-1}{8}}=-(-1)^3=1
        \end{aligned}
      \end{equation}
      Since \(5987\) is prime, then \(x^2 \equiv 3766 \pmod{5987}\) has \(2\) solutions.
    \item
      \begin{equation}
        \begin{aligned}
           & \legendre{3149}{5987}=\legendre{-311}{5987}=\legendre{-1}{5987}\legendre{311}{5987}=(-1)^{\frac{5987-1}{2}}\legendre{78}{311}                                                        \\
           & =-\legendre{2}{311}\legendre{3}{311}\legendre{13}{311}=-(-1)^{\frac{311^2-1}{8}}(-1)^{\frac{310}{2} \frac{2}{2}}\legendre{311}{3}(-1)^{\frac{310}{2} \frac{12}{2}}\legendre{311}{13} \\
           & =\legendre{2}{3}(-1)\legendre{-1}{13}=(-1)^{\frac{3^2-1}{8}}(-1)(-1)^{\frac{13-1}{2}}=1
        \end{aligned}
      \end{equation}
      Since \(5987\) is prime, then \(x^2 \equiv 3149 \pmod{5987}\) has \(2\) solutions.
  \end{enumerate}
\end{solution}

\begin{problem}\label{pro:2}
  \begin{enumerate}
    \item When the equation has solutions, apply theorm 1 in section 2 to
      find the solution of \(x^2 \equiv a \pmod{p}, p=4m + 3\).
    \item When the equation has solutions, apply theorem 1 in section 2 and section 3 to
      find the solution of \(x^2 \equiv a \pmod{p}, p=8m + 5\).
    \item If the equation \(x^2 \equiv a \pmod{p},p=8m + 1\) has solutions, and \(N \) is non quadratic residue.
      Give one way to solve the equation ablow.
  \end{enumerate}
\end{problem}
\begin{solution}
  \begin{enumerate}
    \item Since the equation has solution, we know that \(a^{\frac{p-1}{2}}\equiv 1 \mod p\).
      So \(a^{2m+1} \equiv 1 \mod p\). So \(a^{2m+2}\equiv a \mod p\).
      So \((a^{m+1})^2 \equiv a \mod p\). So the solution is \(x \equiv \pm a^{m+1} \mod p\).
    \item Since the equation has solution, we know that \(a^{\frac{p-1}{2}} \equiv 1 \mod p\), then \(a^{4m+2} \equiv 1 \mod p\).
      So \(a^{2m+1} \equiv \pm 1 \mod p\).
      If \(a^{2m+1}\equiv 1 \mod p\), then we have \((a^{m+1})^2 \equiv a \mod p\), so \(x \equiv \pm a^{m+1} \mod p\).
      Else, since \(\legendre{2}{p}=(-1)^{\frac{p^2-1}{8}}=-1\), we have \(2^{4m+2} \equiv -1 \mod p\).
      So \(2^{4m+2} a^{2m+2}\equiv a \mod p\). So \(x \equiv \pm 2^{2m+1}a^{m+1} \mod p\).
    \item For the same reason, we easily get that \(a^{4m} \equiv 1 \mod p\) and \(N^{4m} \equiv -1 \mod p\).
      We can find the solution by following method:
      \begin{enumerate}
        \item let \(x=4m,y=0\).
        \item \label{it:cir}if \(2 \nmid x\), goto \ref{it:out}.
        \item If \(a^{\frac{x}{2}}N^{\frac{y}{2}} \equiv 1 \mod p\), then let \(x=\frac{x}{2},y=\frac{y}{2}\).
          If \(a^{\frac{x}{2}}N^{\frac{y}{2}} \equiv -1 \mod p\), then let \(x=\frac{x}{2},y=\frac{3y}{2}\).
        \item goto \ref{it:cir}.
        \item \label{it:out} Now we have \(2 \nmid x,2 \mid y,a^x N^y \equiv 1 \mod p\).
          So \(x \equiv a^{\frac{x+1}{2}}N^{\frac{y}{2}} \mod p\).
      \end{enumerate}
      It is easy to prove that this method can end because every turn the calue of \(v_2(x)\) will \(-1\).
      And easy to prove that \(2 \mid y\) because we can use MI to prove that \(v_2(y)>v_2(x)\).
  \end{enumerate}
\end{solution}

\begin{problem}\label{pro:3}
  Solve the equation
  \begin{enumerate}
    \item\( x^2  \equiv 59 \pmod{125} \)
    \item \(    x^2  \equiv 41 \pmod{64}\)
  \end{enumerate}
\end{problem}
\begin{solution}
  \begin{enumerate}
    \item First solve \(x^2 \equiv 4 \mod 5\). Solution is \(x \equiv \pm 2 \mod 5\).
      Second solve \(x^2 \equiv 9 \mod 25\), assume \(x=5y \pm 2\), easy to get that \(x \equiv \pm 3 \mod 25\).
      Finally solve \(x^2 \equiv 59 \mod 125\) and assume \(x = 25y \pm 3\). Easily \(x \equiv \pm 53 \mod 125\).
      So the solution is \(x \equiv \pm 53 \mod 125\).
    \item Easy to find that \(x \equiv \pm 13,\pm 19 \mod 64\).
  \end{enumerate}
\end{solution}

\begin{problem}\label{pro:4}
  \begin{enumerate}
    \item \label{ite:4.1} Prove equation \(x^2 \equiv 1 \pmod{m}\) and \((x + 1)(x-1) \equiv 0 \pmod{m}\) are equal.
    \item Apply \ref{ite:4.1} to give one way of finding all the solutions of \(x^2 \equiv 1 \pmod{m}\).
  \end{enumerate}
\end{problem}
\begin{solution}
  \begin{enumerate}
    \item Obviously because \(x^2-1=(x+1)(x-1)\).
    \item We can solve the equation by this way:
      \begin{enumerate}
        \item Dissolve \(m\) into product of primes, write \(m=2^{\alpha}\prod_{i=1}^{n} p_i^{\alpha_i}\).
        \item For \(p_i^{\alpha_i}\), easy to get that solution of \(x^2 \equiv 1 \mod p_i^{\alpha_i}\) is \(x \equiv \pm 1 \mod p_i^{\alpha_i}\).
        \item For \(2^\alpha\), if \(\alpha \geq 1\), we need to find solution of \(x^2 \equiv 1 \mod 2^\alpha\).
          When \(\alpha=1\), the solution is \(x \equiv 1 \mod 2\).
          When \(\alpha=2\), the solution is \(x \equiv 1,3 \mod 4\).
          When \(\alpha \geq 3\), the solution is \(x \equiv \pm 1,\pm (2^{\alpha-1}+1) \mod 2^\alpha\).
        \item Use Chinese Reminder Theorem to find all the solution of \(x^2 \equiv 1 \mod m\).
      \end{enumerate}
  \end{enumerate}
\end{solution}

\end{document}
