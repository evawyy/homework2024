%!Mode:: "TeX:UTF-8"
%!TEX encoding = UTF-8 Unicode
%!TEX TS-program = xelatex
\documentclass{ctexart}
\newif\ifpreface
%\prefacetrue
\input{../../../global/all}
\begin{document}
\large
\setlength{\baselineskip}{1.2em}
\ifpreface
\input{../../../global/preface}
\newgeometry{left=2cm,right=2cm,top=2cm,bottom=2cm}
\else
\newgeometry{left=2cm,right=2cm,top=2cm,bottom=2cm}
%\maketitle \fi
%from_here_to_type
\begin{problem}\label{pro:1}
  Find the number of all the intergal solution of equations as follow:
  \begin{enumerate}
    \item \(x^2 \equiv 3766 \pmod{5987}\);
    \item \(x^2 \equiv 3149 \pmod{5987}\).
      Where \(5987\) is a prime.
  \end{enumerate}
\end{problem}
\begin{solution}
\begin{enumerate}
  \item
    \begin{equation}
      \begin{aligned}
      &\legendre{3766}{5987}=\legendre{2^3}{5987}\legendre{471}{5987}=\legendre{2}{5987}(-1)^{\frac{5986}{2} \frac{470}{2}}\legendre{5987}{471}\\ 
      &=(-1)^{\frac{5987^2-1}{8}} (-1)\legendre{5987}{471}=\legendre{-136}{471}=\legendre{2}{471}\legendre{17}{471}\legendre{-1}{471}\\ 
      &=(-1)^{\frac{471^2-1}{8}}(-1)^{\frac{471-1}{2} \frac{17-1}{2}}\legendre{471}{17}(-1)^{\frac{471-1}{2}}\\ 
      &=-\legendre{-5}{17}=-(-1)^{16}\legendre{5}{17}=-\legendre{17}{5}=-\legendre{2}{5}=-(-1)^{\frac{5^2-1}{8}}=-(-1)^3=1
    \end{aligned}
    \end{equation}
    Since \(5987\) is prime, then \(x^2 \equiv 3766 \pmod{5987}\) has solutions.
  \item 
    \begin{equation} 
      \begin{aligned}
        &\legendre{3149}{5987}=\legendre{-311}{5987}=\legendre{-1}{5987}\legendre{311}{5987}=(-1)^{\frac{5987-1}{2}}\legendre{78}{311}\\ 
        &=-\legendre{2}{311}\legendre{3}{311}\legendre{13}{311}=-(-1)^{\frac{311^2-1}{8}}(-1)^{\frac{310}{2} \frac{2}{2}}\legendre{311}{3}(-1)^{\frac{310}{2} \frac{12}{2}}\legendre{311}{13}\\ 
        &=\legendre{2}{3}(-1)\legendre{-1}{13}=(-1)^{\frac{3^2-1}{8}}(-1)(-1)^{\frac{13-1}{2}}=1
      \end{aligned}
    \end{equation}
    Since \(5987\) is prime, then \(x^2 \equiv 3149 \pmod{5987}\) has solutions.
\end{enumerate}
  
\end{solution}

\begin{problem}\label{pro:2}
  \begin{enumerate}
    \item When the equation has solutions, apply theorm 1 in section 2 to
      find the solution of \(x^2 \equiv a \pmod{p}, p=4m + 3\).
    \item When the equation has solutions, apply theorem 1 in section 2 and section 3 to
      find the solution of \(x^2 \equiv a \pmod{p}, p=8m + 5\).
    \item If the equation \(x^2 \equiv a \pmod{p},p=8m + 1\) has solutions, and \(N \) is non quadratic residue.
      Give one way to solve the equation ablow.
  \end{enumerate}
\end{problem}
\begin{problem}\label{pro:3}
  Solve the equation \(\begin{cases}
    x^2 & \equiv 59 \pmod{125} \\
    x^2 & \equiv 41 \pmod{64}
  \end{cases}\).
\end{problem}
\begin{problem}\label{pro:4}
  \begin{enumerate}
    \item \label{ite:4.1} Prove equation \(x^2 \equiv 1 \pmod{m}\) and \((x + 1)(x-1) \equiv 0 \pmod{m}\) are equal.
    \item Apply \ref{ite:4.1} to give one way of finding all the solutions of \(x^2 \equiv 1 \pmod{m}\).
  \end{enumerate}
\end{problem}

\end{document}
