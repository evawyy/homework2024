%!Mode:: "TeX:UTF-8"
%!TEX encoding = UTF-8 Unicode
%!TEX TS-program = xelatex
\documentclass{ctexart}
\newif\ifpreface
%\prefacetrue
\input{../../../global/all}
\begin{document}
\large
\setlength{\baselineskip}{1.2em}
\ifpreface
  \input{../../../global/preface}
  \newgeometry{left=2cm,right=2cm,top=2cm,bottom=2cm}
\else
  \newgeometry{left=2cm,right=2cm,top=2cm,bottom=2cm}
  \maketitle \fi
%from_here_to_type
\begin{problem}\label{pro:1}
  Assume \(p,q\) are odd primes, \(a > 1\) is integal.
  Prove:
  \begin{enumerate}
    \item \(q \mid a^p-1\) \(\implies\) \(q \mid a-1\) or \(2p \mid q- 1\).
    \item \(q \mid a^p + 1\) \(\implies\) \(q \mid a +1\) or \(2p \mid q- 1\).
  \end{enumerate}
\end{problem}
\begin{solution}
  \begin{enumerate}
    \item Consider \(a \in \mathbb{Z}_q^{*}\). Since \(a^{p}-1 =0 \) in \(\mathbb{Z}_q^{*}\), then \(o(a) \mid p\).
      So \(o(a)=1\) or \(o(a)=p\).
      If \(o(a)=1\), then \(a \equiv 1 \pmod{q}\), then \(q \mid a-1\).
      If \(o(a)=p\), then \(p=o(a) \mid o(\mathbb{Z}_q^{*})=q-1\).
      Since \(2 \mid q-1\), \((p,2)=1\), then \(2p \mid q-1\).
    \item Consider \(-a \in \mathbb{Z}_q^{*}\). Since \((-a)^p-1 =0 \) in \(\mathbb{Z}_q^{*}\), then \(o(-a) \mid p\).
      So \(o(-a)=1\) or \(o(-a)=p\).
      If \(o(-a)=1\), then \(-a \equiv 1 \pmod{q}\), then \(q \mid a +1\).
      If \(o(-a)=p\), then \(p=o(-a) \mid o(\mathbb{Z}_q^{*})=q-1\).
      Since \(2 \mid q-1\), \((p,2)=1\), then \(2p \mid q-1\).
  \end{enumerate}
\end{solution}

\begin{problem}\label{pro:2}
  Find primitive root for each number \(7,49,343,686\).
\end{problem}
\begin{solution}
  \begin{enumerate}
    \item Obviously, \(3\) is the primitive root of \(7\).
    \item \(3\) is the primitive root of \(49\).
    \item \(3\) is the primitive root of \(343\).
    \item Since the primitive root of \(686\) is the odd one of \(3, 3 + 343\), then \(3\) is
      the primitive root of \(686\).
  \end{enumerate}

\end{solution}

\end{document}
