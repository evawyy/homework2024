%!Mode:: "TeX:UTF-8"
%!TEX encoding = UTF-8 Unicode
%arara: xelatex
%!TEX language=zh
\documentclass{ctexart}
\newif\ifpreface
%\prefacetrue
\input{../../../global/all}
\begin{document}
\large
\setlength{\baselineskip}{1.2em}
\ifpreface
\input{../../../global/preface}
\newgeometry{left=2cm,right=2cm,top=2cm,bottom=2cm}
\else
\newgeometry{left=2cm,right=2cm,top=2cm,bottom=2cm}
\maketitle
\fi
%from_here_to_type
\begin{problem}\label{pro:1.4.2}
  证明\(\sigma \)-代数是集代数。
\end{problem}
\begin{solution}
  假定\(\mathcal{A}\)是\(\sigma\)-代数,那么\(\Omega \in \mathcal{A} \),\(\mathcal{A}\)对补运算封闭。只需证明\(\mathcal{A}\)对有限并封闭。
  \(A,B \in \mathcal{A}\),令\(A_n=\varnothing=\Omega^c \in \mathcal{A}, n \geq 3\),那么\(A \cup B \cup \bigcup_{n \geq 3}A_n=A \cup B \in \mathcal{A}\)。
\end{solution}

\begin{problem}\label{pro:1.4.3}
  设\(\mathcal{C} \)是集类,则\(\forall A \in \sigma(\mathcal{C}),\exists \mathcal{C}_1 \subset \mathcal{C},|\mathcal{C}_1| \leq \aleph_0,A \in \sigma(\mathcal{C}_1) \)。
\end{problem}
\begin{solution}
  令\(\mathcal{A}:=\{A \in \sigma(\mathcal{C}):\exists \mathcal{C}_1 \subset \mathcal{C}, |\mathcal{C}_1| \leq \aleph_0, A \in \sigma(\mathcal{C}_1) \}\).
  下证\(\mathcal{A} \supset \sigma (\mathcal{C}) \),即证\(\mathcal{A} \)为包含\(\mathcal{C} \)的\(\sigma \)-代数。
  \begin{itemize}
    \item 由于\(\forall A \in \mathcal{C} , \sigma(\{A\})=\{\varnothing,\Omega,A,A^c\} \),那么\(\mathcal{C}  \subset \mathcal{A} \)。
    \item 由\(\Omega \in \sigma(\varnothing)=\{\varnothing,\Omega\} \),那么\(\Omega \in \mathcal{A} \)。
    \item 设\(A \in \mathcal{A} \),那么\(\exists \mathcal{C}_1 \subset \mathcal{C}, |\mathcal{C}_1|\leq \aleph_0\),\(A \in \sigma(\mathcal{C}_1) \).
      由于\(\sigma(\mathcal{C}_1) \)是\(\sigma \)-代数,那么\(A^c \in \sigma(\mathcal{C}_1) \).所以\(A^c \in \mathcal{A} \)。
    \item 设\(A_n \in \mathcal{A} ,n \in \mathbb{N}\),那么\(\exists \mathcal{C}_n \subset \mathcal{C} , |\mathcal{C}_n|\leq \aleph_0,n \in \mathbb{N}   \),满足\(A_n \in \sigma(\mathcal{C}_n) \forall n \in \mathbb{N}\)。
      令\(\mathcal{T}=\bigcup_{n \in \mathbb{N}} \mathcal{C}_n \),由\(|\mathcal{C}_n|\leq \aleph_0 ,\mathcal{C}_n \subset \mathcal{C}\),可知\(|\mathcal{T}| \leq \aleph_0,\mathcal{T} \subset \mathcal{C} \),那么\(A_n \in \sigma(\mathcal{C}_n)\subset \sigma (\mathcal{T}), \forall n \in \mathbb{N}\)。
      所以\(\bigcup_{n \in \mathbb{N} }A_n \in \sigma(\mathcal{T}) \)。那么\(\bigcup_{n \in \mathbb{N}}A_n \in \mathcal{A} \)。
  \end{itemize}
 综上,\(\mathcal{A} \)为包含\(\mathcal{C} \)的\(\sigma \)-代数。故\(\mathcal{A} \supset \sigma(\mathcal{C}) \)。 
 又由于\(\mathcal{A} \subset \sigma(\mathcal{C}) \),故\(\mathcal{A}=\sigma(\mathcal{C}) \)。
从而,\(\forall A \in \sigma(\mathcal{C}),\exists \mathcal{C}_1 \subset \mathcal{C},|\mathcal{C}_1| \leq \aleph_0,A \in \sigma(\mathcal{C}_1) \)。
\end{solution}

\begin{problem}\label{pro:1.4.4}
  \(\sigma \)-代数\(\mathcal{A} \)称为可数生成的,如果存在可数的子集类\(\mathcal{C} \subset \mathcal{A} \)使\(\sigma(\mathcal{C})=\mathcal{A} \)。
  证明\(\mathcal{B}^d \)是可数生成的。
\end{problem}
\begin{solution}
  考虑\(\mathcal{A}:=\{B(p,r):p \in \mathbb{Q}^d, r \in \mathbb{Q}_{+}\} \),其中\(B(p,r)=\{x \in \mathbb{R}^d:\norm{x-p} < r\} \)。显然\(|\mathcal{A}|=\aleph_0 \)。
  下证\(\mathcal{B}^d=\sigma(\mathcal{A}) \)。
  令\(\mathcal{O}:=\{\mathcal{B}^d \text{中的开集}\} \),
  由于\(\mathcal{B}^d=\sigma(\mathcal{O}) \),那么\( \mathcal{A} \subset \mathcal{O} \),从而\(\sigma(\mathcal{A})\subset \sigma(\mathcal{O}) \)。
  只需证明\(\mathcal{O} \subset \sigma(\mathcal{A}) \)。\\ 
  \(\forall A \in \mathcal{O} \), \( \forall x \in A \),\(\exists U=B(x,s) \),\(x \in U \subset A \)。由于\(\mathbb{Q}^d \)在\(\mathbb{R}^d \)中稠密,故\(\exists p_x \in B(x,\frac{s}{2}) \cap \mathbb{Q}^d \)。
  取\(r_x \in \mathbb{Q}_{+} \)使\(\norm{x-p_x} <r_x<\frac{s}{2} \), 由 \(\forall y \in B(p_x,r_x) \),有\(\norm{y-x}\leq \norm{y-p_x}+\norm{p_x-x} < r_x+\frac{s}{2}<s \)得 \(B(p_x,r_x) \subset B(x,s)\subset A \),
    则有 \(\bigcup_{x \in A}B(p_x,r_x) \subset  A \)。
  显然\(x \in B(p_x,r_x) \),那么\(A \subset \bigcup_{x \in A}B(p_x,r_x) \),从而 \(\forall A \in \mathcal{O},\bigcup_{x \in A}B(p_x,r_x) = A \)。
  由于\(|\mathcal{A}|=\aleph_0 \),那么\(\bigcup_{x \in A}B(p_x,r_x),\forall A \in \mathcal{O}  \)一定为可数的并。从而\(\forall A \in \mathcal{O},A \in \sigma(\mathcal{A}) \)。
\end{solution}

\begin{problem}\label{pro:1.4.6}
  设\(\mathcal{C} \)是\(\Omega \)中任一集代数,则存在\(\Omega \)中的单调类\(\mathcal{M}_0 \)满足:
  \begin{enumerate}
    \item \(\mathcal{C} \subset \mathcal{M}_0 \),
    \item 对于包含\(\mathcal{C} \)的单调类\(\mathcal{M} \),有\(\mathcal{M}_0 \subset \mathcal{M} \)。
  \end{enumerate}
  称这样的单调类为\(\mathcal{C} \)生成的单调类,记作\(\mathcal{M}(\mathcal{A}) \)。
\end{problem}
\begin{solution}
考虑\(\mathcal{A} :=\{\mathcal{M}:\mathcal{M} \text{为单调类且} \mathcal{C} \subset \mathcal{M} \} \)。由于\(\Omega \)的全体子集\(P(\Omega) \)显然为包含\(\mathcal{C} \)的单调类。
那么\(P(\Omega) \in \mathcal{A} \),故\(\mathcal{A} \neq \varnothing\)。令\(\mathcal{M}_0=\bigcap_{A \in \mathcal{A}} A \),那么\(\mathcal{C} \subset \mathcal{M}_0 \)。
下证\(\mathcal{M}_0 \)为单调类。
\begin{itemize}
  \item \(A_n \in \mathcal{M}_0, n \in \mathbb{N} \),满足\(A_n \subset A_{n + 1}, n \in \mathbb{N} \)。\(\forall A \in \mathcal{A}  \),\(A_n \in A \),由于\(A \)为单调类,那么\(\cup_{n \in \mathbb{N}} A_n \in A \)。
    故\(\cup_{n \in \mathbb{N}} A_n \in \bigcap_{A \in \mathcal{A} }A \)。
  \item 同理可证,\(A_n \in \mathcal{M}_0, n \in \mathbb{N} \),满足\(A_n \supset A_{n + 1}, n \in \mathbb{N} \),则\(\cap_{n \in \mathbb{N}}A_n \in \cap_{A \in \mathcal{A} }A \)。
\end{itemize}
设\(\mathcal{M}  \)为包含\(\mathcal{C} \)的单调类,那么\(\mathcal{M} \in \mathcal{A} \),那么\(\mathcal{M} \supset \mathcal{M}_0 \)。
\end{solution}

\begin{problem}\label{pro:1.4.9}
  设\(\Omega_i ,i=1,2,\cdots,n\)是\(n \)个集合,\(\mathcal{A}_i \)是\(\Omega_i \)上的\(\sigma \)-代数。证明\(\mathcal{C}=\{A_1 \times \cdots \times A_n:A_i \in \mathcal{A}_i\} \)为半集代数。
\end{problem}
\begin{solution}
  \begin{lemma}\label{lem:1.4.9}
    \(\Omega_i,i=1,2 \) 为两个集合,\(A_i,B_i \subset \Omega_i,i=1,2\),那么以下命题正确。
    \begin{enumerate}
      \item \label{ite:1} \((A_1 \times A_2) \cap (B_1 \times B_2)=(A_1 \cap B_1) \times (A_2 \cap B_2) \);
      \item \label{ite:2} 若\(A_1 \times A_2 \subset B_1 \times B_2 \),那么\(B_1 \times B_2 = (A_1 \times A_2) \cup ((B_1 / A_1) \times B_2) \cup (A_1 \times (B_2 / A_2)) \),
        其中\((A_1 \times A_2) , ((B_1 / A_1) \times B_2) , (A_1 \times (B_2 / A_2))\)两两不交。
      \item \label{ite:3} 若\(A_1 \times A_2 \subset B_1 \times B_2 \),那么\(A_1 \subset B_1 \),\(A_2 \subset B_2 \)。
    \end{enumerate}
  \end{lemma}
 \begin{proof}
   \begin{enumerate}
     \item \(\forall (a,b) \in (A_1 \times A_2) \cap (B_1 \times B_2) \),那么\((a,b) \in A_1 \times A_2 \)且\((a,b) \in B_1 \times B_2 \)。
    从而\(a \in A_1, b \in A_2 \),\(a \in B_1,b \in B_2 \)。故\(a \in A_1 \cap B_1 \), \(b \in A_2 \cap B_2 \)。
    那么\((a,b) \in (A_1 \cap B_1) \times (A_2 \cap B_2) \)。另一方面,\(\forall (a,b) \in (A_1 \cap B_1)\times (A_2 \times B_2) \),
    那么\(a \in A_1 \cap B_1,b \in A_2 \times B_2 \),故\((a,b) \in A_1 \times A_2 \),\((a,b) \in B_1 \times B_2 \)。故\((a,b) \in (A_1 \times A_2)\cap (B_1 \times B_2) \)。
  \item 先证 \((A_1 \times A_2) , ((B_1 / A_1) \times B_2) , (A_1 \times (B_2 / A_2))\)两两不交:由\ref{ite:1}知, \((A_1 \times A_2) \cap ((B_1 / A_1)\times B_2)=\varnothing \times B_2 =\varnothing \)。
    \((A_1 \times A_2) \cap (A_1 \times(B_2 / A_2))=A_1 \times \varnothing=\varnothing \)。\(((B_1/A_1)\times B_2)\cap (A_1 \times (B_2 /A_2))= \varnothing \times B_2 \)。
       下证\(B_1 \times B_2 = (A_1 \times A_2) \cup ((B_1 / A_1) \times B_2) \cup (A_1 \times (B_2 / A_2)) \)。由于\(A_1, B_1 / A_1 \subset B_1, A_2, B_2/A_2 \subset B_2 \),
       那么\((A_1 \times A_2) , ((B_1 / A_1) \times B_2) , (A_1 \times (B_2 / A_2)) \subset B_1 \times B_2\),从而\( (A_1 \times A_2) \cup ((B_1 / A_1) \times B_2) \cup (A_1 \times (B_2 / A_2)) \subset B_1 \times B_2 \)。
       又\(B_1 \times B_2 =((B_1 /A_1) \times B_2)\cup (A_1 \times B_2) =((B_1 /A_1) \times B_2) \cup ((A_1 \times (B_2 / A_2)) \cup (A_1 \times A_2))  \),从而结论正确。
     \item 若\(A_1 / B_1 \neq \varnothing \),设\(a \in A_1/B_1 \),取\(b \in A_2 \),那么\((a,b) \in A_1 \times A_2 \),但\((a,b) \notin B_1 \times B_2  \),与\(A_1 \times A_2 \subset B_1 \times B_2 \)矛盾。
   \end{enumerate}
 \end{proof}
由于\(\mathcal{A}_i,1 \leq i \leq n \) 是\(\Omega_i \)  上的\(\sigma \)-代数,从而\(\mathcal{A}_i,1 \leq i \leq n \)为\(\Omega_i \)上的半集代数。
我们可以用数学归纳法证明以下命题:设\(\Omega_i ,i=1,2,\cdots,n\)是\(n \)个集合,\(\mathcal{A}_i \)是\(\Omega_i \)上的半集代数,那么\(\mathcal{C}_n=\{A_1 \times \cdots \times A_n:A_i \in \mathcal{A}_i\} \)为半集代数。
\begin{itemize}
  \item 当\(n=1 \)时,\(\mathcal{C}_1=\mathcal{A}_1 \),显然为半集代数。
  \item 当\(n=2 \)时,\(\mathcal{C}_2=\{A_1 \times A_2: A_i \in \mathcal{A}_i,i=1,2\} \)。下证\(\mathcal{C}_2 \)为半集代数。
    \begin{itemize}
      \item 由于\(A_1,A_2 \)为半集代数,那么\(\Omega_i, \varnothing \in A_i,i=1,2\)。
    从而\(\{\varnothing \times \varnothing , \Omega_1 \times \Omega_2\} \subset \mathcal{C}_1 \)。
  \item 设\(A_1 \times A_2, B_1 \times B_2 \in \mathcal{C} \),那么由引理\ref{lem:1.4.9}中的\ref{ite:1}可知 \((A_1 \times A_2) \cap (B_1 \times B_2)=(A_1 \cap B_1) \times (A_2 \cap  B_2) \)。
    又由\(\mathcal{A}_i,i=1,2 \)均为半集代数,那么\(A_i \cap B_i \in \mathcal{A}_i, i=1,2 \)。那么\((A_1 \cap B_1) \times (A_2 \times B_2) \in \mathcal{C}_2  \)。
    故\((A_1 \times A_2) \cap (B_1 \times B_2) \in \mathcal{C}_2 \)。
  \item 若\(A_1 \times A_2 \subset B_1 \times B_2 \),那么由引理\ref{lem:1.4.9}中的\ref{ite:3}知\(A_1 \subset B_1 \),\(A_2 \subset B_2 \)。
    由于\(A_i, B_i \in \mathcal{A}_i\),那么\(\exists C^i_k \in \mathcal{A}_i,1 \leq k \leq N_i, N_i \in \mathbb{N}_{+} \)两两不交,与\(A_i \)也不交,且\(B_i=A_i \cup (\bigcup_{1 \leq k \leq N_i} C^i_k),i=1,2\)。
    由于\(C^i_k \in \mathcal{A}_i, 1 \leq k \leq N_i,i=1,2\),那么\(C^1_k \times B_2 \in \mathcal{C}_2, 1 \leq k \leq N_1 \), \(A_1 \times C^2_k \in \mathcal{C}_2, 1 \leq k \leq N_2 \)。
    由于\(C^i_k \in \mathcal{A}_i, 1 \leq k \leq N_i,i=1,2\)两两不交,那么\(C^1_k \times B_2, 1 \leq k \leq N_1\)两两不交,\(A_1 \times C^2_k,1 \leq k \leq N_2 \)两两不交。
    又由于\ref{lem:1.4.9}中的\ref{ite:2}知,
        \(B_1 \times B_2 = (A_1 \times A_2) \cup ((B_1 / A_1) \times B_2) \cup (A_1 \times (B_2 / A_2)) \)。那么
\[
 \begin{aligned}
   B_1 \times B_2 =& (A_1 \times A_2) \cup (( \bigcup_{1 \leq k \leq N_1} C^1_k )\times B_2 )\cup (A_1 \times (\bigcup_{1 \leq k \leq N_2} C^2_k))\\ 
   =&(A_1 \times A_2) \cup \bigcup_{1 \leq k \leq N_1} (C^1_k \times B_2) \cup \bigcup_{1 \leq k \leq N_2}(A_1 \times C^2_k)\\
 \end{aligned}
\]
又由\((A_1 \times A_2) , ((B_1 / A_1) \times B_2) , (A_1 \times (B_2 / A_2)) \)两两不交,
        从而\((A_1 \times A_2) , (\bigcup_{1 \leq k \leq N_1} (C^1_k \times B_2) ), ( \bigcup_{1 \leq k \leq N_2} (A_1 \times C^2_k)) \),
        故\((A_1 \times A_2) , (C^1_k \times B_2 ), (A_1 \times C^2_j), 1 \leq  k \leq N_1, 1 \leq j \leq N_2 \)两两不交。
        从而\(B_1 \times B_2 \)能表示成\(A_1 \times A_2 \) 与\(\mathcal{C}_2 \)中元素的不交并。
    \end{itemize}
  \item 设\(n=k, 1 \leq k \leq n-1\)时,\(\mathcal{C}_k:=\{A_1 \times \cdots \times A_k:A_i \in \mathcal{A}_i, 1 \leq i \leq k\} \)为半集代数。
    那么\[
      \begin{aligned}
        \mathcal{C}_{k + 1}:=&\{A_1 \times \cdots \times A_{k + 1}: A_i \in \mathcal{A}_i, 1 \leq i \leq k + 1\}\\
        =&\{(A_1 \times \cdots \times A_k)\times A_{k + 1}: A_i \in \mathcal{A}_i, 1 \leq i \leq k + 1\} \\ 
        =&\{C \times A: C \in \mathcal{C}_k ,A \in \mathcal{A}_{k + 1}\}.
      \end{aligned}
    \]
    由\(n=2 \)的情形可知\(\mathcal{C}_{k + 1} \)为半集代数。
\end{itemize}
\end{solution}

\begin{problem}\label{pro:1.4.11}
  举例说明可加测度未必有限可加。
\end{problem}
\begin{solution}
考虑\(\Omega=\{1,2,3\},\mathcal{T}=\{\{1\},\{2\},\{3\},\{1,2,3\},\varnothing \} \),\(\Phi:\mathcal{T} \to \{0,1\} \),
其中\(\Phi(\{\varnothing\})=0,\Phi(A)=1,A \in \mathcal{T}\setminus\{\varnothing\} \)。
那么\(\Phi \)为\(\mathcal{T} \)上的可加测度。考虑\(\{1\},\{2\},\{3\} \)两两不交且\(\{1,2,3\}=\{1\} \cup \{2\}\cup \{3\} \in \mathcal{T} \),
则\(\Phi(\{1\} \cup \{2\} \cup \{3\})=\Phi(\{1,2,3\})=1\),\(\sum_{k=1}^{3}\Phi(\{k\})=\sum_{k=1}^{3}1=3 \)。
故\(\sum_{k=1}^{3}\Phi(\{k\}) \neq \Phi(\{1,2,3\}) \)。故\(\Phi \)不是有限可加测度。
\end{solution}

% \begin{problem}\label{pro:10}
%   举例说明半集代数\(\mathcal{T}\)生成的\(\sigma\)-代数不能一般性地表述为
%   \[
%     \sigma(\mathcal{T})=\{\sum_{n=1}^{\infty} A_n:\forall n \geq 1, A_n \in \mathcal{T}\}.
%   \]
%   但如果\(\Omega\)至多可数时,如上表述是正确的.
% \end{problem}
% \begin{solution}
%   下证如上表述在\(\Omega \)可数时,不正确。
%  考虑\(\mathcal{T}:=\{A \subset \mathbb{N}: 0 \in A, |A^c| < \infty \text{或}
%   0 \notin A, |A| < \infty  \} \)。令\(\mathcal{A}:=\{\sum_{n=1}^{\infty}A_n:\forall n \geq 1, A_n \in \mathcal{T}\} \),
% 先证\(\mathcal{T} \)为半集代数。
%   \begin{itemize}
%     \item 由于\(0 \notin \varnothing\),且\(|\varnothing| =0 \),那么\(\varnothing \in \mathcal{T} \)。
%       又由于\(0 \in \mathbb{N} \),且\(|\mathbb{N}^c|=|\varnothing|=0 \),那么\(\mathbb{N} \in \mathcal{T} \)。
%     \item \(\forall A, B \in \mathcal{T} \),若\(0 \in A, 0 \in B\),那么\(0 \in A \cap B \)。
%       由于\(|A^c|,|B^c|< \infty \),那么\(|(A \cap B)^c|=|A^c \cup B^c|\leq |A^c| + |B^c| < \infty \)。
%       故\(A \cap B \in \mathcal{T} \)。若\(0 \in A, 0 \notin B \),那么\(0 \notin A \cap B \)。
%       由于\(|A^c|,|B| < \infty \), 那么\(|A \cap B | \leq |B| < \infty \)。从而\(A \cap B \in \mathcal{T} \)。
%       若\(0 \notin A, 0 \notin B \),那么\(0 \notin A \cap B \)。
%       又由于\(|A|,|B|< \infty \),那么\(|A \cap B| \leq |A| < \infty \)。
%       从而\(A \cap B \in \mathcal{T} \)。综上,\(A \cap B \in \mathcal{T} \)。
%     \item \(\forall A \in \mathcal{T} \),若\(0 \in A \),那么\(|A^c|< \infty \)。又由于\(0 \notin A^c \),
%       那么\(A^c \in \mathcal{T} \)。若\(0 \notin A \),那么\(|A| < \infty \)。又由于\(0 \in A^c \),
%       那么\(|(A^c)^c|< \infty \)。故\(A^c \in \mathcal{T} \)。综上,\(A^c \in \mathcal{T} \)。
%   \end{itemize}
%    下证\(\sigma(\mathcal{T}) \neq \mathcal{A} \)。 事实上,\(\{0\} \in \sigma(\mathcal{T}) \setminus \mathcal{A} \)。
%    若\(\{0\} \in \mathcal{A} \),那么\(\exists A_n \in \mathcal{T},n \geq 1,A_i \cap A_j = \varnothing ,i \neq j \),使得\(\{0\}=\sum_{n=1}^{\infty}A_n \)。
%    由于\(|\{0\}|=1 < \infty \),那么\(|A_n|<\infty ,\forall n \geq 1\)。
%    故\(0 \notin A_n ,\forall n \geq 1\)。从而\(0 \notin \cup_{n \geq 1}A_n \)。
%    故\(\{0\} \notin \mathcal{A} \)。而\(\{0\}=(\cup_{k \geq 1}\{k\})^c \),\(\{k\} \in \mathcal{T},k \geq1 \)。故\(\{0\} \in \sigma(\mathcal{T}) \)。
% \end{solution}
\begin{problem} 
  设\(\mathcal{C}_n \)为单调上升的子集类:
  \begin{enumerate}
    \item 若\(\mathcal{C}_n \)为集代数,则\(\cup_{n=1}^{\infty}\mathcal{C}_n \)为集代数。
    \item 若\(\mathcal{C}_n \)为\(\sigma \)代数,举例\(\cup_{n=1}^{\infty}\mathcal{C}_n \)不为\(\sigma \)代数。
  \end{enumerate}
\end{problem}
\begin{solution}
由于\(\mathcal{C}_n \)单调上升,那么\(\mathcal{C}_n \subset \mathcal{C}_{n + 1}, n \geq 1 \).
\begin{enumerate}
  \item 令\(\mathcal{A}:=\cup_{n=1}^{\infty} \mathcal{C}_n \)。下证\(\mathcal{A} \)为集代数。
    \begin{itemize}
      \item 由于\(\mathcal{C}_n,n \geq 1\)为集代数,那么\(\Omega \in \mathcal{C}_n ,n \geq 1 \)。 故\(\Omega \in \cup_{n=1}^{\infty} \mathcal{C}_n \)。
      \item \(\forall A,B \in \mathcal{A} \),那么\(\exists n_1,n_2 \geq 1 \),满足 \(A \in \mathcal{C}_{n_1},B \in \mathcal{C}_{n_2} \)。
        不妨\(n = \max\{n_1,n_2\} \),那么\(A \in \mathcal{C}_{n_1} \subset \mathcal{C}_n, B \in \mathcal{C}_{n_2} \subset \mathcal{C}_n\).
        由于\(\mathcal{C}_n \)为集代数,那么\(A \setminus B \in \mathcal{C}_n \)。故\(A \setminus B \in \mathcal{A} \)。
    \end{itemize}
  \item 考虑\(\Omega=\mathbb{N},\mathcal{C}_n=\sigma(\{\{k\}: 1 \leq k \leq n\}\}) \)。令\( \mathcal{A}:=\cup_{k=1}^{\infty}\mathcal{C}_n\)。
    \(\forall n \geq 1 \),令\(T_n:=\{k: k >n\} \),那么\(\mathcal{T}_n:=\{\{k\} : 1 \leq k \leq n\} \cup \{T_n\} \)为\(\mathcal{C}_n \)的一个划分。
    下证\(\mathcal{C}_n=\{A \subset \mathbb{N}: A \cap T_n = \varnothing \text{或} A \cap T_n =T_n\}=:\mathcal{A}_n \)。
    显然\( \mathcal{T}_n \subset \mathcal{A}_n\),故只需证明,\(\mathcal{A}_n \)为\(\sigma \)-代数。
    \begin{enumerate}
      \item \(\mathbb{N} \in \mathcal{A}_n \)显然。
      \item \(\forall A,B  \in \mathcal{A}_n\),若\(A \cap T_n=B \cap T_n=\varnothing \),那么\((A \cap B) \cap T_n =\varnothing \),那么\(A \cap B \in \mathcal{A}_n \)。
        若\(A \cap T_n = \varnothing \), \(B \cap T_n =T_n \),那么\((A \cap T_n) \cap (B \cap T_n)=(A \cap B) \cap T_n=\varnothing \),那么\(A \cap B \in \mathcal{A}_n \)。
        若\(A \cap T_n = B \cap T_n = T_n \),那么\((A \cap B)\cap T_n =T_n \), 从而\(A \cap B \in \mathcal{A}_n \)。
      \item \(\forall A_t \in \mathcal{A}_n, t \geq 1, A_i \cap A_j = \varnothing , i \neq j\),那么至多一个\(A_t \)满足\(A_t \cap T_n =T_n \)。 
        若\(\forall t \geq 1, A_t \cap T_n = \varnothing \),那么\(\cup_{t \geq 1}A_t \cap T_n =\varnothing \),从而\(\cup_{t \geq 1} A_t \in \mathcal{A}_n \)。
        若\(\exists t \geq 1 , A_t \cap T_n = T_n\),不妨设\(A_1 \cap T_n =T_n \),那么\(\cup_{t \geq 1} A_t \cap T_n =T_n \),从而\(\cup_{t \geq 1} A_t \in \mathcal{A}_n \)。
    \end{enumerate}
    那么\(\forall C \in \mathcal{C}_n \),\(|C |< \infty \)或\(|C^c| \leq |T_n^c| <\infty \)。
    由于\(\forall n \geq 1 \),\(\{2n\} \in \mathcal{C}_{2n} \subset \mathcal{A} \),而\(\cup_{n \geq 1} \{2n\} =2\mathbb{N} \),\(|2\mathbb{N}|,|\mathbb{N} \setminus 2\mathbb{N}| = \infty \)。
    故\(2\mathbb{N} \notin \mathcal{C}_n, \forall n \geq 1  \)。 从而\(2\mathbb{N} \notin \mathcal{A} \)。
\end{enumerate}
\end{solution}

\begin{problem} 
  证明\(\sigma \)-代数不可能是可数无穷的。
\end{problem}
\begin{lemma}\label{lem:atom}
  \(\Omega \neq \varnothing,\{A_{\alpha} \subset \Omega:\alpha \in I\}=:\mathcal{A} \)为\(\Omega \)的一个划分,令\(\mathcal{F}:=\sigma(\mathcal{A}) \),
  那么\(|\mathcal{F}|=2^{|\mathcal{A}|} \)。
\end{lemma}
\begin{solution}
  考虑\(\phi:\mathcal{P}(I) \to \mathcal{P}(\Omega) \),\(J \mapsto \cup_{i \in J} A_i \)。显然\(\phi \)为单射。下证\(\mathrm{Im}(\phi) = \mathcal{F} \)。
  由于\(\forall J \in \mathcal{P}(I) \),\(\cup_{i \in J}A_i \in \mathcal{F} \),那么\(\mathrm{Im}(\phi) \subset \mathcal{F} \)。故只需证明\(\mathcal{F} \subset \mathrm{Im}(\phi) \)。
  由于\(\forall j \in I \),\(\{j\} \in \phi(I) \),那么\(\cup_{i \in \{j\}}A_i=A_j \in \mathrm{Im}(\phi) \)。从而\(\mathcal{A} \subset \mathrm{Im}(\phi) \)。
  故只需证明\(\mathrm{Im}(\phi) \)为\(\sigma \)代数。
  \begin{itemize}
    \item 由于\(\phi(I)=\cup_{i \in I} A_i =\Omega \),那么\(\Omega \in \mathrm{Im}(\phi) \)。
    \item \(\forall I,J \in \mathcal{P}(I) \),那么\(I \cap J \in \mathcal{P}(I) \),故\(\cup_{i \in I \cap J} A_i = (\cup_{i \in I}A_i) \cap (\cup_{i \in J}A_i) \in \mathrm{Im}(\phi) \)。
    \item \(\forall I_n \in \mathcal{P}(I), n \geq 1 \),那么\( \cup_{n \geq 1} I_n \in \mathcal{P}(I)\),故\(\cup_{n \geq 1}(\cup_{i \in I_n} A_i )= \cup_{i \in \cup_{n \geq 1}I_n}A_i \in \mathrm{Im}(\phi)\)。
  \end{itemize}
\end{solution}


\begin{solution}
设\(\mathcal{F} \)为\(\Omega \)上\(\sigma \)-代数。\( \mathcal{A}\)为\(\mathcal{F} \)上的所有原子集组成的集合。
\begin{itemize}
  \item 若\(\cup \mathcal{A} = \Omega\), 且\(|\mathcal{A}| < \infty \),那么由引理 \ref{lem:atom} 知,\(|\mathcal{F}|=2^{|\mathcal{A}|} \)。 故\(\mathcal{F} \)为有限集。
  \item 若\(\cup \mathcal{A} =\Omega \),且\(|\mathcal{A}| = \infty \),那么取\(A_n \in \mathcal{A}, n \geq 1 \),以及\(A_0=\Omega \setminus \cup_{n \geq 1}A_n \)。那么\(\{A_n:n \in \mathbb{N}\} =: \mathcal{B} \)为\(\Omega \)的 
    可数分割。由引理 \ref{lem:atom}知,\(|\mathcal{F}|=2^{|\mathcal{B}|}> \aleph_0 \).
  \item 若\(\cup \mathcal{A} \subseteq \Omega \),那么\(\exists F_n, n \geq 1 \), 满足 \(F_n\supset F_{n + 1}, n \geq 1, F_0=\Omega\)。考虑\(A_n:=F_{n } \setminus F_{n + 1}, n \geq 0\), 那么\(\mathcal{B}:=\{A_n: n \in \mathbb{N}\} \)为 
    \(\Omega \)的可数分割。由引理 \ref{lem:atom}知,\(|\mathcal{F}|=2^{|\mathcal{B}|} > \aleph_0 \)
\end{itemize}

\end{solution}

\begin{problem} 
  设\((\Omega_n,\mathcal{A}_n,\mu_{n}),n \geq 1 \)为一列测度空间,\(\Omega_n \)两两不交。令 \[
    \Omega=\sum_{n=1}^{\infty} \Omega_n, \mathcal{A}=\{A \subset \Omega : \forall n \geq 1,A \cap \Omega_n \in \mathcal{A}_n\},\mu (A)=\sum_{n=1}^{\infty} \mu_{n} (A \cap \Omega_n), A \in \mathcal{A} 
  \]
  证明\((\Omega , \mathcal{A} , \mu ) \)为测度空间。
\end{problem}
\begin{solution}
  先证\(\mathcal{A} \)为\(\sigma \)-代数。
  \begin{itemize}
    \item 由于\(\Omega=\sum_{n=1}^{\infty} \Omega_n \),那么\(\Omega \cap \Omega_n= \Omega_n \in \mathcal{A}_n, \forall n \geq 1 \)。 故\(\Omega \in \mathcal{A}\)。
    \item \(\forall A, B \in \mathcal{A}\),那么\(\forall n \geq 1, A \cap \Omega_n, B \cap \Omega_n \in \mathcal{A}_n  \)。由于\(\mathcal{A}_n \)为\(\sigma \)代数,
      那么\((A \cap \Omega_n) \cap ( B \cap \Omega_n)=(A \cap B) \cap \Omega_n \in \mathcal{A}_n\)。从而\(A \cap B \in \mathcal{A} \)。
    \item \(A_n \in \mathcal{A} , n \geq 1 \), 那么\(\forall k \geq 1, A_n \cap \Omega_k  \in  \mathcal{A}_k \)。由于\(\mathcal{A}_k \)为\(\sigma \)代数,
      那么\(\bigcup_{n \geq 1}(A_n \cap \Omega_k) = (\bigcup_{n \geq 1} A_n) \cap \Omega_k \in \mathcal{A}_k \),从而\(\bigcup_{n \geq 1}A_n \in \mathcal{A} \)。
  \end{itemize}
 再证\(\mu \)为测度。
  \begin{itemize}
    \item 由于\(\forall A \in \mathcal{A} \), \(\mu(A)=\sum_{n=1}^{\infty} \mu_n(A \cap \Omega_n) \),而\(A \cap \Omega_n \in \mathcal{A}_n, \forall n \geq 1 \),故\(\mu_n(A \cap \Omega_n) \geq 0 \).
      从而,\(\mu(A)= \sum_{n=1}^{\infty} \mu_n(A \cap \Omega_n) \geq 0 \).
    \item \(\forall A_n \in \mathcal{A}, n \geq 1, A_i \cap A_j =\varnothing, i \neq j \),\(\sum_{n \geq 1}A_n=\sum_{n \geq 1}(A_n \cap \bigcup_{k \geq 1}\Omega_k)=\sum_{n \geq 1}(\sum_{k \geq 1} (A_n \cap \Omega_k))= 
      \sum_{k \geq 1}\sum_{n \geq 1} (A_n \cap \Omega_k)  = \sum_{k \geq 1} (\sum_{n \geq 1} A_n) \cap \Omega_k\)。
      由于\(\forall n \geq 1, A_n \cap \Omega_k \in \mathcal{A}_k \),那么\(\sum_{n \geq 1} A_n \cap \Omega_k  \in \mathcal{A}_k\) 。故有\(\mu(\sum_{n \geq 1}A_k \cap \Omega_k)=\mu_k(\sum_{n \geq 1}A_n \cap \Omega_k)=\sum_{n \geq 1} \mu_k(A_n \cap \Omega_k) \)。
      \[
        \begin{aligned}
          \mu(\sum_{n \geq 1} A_n) &= \mu(\sum_{n \geq 1} (\sum_{k \geq 1} (A_n \cap \Omega_k)))\\ 
          &= \mu(\sum_{k \geq 1} (\sum_{n \geq 1} (A_n \cap \Omega_k)))\\ 
          &= \sum_{k \geq 1}\mu_k(\sum_{n \geq 1}(A_n \cap \Omega_k))\\ 
          &= \sum_{k \geq 1}\sum_{n \geq 1} \mu_k(A_n \cap \Omega_k)\\ 
          &=\sum_{n \geq 1}\sum_{k \geq 1}\mu_k(A_n \cap \Omega_k)\\ 
          &=\sum_{n \geq 1}\mu(A_n)
        \end{aligned}
      \]
  \end{itemize}
\end{solution}

\begin{problem} 
  设\(\Omega \)为一无穷集,令\(\mathcal{F} \)为\(\Omega \)中的有限集或者余有限集构成的集合,\(\mathbb{P} \)在两类集合上取值
  分别为\(0 \)或\(1 \)。
  \begin{itemize}
    \item 证明\(\mathcal{F} \)为集代数,\(\mathbb{P} \)为有限可加。
    \item 若\(\Omega \)为可数集,则\(\mathbb{P} \)不可能为\(\sigma \)可加。
    \item 若\(\Omega \)为不可数集,则\(\mathbb{P} \)为可数可加。
  \end{itemize}
\end{problem}
\begin{solution}
 \begin{enumerate}
   \item 先证\(\mathcal{F} \)为集代数:
     \begin{itemize}
       \item 由于\(\Omega^c=\varnothing \),\(|\varnothing| =0 \),故\(\Omega \in \mathcal{F} \)。
       \item \(\forall A,B \in \mathcal{F} \),若 \(|A|<\infty \),那么\(|A\setminus B|=|A \cap B^c |\leq |A|<\infty \)。若\(|A^c|,|B|<\infty \),那么\(|(A\setminus B)^c|=|A^c \cup B|\leq |A^c| + |B|< \infty \)。
         若\(|A^c|,|B^c| <\infty \),那么 \(|A\setminus B|=|A \cap B^c|\leq |B^c|< \infty \)。综上所述,\(A\setminus B \in \mathcal{F} \)。
     \end{itemize}
    再证\(\mathbb{P} \)有限可加。由于\(\mathcal{F} \)为集代数,故只需证明\(\mathbb{P} \)可加。
    设 \(A,B \in \mathcal{F} \),\(A \cap B =\varnothing \),\(A \cup B \in \mathcal{F} \),若\(|A|,|B| <\infty \),\(\mathbb{P}(A \cup B) =0 = \mathbb{P}(A) + \mathbb{P}(B) \)。
     若\(|A|,|B^c|<\infty \),那么\((A \cup B)^c=A^c \cap B^c \subset B^c \),故\(\mathbb{P}(A \cup B)=1=\mathbb{P}(A) + \mathbb{P}(B) \)。
     若\(|A^c|,|B^c|< \infty \),那么由 \(A \cap B=\varnothing\),知\(B \subset A^c \),那么\(|B| \leq |A^c|<\infty \),而\(|\Omega|=\infty \), 
     故\(|B|=|\Omega \setminus B^c|=\infty \),矛盾。从而\(|A^c|,|B^c|<\infty \)不成立。
   \item \(\mathbb{P}(\sum_{a \in \Omega}\{a\})=\mathbb{P}(\Omega)=1 \),而\(\sum_{a \in \Omega }\mathbb{P}(\{a\})=0 \),故\(\mathbb{P} \)不是\(\sigma \)可加。
   \item \(A_n \in \mathcal{F}, n \geq 1 \),\(A_i \cap A_j =\varnothing, i \neq j,\sum_{n \geq 1}A_n \in  \mathcal{F} \)。那么至多一个 \(A_n\)是余有限的。否则,根据上一小问的证明,
     两余有限集必有交。
     若\(A_n \)中没有余有限集,那么 \(\sum_{n \geq 1}A_n \)至多可数,而\(\Omega \)为不可数集,从而\(\sum_{n \geq 1}A_n \)不为余有限集。
     又由\(\sum_{n \geq 1}A_n \in \mathcal{F} \),则\(\sum_{n \geq 1}A_n \)为有限集,此时\(A_n \)中只有有限个集合非空,且为有限集。那么\(\mathbb{P}(\sum_{n \geq 1}A_n)=0=\sum_{n \geq 1}\mathbb{P}(A_n) \)。
     若\(A_n \)中有余有限集,不妨假设\(A_1 \)余有限,那么\(\mathbb{P}(\sum_{n \geq 1}A_n)=1=\sum_{n \geq 1} \mathbb{P}(A_n) \)
 \end{enumerate}
\end{solution}

\end{document}


