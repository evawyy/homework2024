%!Mode:: "TeX:UTF-8"
%!TEX encoding = UTF-8 Unicode
%arara: xelatex
\documentclass{ctexart}
\newif\ifpreface
%\prefacetrue
\input{../../../global/all}
\begin{document}
\large
\setlength{\baselineskip}{1.2em}
\ifpreface
\input{../../../global/preface}
\newgeometry{left=2cm,right=2cm,top=2cm,bottom=2cm}
\else
\newgeometry{left=2cm,right=2cm,top=2cm,bottom=2cm}
\maketitle
\fi
%from_here_to_type
\begin{problem}\label{pro:1.4.2}
  证明\(\sigma \)-代数是集代数。
\end{problem}
\begin{solution}
  假定\(\mathcal{A}\)是\(\sigma\)-代数,那么\(\Omega \in \mathcal{A} \),\(\mathcal{A}\)对补运算封闭。只需证明\(\mathcal{A}\)对有限并封闭。
  \(A,B \in \mathcal{A}\),令\(A_n=\varnothing=\Omega^c \in \mathcal{A}, n \geq 3\),那么\(A \cup B \cup \bigcup_{n \geq 3}A_n=A \cup B \in \mathcal{A}\)。
\end{solution}

\begin{problem}\label{pro:1.4.3}
  设\(\mathcal{C} \)是集类,则\(\forall A \in \sigma(\mathcal{C}),\exists \mathcal{C}_1 \subset \mathcal{C},|\mathcal{C}_1| \leq \aleph_0,A \in \sigma(\mathcal{C}_1) \)。
\end{problem}
\begin{solution}
  令\(\mathcal{A}:=\{A \in \sigma(\mathcal{C}):\exists \mathcal{C}_1 \subset \mathcal{C}, |\mathcal{C}_1| \leq \aleph_0, A \in \sigma(\mathcal{C}_1) \}\).
  下证\(\mathcal{A} \supset \sigma (\mathcal{C}) \),即证\(\mathcal{A} \)为包含\(\mathcal{C} \)的\(\sigma \)-代数。
  \begin{itemize}
    \item 由于\(\forall A \in \mathcal{C} , \sigma(\{A\})=\{\varnothing,\Omega,A,A^c\} \),那么\(\mathcal{C}  \subset \mathcal{A} \)。
    \item 由\(\Omega \in \sigma(\varnothing)=\{\varnothing,\Omega\} \),那么\(\Omega \in \mathcal{A} \)。
    \item 设\(A \in \mathcal{A} \),那么\(\exists \mathcal{C}_1 \subset \mathcal{C}, |\mathcal{C}_1|\leq \aleph_0\),\(A \in \sigma(\mathcal{C}_1) \).
      由于\(\sigma(\mathcal{C}_1) \)是\(\sigma \)-代数,那么\(A^c \in \sigma(\mathcal{C}_1) \).所以\(A^c \in \mathcal{A} \)。
    \item 设\(A_n \in \mathcal{A} ,n \in \mathbb{N}\),那么\(\exists \mathcal{C}_n \subset \mathcal{C} , |\mathcal{C}_n|\leq \aleph_0,n \in \mathbb{N}   \),满足\(A_n \in \sigma(\mathcal{C}_n) \forall n \in \mathbb{N}\)。
      令\(\mathcal{T}=\bigcup_{n \in \mathbb{N}} \mathcal{C}_n \),由\(|\mathcal{C}_n|\leq \aleph_0 ,\mathcal{C}_n \subset \mathcal{C}\),可知\(|\mathcal{T}| \leq \aleph_0,\mathcal{T} \subset \mathcal{C} \),那么\(A_n \in \sigma(\mathcal{C}_n)\subset \sigma (\mathcal{T}), \forall n \in \mathbb{N}\)。
      所以\(\bigcup_{n \in \mathbb{N} }A_n \in \sigma(\mathcal{T}) \)。那么\(\bigcup_{n \in \mathbb{N}}A_n \in \mathcal{A} \)。
  \end{itemize}
 综上,\(\mathcal{A} \)为包含\(\mathcal{C} \)的\(\sigma \)-代数。故\(\mathcal{A} \supset \sigma(\mathcal{C}) \)。 
 又由于\(\mathcal{A} \subset \sigma(\mathcal{C}) \),故\(\mathcal{A}=\sigma(\mathcal{C}) \)。
从而,\(\forall A \in \sigma(\mathcal{C}),\exists \mathcal{C}_1 \subset \mathcal{C},|\mathcal{C}_1| \leq \aleph_0,A \in \sigma(\mathcal{C}_1) \)。
\end{solution}

\begin{problem}\label{pro:1.4.4}
  \(\sigma \)-代数\(\mathcal{A} \)称为可数生成的,如果存在可数的子集类\(\mathcal{C} \subset \mathcal{A} \)使\(\sigma(\mathcal{C})=\mathcal{A} \)。
  证明\(\mathcal{B}^d \)是可数生成的。
\end{problem}
\begin{solution}
  考虑\(\mathcal{A}:=\{B(p,r):p \in \mathbb{Q}^d, r \in \mathbb{Q}_{+}\} \),其中\(B(p,r)=\{x \in \mathbb{R}^d:\norm{x-p} < r\} \)。显然\(|\mathcal{A}|=\aleph_0 \)。
  下证\(\mathcal{B}^d=\sigma(\mathcal{A}) \)。
  令\(\mathcal{O}:=\{\mathcal{B}^d \text{中的开集}\} \),
  由于\(\mathcal{B}^d=\sigma(\mathcal{O}) \),那么\( \mathcal{A} \subset \mathcal{O} \),从而\(\sigma(\mathcal{A})\subset \sigma(\mathcal{O}) \)。
  只需证明\(\mathcal{O} \subset \sigma(\mathcal{A}) \)。\\ 
  \(\forall A \in \mathcal{O} \), \( \forall x \in A \),\(\exists U=B(x,s) \),\(x \in U \subset A \)。由于\(\mathbb{Q}^d \)在\(\mathbb{R}^d \)中稠密,故\(\exists p_x \in B(x,\frac{s}{2}) \cap \mathbb{Q}^d \)。
  取\(r_x \in \mathbb{Q}_{+} \)使\(\norm{x-p_x} <r_x<\frac{s}{2} \), 由 \(\forall y \in B(p_x,r_x) \),有\(\norm{y-x}\leq \norm{y-p_x}+\norm{p_x-x} < r_x+\frac{s}{2}<s \)得 \(B(p_x,r_x) \subset B(x,s)\subset A \),
    则有 \(\bigcup_{x \in A}B(p_x,r_x) \subset  A \)。
  显然\(x \in B(p_x,r_x) \),那么\(A \subset \bigcup_{x \in A}B(p_x,r_x) \),从而 \(\forall A \in \mathcal{O},\bigcup_{x \in A}B(p_x,r_x) = A \)。
  由于\(|\mathcal{A}|=\aleph_0 \),那么\(\bigcup_{x \in A}B(p_x,r_x),\forall A \in \mathcal{O}  \)一定为可数的并。从而\(\forall A \in \mathcal{O},A \in \sigma(\mathcal{A}) \)。
\end{solution}


\begin{problem}\label{pro:1.4.6}
  设\(\mathcal{C} \)是\(\Omega \)中任一集代数,则存在\(\Omega \)中的单调类\(\mathcal{M}_0 \)满足:
  \begin{enumerate}
    \item \(\mathcal{C} \subset \mathcal{M}_0 \),
    \item 对于包含\(\mathcal{C} \)的单调类\(\mathcal{M} \),有\(\mathcal{M}_0 \subset \mathcal{M} \)。
  \end{enumerate}
  称这样的单调类为\(\mathcal{C} \)生成的单调类,记作\(\mathcal{M}(\mathcal{A}) \)。
\end{problem}
\begin{solution}
考虑\(\mathcal{A} :=\{\mathcal{M}:\mathcal{M} \text{为单调类且} \mathcal{C} \subset \mathcal{M} \} \)。由于\(\Omega \)的全体子集\(P(\Omega) \)显然为包含\(\mathcal{C} \)的单调类。
那么\(P(\Omega) \in \mathcal{A} \),故\(\mathcal{A} \neq \varnothing\)。令\(\mathcal{M}_0=\bigcap_{A \in \mathcal{A}} A \),那么\(\mathcal{C} \subset \mathcal{M}_0 \)。
下证\(\mathcal{M}_0 \)为单调类。
\begin{itemize}
  \item \(A_n \in \mathcal{M}_0, n \in \mathbb{N} \),满足\(A_n \subset A_{n + 1}, n \in \mathbb{N} \)。\(\forall A \in \mathcal{A}  \),\(A_n \in A \),由于\(A \)为单调类,那么\(\cup_{n \in \mathbb{N}} A_n \in A \)。
    故\(\cup_{n \in \mathbb{N}} A_n \in \bigcap_{A \in \mathcal{A} }A \)。
  \item 同理可证,\(A_n \in \mathcal{M}_0, n \in \mathbb{N} \),满足\(A_n \supset A_{n + 1}, n \in \mathbb{N} \),则\(\cap_{n \in \mathbb{N}}A_n \in \cap_{A \in \mathcal{A} }A \)。
\end{itemize}
设\(\mathcal{M}  \)为包含\(\mathcal{C} \)的单调类,那么\(\mathcal{M} \in \mathcal{A} \),那么\(\mathcal{M} \supset \mathcal{M}_0 \)。
\end{solution}

\begin{problem}\label{pro:1.4.9}
  设\(\Omega_i ,i=1,2,\cdots,n\)是\(n \)个集合,\(\mathcal{A}_i \)是\(\Omega_i \)上的\(\sigma \)-代数。证明\(\mathcal{C}=\{A_1 \times \cdots \times A_n:A_i \in \mathcal{A}_i\} \)为半集代数。
\end{problem}
\begin{solution}
  \begin{lemma}\label{lem:1.4.9}
    \(\Omega_i,i=1,2 \) 为两个集合,\(A_i,B_i \subset \Omega_i,i=1,2\),那么以下命题正确。
    \begin{itemize}
      \item \label{ite:1} \((A_1 \times A_2) \cap (B_1 \times B_2)=(A_1 \cap B_1) \times (A_2 \cap B_2) \);
      \item \label{ite:2} 若\(A_1 \times A_2 \subset B_1 \times B_2 \),那么\(B_1 \times B_2 = (A_1 \times A_2) \cup ((B_1 / A_1) \times B_2) \cup (A_1 \times (B_2 / A_2)) \),
        其中\((A_1 \times A_2) , ((B_1 / A_1) \times B_2) , (A_1 \times (B_2 / A_2))\)两两不交。
      \item \label{ite:3} 若\(A_1 \times A_2 \subset B_1 \times B_2 \),那么\(A_1 \subset B_1 \),\(A_2 \subset B_2 \)。
    \end{itemize}
  \end{lemma}
 \begin{proof}
   \begin{itemize}
     \item \(\forall (a,b) \in (A_1 \times A_2) \cap (B_1 \times B_2) \),那么\((a,b) \in A_1 \times A_2 \)且\((a,b) \in B_1 \times B_2 \)。
    从而\(a \in A_1, b \in A_2 \),\(a \in B_1,b \in B_2 \)。故\(a \in A_1 \cap B_1 \), \(b \in A_2 \cap B_2 \)。
    那么\((a,b) \in (A_1 \cap B_1) \times (A_2 \cap B_2) \)。另一方面,\(\forall (a,b) \in (A_1 \cap B_1)\times (A_2 \times B_2) \),
    那么\(a \in A_1 \cap B_1,b \in A_2 \times B_2 \),故\((a,b) \in A_1 \times A_2 \),\((a,b) \in B_1 \times B_2 \)。故\((a,b) \in (A_1 \times A_2)\cap (B_1 \times B_2) \)。
  \item 先证 \((A_1 \times A_2) , ((B_1 / A_1) \times B_2) , (A_1 \times (B_2 / A_2))\)两两不交:由\ref{ite:1}知, \((A_1 \times A_2) \cap ((B_1 / A_1)\times B_2)=\varnothing \times B_2 =\varnothing \)。
    \((A_1 \times A_2) \cap (A_1 \times(B_2 / A_2))=A_1 \times \varnothing=\varnothing \)。\(((B_1/A_1)\times B_2)\cap (A_1 \times (B_2 /A_2))= \varnothing \times B_2 \)。
       下证\(B_1 \times B_2 = (A_1 \times A_2) \cup ((B_1 / A_1) \times B_2) \cup (A_1 \times (B_2 / A_2)) \)。由于\(A_1, B_1 / A_1 \subset B_1, A_2, B_2/A_2 \subset B_2 \),
       那么\((A_1 \times A_2) , ((B_1 / A_1) \times B_2) , (A_1 \times (B_2 / A_2)) \subset B_1 \times B_2\),从而\( (A_1 \times A_2) \cup ((B_1 / A_1) \times B_2) \cup (A_1 \times (B_2 / A_2)) \subset B_1 \times B_2 \)。
       又\(B_1 \times B_2 =((B_1 /A_1) \times B_2)\cup (A_1 \times B_2) =((B_1 /A_1) \times B_2) \cup ((A_1 \times (B_2 / A_2)) \cup (A_1 \times A_2))  \),从而结论正确。
     \item 若\(A_1 / B_1 \neq \varnothing \),设\(a \in A_1/B_1 \),取\(b \in A_2 \),那么\((a,b) \in A_1 \times A_2 \),但\((a,b) \notin B_1 \times B_2  \),与\(A_1 \times A_2 \subset B_1 \times B_2 \)矛盾。
   \end{itemize}
 \end{proof}
由于\(\mathcal{A}_i,1 \leq i \leq n \) 是\(\Omega_i \)  上的\(\sigma \)-代数,从而\(\mathcal{A}_i,1 \leq i \leq n \)为\(\Omega_i \)上的半集代数。
我们可以用数学归纳法证明以下命题:设\(\Omega_i ,i=1,2,\cdots,n\)是\(n \)个集合,\(\mathcal{A}_i \)是\(\Omega_i \)上的半集代数,那么\(\mathcal{C}_n=\{A_1 \times \cdots \times A_n:A_i \in \mathcal{A}_i\} \)为半集代数。
\begin{itemize}
  \item 当\(n=1 \)时,\(\mathcal{C}_1=\mathcal{A}_1 \),显然为半集代数。
  \item 当\(n=2 \)时,\(\mathcal{C}_2=\{A_1 \times A_2: A_i \in \mathcal{A}_i,i=1,2\} \)。下证\(\mathcal{C}_2 \)为半集代数。
    \begin{itemize}
      \item 由于\(A_1,A_2 \)为半集代数,那么\(\Omega_i, \varnothing \in A_i,i=1,2\)。
    从而\(\{\varnothing \times \varnothing , \Omega_1 \times \Omega_2\} \subset \mathcal{C}_1 \)。
  \item 设\(A_1 \times A_2, B_1 \times B_2 \in \mathcal{C} \),那么由\ref{lem:1.4.9}中的\ref{ite:1}可知 \((A_1 \times A_2) \cap (B_1 \times B_2)=(A_1 \cap B_1) \times (A_2 \cap  B_2) \)。
    又由\(\mathcal{A}_i,i=1,2 \)均为半集代数,那么\(A_i \cap B_i \in \mathcal{A}_i, i=1,2 \)。那么\((A_1 \cap B_1) \times (A_2 \times B_2) \in \mathcal{C}_2  \)。
    故\((A_1 \times A_2) \cap (B_1 \times B_2) \in \mathcal{C}_2 \)。
  \item 若\(A_1 \times A_2 \subset B_1 \times B_2 \),那么由\ref{lem:1.4.9}中的\ref{ite:3}知\(A_1 \subset B_1 \),\(A_2 \subset B_2 \)。
    由于\(A_i, B_i \in \mathcal{A}_i\),那么\(\exists C^i_k \in \mathcal{A}_i,1 \leq k \leq N_i, N_i \in \mathbb{N}_{+} \)两两不交,与\(A_i \)也不交,且\(B_i=A_i \cup (\bigcup_{1 \leq k \leq N_i} C^i_k),i=1,2\)。
    由于\(C^i_k \in \mathcal{A}_i, 1 \leq k \leq N_i,i=1,2\),那么\(C^1_k \times B_2 \in \mathcal{C}_2, 1 \leq k \leq N_1 \), \(A_1 \times C^2_k \in \mathcal{C}_2, 1 \leq k \leq N_2 \)。
    由于\(C^i_k \in \mathcal{A}_i, 1 \leq k \leq N_i,i=1,2\)两两不交,那么\(C^1_k \times B_2, 1 \leq k \leq N_1\)两两不交,\(A_1 \times C^2_k,1 \leq k \leq N_2 \)两两不交。
    又由于\ref{lem:1.4.9}中的\ref{ite:2}知,
        \(B_1 \times B_2 = (A_1 \times A_2) \cup ((B_1 / A_1) \times B_2) \cup (A_1 \times (B_2 / A_2)) \)。那么
\[
 \begin{aligned}
   B_1 \times B_2 =& (A_1 \times A_2) \cup (( \bigcup_{1 \leq k \leq N_1} C^1_k )\times B_2 )\cup (A_1 \times (\bigcup_{1 \leq k \leq N_2} C^2_k))\\ 
   =&(A_1 \times A_2) \cup \bigcup_{1 \leq k \leq N_1} (C^1_k \times B_2) \cup \bigcup_{1 \leq k \leq N_2}(A_1 \times C^2_k)\\
 \end{aligned}
\]
又由\((A_1 \times A_2) , ((B_1 / A_1) \times B_2) , (A_1 \times (B_2 / A_2)) \)两两不交,
        从而\((A_1 \times A_2) , (\bigcup_{1 \leq k \leq N_1} (C^1_k \times B_2) ), ( \bigcup_{1 \leq k \leq N_2} (A_1 \times C^2_k)) \),
        故\((A_1 \times A_2) , (C^1_k \times B_2 ), (A_1 \times C^2_j), 1 \leq  k \leq N_1, 1 \leq j \leq N_2 \)两两不交。
        从而\(B_1 \times B_2 \)能表示成\(A_1 \times A_2 \) 与\(\mathcal{C}_2 \)中元素的不交并。
    \end{itemize}
  \item 设\(n=k, 1 \leq k \leq n-1\)时,\(\mathcal{C}_k:=\{A_1 \times \cdots \times A_k:A_i \in \mathcal{A}_i, 1 \leq i \leq k\} \)为半集代数。
    那么\[
      \begin{aligned}
        \mathcal{C}_{k + 1}:=&\{A_1 \times \cdots \times A_{k + 1}: A_i \in \mathcal{A}_i, 1 \leq i \leq k + 1\}\\
        =&\{(A_1 \times \cdots \times A_k)\times A_{k + 1}: A_i \in \mathcal{A}_i, 1 \leq i \leq k + 1\} \\ 
        =&\{C \times A: C \in \mathcal{C}_k ,A \in \mathcal{A}_{k + 1}\}.
      \end{aligned}
    \]
    由\(n=2 \)的情形可知\(\mathcal{C}_{k + 1} \)为半集代数。
\end{itemize}
\end{solution}

\begin{problem}\label{pro:1.4.11}
  举例说明可加测度未处有限可加。
\end{problem}
\begin{problem}
  举例说明半集代数\(\mathcal{T}\)生成的\(\sigma\)代数不能一般性地表述为
  \[
    \sigma(\mathcal{T})=\{\sum_{n=1}^{\infty} A_n:\forall n \geq 1, A_n \in \mathcal{T}\}.
  \]
  但如果\(\omega\)至多可数时,如上表述是正确的.
\end{problem}
\begin{solution}
  
\end{solution}

\end{document}


