%!Mode:: "TeX:UTF-8"
%!TEX encoding = UTF-8 Unicode
%arara: xelatex
\documentclass{ctexart}
\newif\ifpreface
%\prefacetrue
\input{../../../global/all}
\begin{document}
\large
\setlength{\baselineskip}{1.2em}
\ifpreface
    \input{../../../global/preface}
\newgeometry{left=2cm,right=2cm,top=2cm,bottom=2cm}
\else
\newgeometry{left=2cm,right=2cm,top=2cm,bottom=2cm}
\maketitle
\fi
%from_here_to_type
\begin{problem}\label{pro:10}
  举例说明半集代数\(\mathcal{T}\)生成的\(\sigma\)-代数不能一般性地表述为
  \[
    \sigma(\mathcal{T})=\{\sum_{n=1}^{\infty} A_n:\forall n \geq 1, A_n \in \mathcal{T}\}.
  \]
  但如果\(\Omega\)至多可数时,如上表述是正确的.
\end{problem}
\begin{solution}
  下证如上表述在\(\Omega \)可数时,不正确。
 考虑\(\mathcal{T}:=\{A \subset \mathbb{N}: 0 \in A, |A^c| < \infty \text{或}
  0 \notin A, |A| < \infty  \} \)。令\(\mathcal{A}:=\{\sum_{n=1}^{\infty}A_n:\forall n \geq 1, A_n \in \mathcal{T}\} \),
先证\(\mathcal{T} \)为半集代数。
  \begin{itemize}
    \item 由于\(0 \notin \varnothing\),且\(|\varnothing| =0 \),那么\(\varnothing \in \mathcal{T} \)。
      又由于\(0 \in \mathbb{N} \),且\(|\mathbb{N}^c|=|\varnothing|=0 \),那么\(\mathbb{N} \in \mathcal{T} \)。
    \item \(\forall A, B \in \mathcal{T} \),若\(0 \in A, 0 \in B\),那么\(0 \in A \cap B \)。
      由于\(|A^c|,|B^c|< \infty \),那么\(|(A \cap B)^c|=|A^c \cup B^c|\leq |A^c| + |B^c| < \infty \)。
      故\(A \cap B \in \mathcal{T} \)。若\(0 \in A, 0 \notin B \),那么\(0 \notin A \cap B \)。
      由于\(|A^c|,|B| < \infty \), 那么\(|A \cap B | \leq |B| < \infty \)。从而\(A \cap B \in \mathcal{T} \)。
      若\(0 \notin A, 0 \notin B \),那么\(0 \notin A \cap B \)。
      又由于\(|A|,|B|< \infty \),那么\(|A \cap B| \leq |A| < \infty \)。
      从而\(A \cap B \in \mathcal{T} \)。综上,\(A \cap B \in \mathcal{T} \)。
    \item \(\forall A \in \mathcal{T} \),若\(0 \in A \),那么\(|A^c|< \infty \)。又由于\(0 \notin A^c \),
      那么\(A^c \in \mathcal{T} \)。若\(0 \notin A \),那么\(|A| < \infty \)。又由于\(0 \in A^c \),
      那么\(|(A^c)^c|< \infty \)。故\(A^c \in \mathcal{T} \)。综上,\(A^c \in \mathcal{T} \)。
  \end{itemize}
   下证\(\sigma(\mathcal{T}) \neq \mathcal{A} \)。 事实上,\(\{0\} \in \sigma(\mathcal{T}) \setminus \mathcal{A} \)。
   若\(\{0\} \in \mathcal{A} \),那么\(\exists A_n \in \mathcal{T},n \geq 1,A_i \cap A_j = \varnothing ,i \neq j \),使得\(\{0\}=\sum_{n=1}^{\infty}A_n \)。
   由于\(|\{0\}|=1 < \infty \),那么\(|A_n|<\infty ,\forall n \geq 1\)。
   故\(0 \notin A_n ,\forall n \geq 1\)。从而\(0 \notin \cup_{n \geq 1}A_n \)。
   故\(\{0\} \notin \mathcal{A} \)。而\(\{0\}=(\cup_{k \geq 1}\{k\})^c \),\(\{k\} \in \mathcal{T},k \geq1 \)。故\(\{0\} \in \sigma(\mathcal{T}) \)。
\end{solution}

\begin{problem}\label{pro:21}
  设\(\mu^* \)为\(\mu \)生成的外测度,则测度空间\((\Omega,\mathcal{A},\mu) \)是完全的\(\iff \) \(\mathcal{A} \supset \{A \subset  \Omega:\mu^*(A)=0\} \).
\end{problem}
\begin{solution}
    由于\((\Omega,\mathcal{A},\mu) \)为测度空间,那么\(\mu \)的外侧度\(\mu^* \)可定义为
  \[
    \mu^*(A):=\{\sum_{n}\mu(A_n):A \subset \cup_n A_n, A_n \in \mathcal{A}\}, A \subset \Omega
  \]
  ``\(\implies\)'':由于\((\Omega,\mathcal{A},\mu) \)为完全的,那么\(\forall N \)满足\(\exists B \in \mathcal{A} \),\(N \subset B \),\(\mu(B)=0 \),都有\(N \in \mathcal{A} \)。
  若 \(\mu^*(A)=0\),那么\(\forall k \in \mathbb{N}_{+} \),\(\exists A_{n,k} \in \mathcal{A}, n \in \mathbb{N} \),满足\(\cup_{n} A_{n,k} \supset A \),\( \mu(\cup_n A_{n,k}) \leq \sum_n \mu(A_{n,k}) < \mu^*(A) + \frac{1}{k} =\frac{1}{k}\)。
  那么\( \mathcal{A}\ni  B:=\cap_k \cup_n A_{n,k} \supset A \)。而由于\(\mu(\cap_k^m \cup_n A_{n,k} ) < \frac{1}{m}<\infty, m \in \mathbb{N}\),由\(\mu \)的连续性可知,
  \(0 \leq \mu(B)=\mu(\cap_k \cup_n A_{n,k}) = \lim_{m \to \infty } \mu(\cap_k^m \cup_n A_{n,k}) \leq \lim_{m \to \infty} \frac{1}{m} =0 \)。故\(\mu(B)=0 \).
  从而\(A \)为\(\mu \)零测集。故\(A \in \mathcal{A} \)。

  ``\(\impliedby\)'':\(\forall N \)满足\(\exists B \in \mathcal{A} \),\(N \subset B \),\(\mu(B)=0 \),都有\(\mu^*(N) \leq \mu(B)=0 \)。那么\(\mu^*(N)=0 \).
  从而\(N \in \mathcal{A} \)。故\((\Omega, \mathcal{A}, \mu) \)为完全测度空间。
\end{solution}

\begin{problem}\label{pro:22}
  \(\mathcal{T} \)为半集代数,\(\mu \)为\(\mathcal{T} \)上的有限测度。记\((\Omega,\mathcal{A}^*,\mu^*) \)为\(\mu \)扩张至\(\sigma(\mathcal{T}) \)的完全化,令 
  \[
    \mu_*(A):=\sup \{\sum_{n}\mu(A_n):A_n \in \mathcal{T} \text{两两不交},\sum_{n}A_n \subset A\},
  \]
  \[
    \mathcal{A}_*:=\{A \subset \Omega:\mu^*(A)=\mu_*(A)\}.
  \]
  证明:\(\mathcal{A}^* \supset \mathcal{A}_* \)。
\end{problem}
\begin{solution}
  \(\forall A \in \mathcal{A}_* \),那么\(\mu_*(A)=\mu^*(A) \)。由\(\mu_* \)的定义及 \(\mu \)有限知,\(\forall k \in \mathbb{N}_{+} \),\(\exists A_{n,k} \in \mathcal{T} , n \in \mathbb{N}\),两两不交,且\(\cup_n A_{n,k} \subset A \),
  满足\(\sum_{n}\mu(A_{n,k}) > \mu_*(A)-\frac{1}{k} \)。由于\(A_{n,k} \)两两不交,那么\(\mu(\cup_n A_{n,k})=\sum_{n}\mu(A_{n,k}) \)。
  由于书本定理\(1.51 \)知,\(\mu^* \)即为\(\mu \)的外测度,\(\mathcal{A}^* \)上的测度。
  故\(\mu^*(\cup_n A_{n,k})=\mu(\cup_n A_{n,k})=\sum_{n}\mu(A_{n,k}) \)。
  令\(B:= \cup_k \cup_n A_{n,k} \in \sigma(\mathcal{T}) \subset \mathcal{A}^*\),那么\(B= \cup_{m=1}^{\infty} \cup_{k=1}^{m} \cup_{n=0}^{\infty}A_{n,k} \subset A \subset \Omega\),\(\mu^*(B) \leq \mu^*(A) \leq \mu^*(\Omega)<\infty \)。
  且\(A_{n,k} \)两两不交\(\forall n,k \)。
  由于\(\mu^*(B)=\lim_{m \to \infty }\mu^*(\cup_{k=1}^m \cup_{n=0}^{\infty}A_{n,k}) \geq \lim_{m \to \infty} \mu^*(\cup_n A_{n,m})=\lim_{m \to \infty}\mu(\cup_n A_{n,m})=\lim_{m \to \infty} \mu_*(A)-\frac{1}{m}=\mu_*(A)\),
  从而\(\mu^*(B) \geq \mu_*(A)=\mu^*(A) \)。故\(\mu^*(B)=\mu^*(A) \)。令\(C=A\setminus B \),由于\(B \in \mathcal{A}^* \),那么\(\mu^*(A)=\mu^*(A \cap B) + \mu^*(A \cap B^c)=\mu^*(B) + \mu^*(C) \)。
  从而\(\mu^*(C)=0 \)。那么\(C \)为\(\mu^* \)-零测集。故\(A=B \cup C \)。由\((\Omega,\mathcal{A}^*,\mu^*) \) 完全及书本定理\(1.51 \)知,\(\mathcal{A}^*=\{B \cup N: B \in \mathcal{A}^*,N \text{为}\mu^* \text{零测集} \} \)。
  从而,\(A \in \mathcal{A}^* \)。

  % 事实上,\(\mathcal{A}^* \supsetneq \mathcal{A}_* \)。下证存在\(A \in \mathcal{A}^* \setminus \mathcal{A}_* \)。
  % 考虑\(\Omega=\mathbb{R}, \mathcal{T}:=\{\text{有限集或余有限集}\} \),\(\mu(A)=\begin{cases}
  % 0, & A \text{有限}\\
  % 1, & A \text{余有限}
  % \end{cases} \)。
\end{solution}

\begin{problem}\label{pro:23}
 设\((\Omega,\mathcal{A},\mu) \)为测度空间,\(\mu^* \)为由\(\mu \)生成的外测度。证明\(N \subset \Omega \)为\(\mu  \)零测集当且仅当\(\mu^*(N)=0 \). 
\end{problem}
\begin{solution}
``\(\implies\)'': 若\(N \)为\(\mu \)零测集,那么\(\exists B \in \mathcal{A} \),\(\mu(B)=0 \),\(N \subset B \)。
那么\(\mu^*(N) \leq \mu(B)=0 \)。故\(\mu^*(N)=0 \)。 

``\(\impliedby\)'': 若\(\mu^*(N)=0 \),那么\(\mu^*(N)=\inf\{\sum_{n}\mu(A_n):A_n \in \mathcal{A}, n \in \mathbb{N}, \cup_n A_n \supset N\}=0\)。故\(\forall k  \geq 1, \exists A_{n,k} \in \mathcal{A}, n  \geq 1, \)满足\(\cup_n A_{n,k} \supset N, \sum_{n}\mu(A_{n,k}) < \mu^*(N) + \frac{1}{k}=\frac{1}{k} \)。
令\(B := \cap_k \cup_n A_{n,k} \),则\(B \supset N, B \in \mathcal{A} \)。那么\(\mu(B)=\mu(\cap_{m} \cap_{ 1 \leq k \leq m} \cup_{n}A_{n,k})=\lim_{m \to \infty}\mu(\cap_{1 \leq k \leq m} \cup_n A_{n,k}) \leq \lim_{m \to \infty} \mu(\cup_n A_{n,m}) = \lim_{m \to \infty} \frac{1}{m} =0  \)。
那么\(N \)为\(\mu \)零测集。
\end{solution}
\begin{problem}\label{pro:2.5.2}
  \begin{enumerate}
    \item   设\(g \)是\((\overline{\mathbb{R}}^n,\overline{\mathcal{B}}^n) \)上的实(复)可测函数,\(f_1,\cdots,f_n \)是\((\Omega,\mathcal{A}) \)上的实可测函数。
      则\(g(f_1,\cdots,f_n) \)是\((\Omega,\mathcal{A}) \)上的实(复)可测函数。
    \item   设\(g \)是\((\overline{\mathbb{C}}^n,\overline{\mathcal{B}}^n_c) \)上的实(复)可测函数,\(f_1,\cdots,f_n \)是\((\Omega,\mathcal{A}) \)上的复可测函数。
      则\(g(f_1,\cdots,f_n) \)是\((\Omega,\mathcal{A}) \)上的实(复)可测函数。
  \end{enumerate}
\end{problem}
\begin{solution}
  \begin{enumerate}
    \item 由定理2.6(2)可知\(F:=(f_1,\cdots,f_n) \)是{ \((\Omega,\mathcal{A}) \) }上的{实(复)可测函数}。
      故由定理2.7可得\(g \circ F=g(f_1,\cdots,f_n) \)是\((\Omega,\mathcal{A}) \)上的{实(复)可测函数}。
    \item 由定理2.6(2)可知\(F:=(f_1,\cdots,f_n) \)是{ \((\Omega,\mathcal{A}) \) }上的{实(复)可测函数}。
      故由定理2.7可得\(g \circ F=g(f_1,\cdots,f_n) \)是\((\Omega,\mathcal{A}) \)上的{实(复)可测函数}。
  \end{enumerate}
\end{solution}

\begin{problem}\label{pro:2.5.3}
  \begin{enumerate}
    \item   设\(g \)是\((\overline{\mathbb{R}}^n,\overline{\mathcal{B}}^n) \)上的实(复)可测函数,\(f_1,\cdots,f_n \)是\((\Omega,\mathcal{A},\mathbb{P}) \)上的随机变量。
      且\(\mathbb{P}(|g(f_1,\cdots,f_n)|=\infty)=0 \)。 则\(g(f_1,\cdots,f_n) \)是\((\Omega,\mathcal{A}) \)上的实(复)随机变量。
    \item   设\(g \)是\((\overline{\mathbb{C}}^n,\overline{\mathcal{B}}^n_c) \)上的实(复)可测函数,\(f_1,\cdots,f_n \)是\((\Omega,\mathcal{A},\mathbb{P}) \)上的复随机变量。
      且\(\mathbb{P}(|g(f_1,\cdots,f_n)|=\infty)=0 \)。 则\(g(f_1,\cdots,f_n) \)是\((\Omega,\mathcal{A}) \)上的实(复)随机变量。
  \end{enumerate}
\end{problem}
\begin{solution}
  \begin{enumerate}
    \item 由于\((\Omega,\mathcal{A},\mathbb{P}) \)上的随机变量是一种实可测函数,故由题目 \ref{pro:2.5.2} 可知,\(g(f_1,\cdots,f_n) \)为 
  \((\Omega,\mathcal{A}) \)上的实(复)可测函数。又由于\(\mathbb{P}(|g(f_1,\cdots,f_n)|=\infty)=0 \),故\(g(f_1,\cdots,f_n) \)为实(复)可测函数。
\item 由于\((\Omega,\mathcal{A},\mathbb{P}) \)上的随机变量是一种实可测函数,故由题目 \ref{pro:2.5.2} 可知,\(g(f_1,\cdots,f_n) \)为 
  \((\Omega,\mathcal{A}) \)上的实(复)可测函数。又由于\(\mathbb{P}(|g(f_1,\cdots,f_n)|=\infty)=0 \),故\(g(f_1,\cdots,f_n) \)为实(复)可测函数。
  \end{enumerate}

\end{solution}

\end{document}
