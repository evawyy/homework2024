%!Mode:: "TeX:UTF-8"
%!TEX encoding = UTF-8 Unicode
%!TEX language = zh
%arara: xelatex
\documentclass{ctexart}
\newif\ifpreface
%\prefacetrue
\input{../../../global/all}
\begin{document}
\large
\setlength{\baselineskip}{1.2em}
\ifpreface
\input{../../../global/preface}
\newgeometry{left=2cm,right=2cm,top=2cm,bottom=2cm}
\else
\newgeometry{left=2cm,right=2cm,top=2cm,bottom=2cm}
\maketitle
\fi
%from_here_to_type
\begin{problem}\label{pro:2}
  设\(\mathcal{C} \)为\(\mathcal{A} \)的子\(\sigma \)代数,\(\phi(B)=\int_{B} \xi \d \mathbb{P} , B \in \mathcal{C}  \), 则\(\phi  \)是\( \mathcal{C} \) 上\(\sigma \)可加集函数,
  \(\phi \ll \res{\mathbb{P}}{\mathcal{C}}\)。从而存在\(\mathbb{E}(\xi \mid \mathcal{C}) = \frac{\d \phi}{\d \mathbb{P}_{\mathcal{C}}}, \mathbb{P}_{\mathcal{C}}\)-a.e.
\end{problem}
\begin{problem}\label{pro:5}
  证明赫尔德不等式\(\mathbb{E}(\xi \eta \mid \mathcal{C}) \leq \mathbb{E}(|\xi|^p \mid \mathcal{C})^{\frac{1}{p}} \mathbb{E}(|\eta|^q \mid \mathcal{C})^{\frac{1}{q}}, p >1, \frac{1}{p} + \frac{1}{q} =1 \)。
\end{problem}
\begin{problem}\label{pro:7}
  叙述并证明关于条件期望 Jensen 不等式和 Minikowski 不等式。
\end{problem}
\begin{problem}\label{pro:9}
  设\(\mathcal{C}_1,\mathcal{C}_2 \)为\(\mathcal{A} \)的子\(\sigma \)代数。举例说明\(\mathbb{E}(\xi \mid \mathcal{C}_1 \cap \mathcal{C}_2) \neq \mathbb{E}(\mathbb{E}(\xi \mid \mathcal{C}_1) \mid \mathcal{C}_2) \)。
\end{problem}

\begin{problem}\label{pro:11}
  设\(\mathcal{A}_n \)是一列单调上升的子\(\sigma \)代数,若随机变量\(\xi_n \)满足
  \[
    \mathbb{E}(\xi_{n + 1} \mid \mathcal{A}_n)=\xi_n,n \geq 1
  \]则称为鞅序列。证明\(\xi_n=\mathbb{E}(\xi \mid \mathcal{A}_n) \)是鞅。
\end{problem}
\begin{problem}\label{pro:12}
  设\(\xi_n \)为随机变量,令\(\mathcal{A}_n =\sigma(\{\xi_m:m \leq n\})\)。若
\end{problem}

\end{document}
