%!Mode:: "TeX:UTF-8"
%!TEX encoding = UTF-8 Unicode
%arara: xelatex
\documentclass{ctexart}
\newif\ifpreface
%\prefacetrue
\input{../../../global/all}
\begin{document}
\large
\setlength{\baselineskip}{1.2em}
\ifpreface
\input{../../../global/preface}
\newgeometry{left=2cm,right=2cm,top=2cm,bottom=2cm}
\else
\newgeometry{left=2cm,right=2cm,top=2cm,bottom=2cm}
\maketitle
\fi
%from_here_to_type
\begin{problem}\label{pro:5}
  分布函数是否是不降的?举出反例或者给出证明。
\end{problem}

\begin{problem}\label{pro:7}
  证明若\(F(x)=\mathbb{P}(\xi < x) \)是连续的,则\(\eta = F(\xi) \)具有\((0,1) \)上的均匀分布。
\end{problem}
\begin{solution}
  令\(\eta = F(\xi ) \),只需证明\(\forall x \in (0,1) \),\(\mathbb{P}(\eta < x)=x \)。令\(G(x)=\inf\{y: F(y) \geq x\} \)。
  下证\(\forall x \in (0,1)\),\( F^{-1}(\{x\}) =\varnothing\)或者\(\exists a,b \in \mathbb{R},a \leq b ,F^{-1}(\{x\})=[a,b] \)。
  若\(F^{-1}(\{x\})=\varnothing \),满足要求,故只需考虑\(F^{-1}(\{x\} ) \neq \varnothing\)。不妨\(z \in F^{-1}(\{x\}) \),\(a =\inf F^{-1}(\{x\}),b = \sup F^{-1}(\{x\}) \)。
  \(\forall n \in \mathbb{N}_{+} \),\(\exists a_n < a + \frac{1}{n}, b_n > b - \frac{1}{n} \),\(a_n,b_n \in F^{-1}(\{x\}) \)。故\(F(a_n) \leq F(a + \frac{1}{n}), F(b_n) \geq F(b- \frac{1}{n}) \)。
  从而\(x=\lim_{n}F(a_n) \leq \lim_{n}F(a + \frac{1}{n}), x= \lim_{n}F(b_n) \geq \lim_{n}F(b -\frac{1}{n}) \)。由于\(F \)连续,从而\(\lim_{n}F(a + \frac{1}{n})=F(a),\lim_{n}F(b-\frac{1}{n})=F(b) \)。
  故\(F(a) \geq x, F(b ) \leq x \)。
  若\(F(a) > x \),由于\(F \)连续且不降,\(z \in F^{-1}(\{x\}) \),\(z \geq a \),那么\(F(z) \geq F(a) >x \)。与\(z \in F^{-1}(\{x\}) \)矛盾。
  故\(F(a)=x \)。
  若\(F(b) < x \),由于\(F \)连续且不降,\(z \in F^{-1}(\{x\}) \),\(z \leq b \),那么\(F(z) \leq F(b) <x \)。与\(z \in F^{-1}(\{x\}) \)矛盾。
  故\(F(b)=x \)。
  \(\forall z \in (a,b) \),由于\(F \)不降,那么\( x=F(a) \leq F(z)\leq F(b)=x \)。从而\(F(z)=x \)。故\([a,b] \subset F^{-1}(\{x\})\)。
  \(\forall z <a \),那么\(F(z) \leq F(a)=x \)。若\(F(z) =x \),那么\(z \in F^{-1}(\{x\}) \)。从而\(a \leq z \),与\(a \)为下确界矛盾。
  故\(F(z) < x \)。
  \(\forbll z <b \),那么\(F(z) \leq F(b)=x \)。若\(F(z) =x \),那么\(z \in F^{-1}(\{x\}) \)。从而\(b \leq z \),与\(b \)为下确界矛盾。
  故\(F(z) < x \)。
  下证\(\eta < x \iff  \)
\end{solution}

\begin{problem}\label{pro:9}
  设\(\xi_n,n \in \mathbb{N}_{+} \)为\(\mathrm{i.i.d.}  \)随机变量,分布为\(\mu \)。
  给定\(A \in \mathcal{B} \),\(\mu(A) >0 \),定义\(\tau = \inf\{k:\xi_k \in A\} \)。
  证明\(\xi_{\tau} \)的分布为\(\frac{\mu(\cdot \cap A)}{\mu(A)} \)。
\end{problem}
\begin{solution}
  \(\forall B \in \mathcal{C} \),\(\mathbb{P}(\xi_{\tau} \in B)=\sum_{n=1}^{\infty} \mathbb{P}(\xi_{\tau} \in B | \tau =n)\mathbb{P}(\tau =n) \)。
  考虑到\(\{\tau = n\} = \{\xi_n \in A, \xi_k \notin A, 1 \leq k \leq n-1\} \), \(\{\xi_{\tau} \in B | \tau =n\}=\{\xi_n \in B | \xi_n \in A, \xi_k \notin A, 1 \leq k \leq n-1\}\),
  及\(\xi_k, 1 \leq k \leq n-1 \)与\(\xi_n \)独立,从而\(\{\xi_{\tau} \in B| \tau =n\} = \{\xi_n \in B | \xi_n \in A\}=\{\xi_n \in A \cap B | \xi_n \in A\} \)。
  从而,\[
    \begin{aligned}
      \mathbb{P}(\xi_{\tau} \in B) &= \sum_{n=1}^{\infty}\mathbb{P}(\xi_n \in A \cap B | \xi_n \in A)\mathbb{P}(\tau =n)\\
      &=\sum_{n=1}^{\infty}\frac{\mathbb{P}(\xi_n \in A \cap B)}{\mathbb{P}(\xi_n \in A)}\mathbb{P}(\tau =n) \\
      &=\frac{\mathbb{P}(\xi_n \in A \cap B)}{\mathbb{P}(\tau \in A)}\\
      &=\frac{\mu(B \cap A)}{\mu(A)}
    \end{aligned}
  \]
\end{solution}

\begin{problem}\label{pro:11}
  若\(\mathcal{C}_1,\cdots,\mathcal{C}_n \)为独立的\(\pi \)系,那么\(\sigma(\mathcal{C}_1),\cdots,\sigma(\mathcal{C}_n)\)独立。
\end{problem}
\begin{solution}
  该题的证明需要\(\Omega \in \mathcal{C}_i,1 \leq i \leq n \)。

  只需证明若\(\forall 1 \leq l \leq n,  1 \leq k \leq n, \mathcal{C}_k\)为独立的\(\pi \)系且\(\Omega \in \mathcal{C}_k \),那么\(\mathcal{C}_k, 1 \leq k \leq n, k \neq l, \sigma(\mathcal{C}_l)  \)独立。
  考虑\[\mathcal{A}:=\{A \in \sigma(\mathcal{C}_l): \forall J \subset 2^{\{1,\cdots,n\}}\setminus \{\varnothing\},\mathbb{P}(A \cap \bigcap_{i \in J \setminus \{l\}}A_i)=\mathbb{P}(A)\prod_{i \in J\setminus\{l\}}\mathbb{P}(A_i), A_i \in \mathcal{C}_i, i \in \{1,\cdots,n\}\setminus\{l\}\} \]。

\end{solution}

\begin{problem}\label{pro:12}
  \begin{enumerate}
    \item 设\(\{A_n\}_{n \geq 1} \) 为独立事件序列,令\(\mathcal{J} = \bigcap_{n=1}^{\infty} \sigma\{A_n,A_{n + 1},\cdots\}. \)
      证明\(\forall A \in \mathcal{J} \),有\(\mathbb{P} (A) =0 \)或\(1 \).
    \item 设 \(\{\xi_n\}_{n \geq 1} \)为独立随机变量,令\(\mathcal{J} = \bigcap_{n=1}^{\infty} \sigma\{\xi_n,\xi_{n + 1},\cdots\} \)。
      证明\(\forall A \in \mathcal{J} \),有\(\mathbb{P}(A)=0 \)或\(1 \).
  \end{enumerate}
\end{problem}

\end{document}
