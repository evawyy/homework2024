%!Mode:: "TeX:UTF-8"
%!TEX encoding = UTF-8 Unicode
%arara: xelatex
\documentclass{ctexart}
\newif\ifpreface
%\prefacetrue
\input{../../../global/all}
\begin{document}
\large
\setlength{\baselineskip}{1.2em}
\ifpreface
\input{../../../global/preface}
\newgeometry{left=2cm,right=2cm,top=2cm,bottom=2cm}
\else
\newgeometry{left=2cm,right=2cm,top=2cm,bottom=2cm}
\maketitle
\fi
%from_here_to_type
\begin{problem}\label{pro:5}
  分布函数是否是不降的?举出反例或者给出证明。
\end{problem}

\begin{problem}\label{pro:7}
  证明若\(F(x)=\mathbb{P}(\xi < x) \)是连续的,则\(\eta = F(\xi) \)具有\((0,1) \)上的均匀分布。
\end{problem}
\begin{solution}
 \begin{lemma}\label{lem:pro7}
   \(F(x) \) 为随机变量\(\xi \)的分布函数,若\(F \)为左连续,右极限存在,那么\(\forall x \in \mathbb{R}, F^{-1}(\{x\}) \)或为\(\varnothing \),或为\((a,b],a,b \in \mathbb{R} \)。
 \end{lemma}
\begin{proof}
 \(\forall x \in \mathbb{R} \),若\(F^{-1}\{x\} \neq \varnothing\),需证明\(\exists a, b \in \mathbb{R} \), 
\end{proof}
  
\end{solution}

\begin{problem}\label{pro:9}
  设\(\xi_n,n \in \mathbb{N}_{+} \)为\(\mathrm{i.i.d.}  \)随机变量,分布为\(\mu \)。
  给定\(A \in \mathcal{B} \),\(\mu(A) >0 \),定义\(\tau = \inf\{k:\xi_k \in A\} \)。
  证明\(\xi_{\tau} \)的分布为\(\frac{\mu(\cdot \cap A)}{\mu(A)} \)。
\end{problem}
\begin{solution}
  \(\forall B \in \mathcal{C} \),\(\mathbb{P}(\xi_{\tau} \in B)=\sum_{n=1}^{\infty} \mathbb{P}(\xi_{\tau} \in B | \tau =n)\mathbb{P}(\tau =n) \)。 
  考虑到\(\{\tau = n\} = \{\xi_n \in A, \xi_k \notin A, 1 \leq k \leq n-1\} \), \(\{\xi_{\tau} \in B | \tau =n\}=\{\xi_n \in B | \xi_n \in A, \xi_k \notin A, 1 \leq k \leq n-1\}\),
  及\(\xi_k, 1 \leq k \leq n-1 \)与\(\xi_n \)独立,从而\(\{\xi_{\tau} \in B| \tau =n\} = \{\xi_n \in B | \xi_n \in A\}=\{\xi_n \in A \cap B | \xi_n \in A\} \)。
  从而,\[
    \begin{aligned}
      \mathbb{P}(\xi_{\tau} \in B) &= \sum_{n=1}^{\infty}\mathbb{P}(\xi_n \in A \cap B | \xi_n \in A)\mathbb{P}(\tau =n)\\
      &=\sum_{n=1}^{\infty}\frac{\mathbb{P}(\xi_n \in A \cap B)}{\mathbb{P}(\xi_n \in A)}\mathbb{P}(\tau =n) \\ 
      &=\frac{\mathbb{P}(\xi_n \in A \cap B)}{\mathbb{P}(\tau \in A)}\\ 
      &=\frac{\mu(B \cap A)}{\mu(A)}
    \end{aligned}
  \]
\end{solution}

\begin{problem}\label{pro:11}
  若\(\mathcal{C}_1,\cdots,\mathcal{C}_n \)为独立的\(\pi \)系,那么\(\sigma(\mathcal{C}_1),\cdots,\sigma(\mathcal{C}_n)\)独立。
\end{problem}
\begin{solution}
  该题的证明需要\(\Omega \in \mathcal{C}_i,1 \leq i \leq n \)。

  只需证明若\(\forall 1 \leq l \leq n,  1 \leq k \leq n, \mathcal{C}_k\)为独立的\(\pi \)系且\(\Omega \in \mathcal{C}_k \),那么\(\mathcal{C}_k, 1 \leq k \leq n, k \neq l, \sigma(\mathcal{C}_l)  \)独立。
  考虑\[\mathcal{A}:=\{A \in \sigma(\mathcal{C}_l): \forall J \subset 2^{\{1,\cdots,n\}}\setminus \{\varnothing\},\mathbb{P}(A \cap \bigcap_{i \in J \setminus \{l\}}A_i)=\mathbb{P}(A)\prod_{i \in J\setminus\{l\}}\mathbb{P}(A_i), A_i \in \mathcal{C}_i, i \in \{1,\cdots,n\}\setminus\{l\}\} \]。

\end{solution}

\begin{problem}\label{pro:12}
  \begin{enumerate}
    \item 设\(\{A_n\}_{n \geq 1} \) 为独立事件序列,令\(\mathcal{J} = \bigcap_{n=1}^{\infty} \sigma\{A_n,A_{n + 1},\cdots\}. \)
      证明\(\forall A \in \mathcal{J} \),有\(\mathbb{P} (A) =0 \)或\(1 \).
    \item 设 \(\{\xi_n\}_{n \geq 1} \)为独立随机变量,令\(\mathcal{J} = \bigcap_{n=1}^{\infty} \sigma\{\xi_n,\xi_{n + 1},\cdots\} \)。
      证明\(\forall A \in \mathcal{J} \),有\(\mathbb{P}(A)=0 \)或\(1 \).
  \end{enumerate}
\end{problem}

\end{document}
