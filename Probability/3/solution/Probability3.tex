%!Mode:: "TeX:UTF-8"
%!TEX encoding = UTF-8 Unicode
%arara: xelatex
\documentclass{ctexart}
\newif\ifpreface
%\prefacetrue
\input{../../../global/all}
\begin{document}
\large
\setlength{\baselineskip}{1.2em}
\ifpreface
\input{../../../global/preface}
\newgeometry{left=2cm,right=2cm,top=2cm,bottom=2cm}
\else
\newgeometry{left=2cm,right=2cm,top=2cm,bottom=2cm}
\maketitle
\fi
%from_here_to_type
\begin{problem}\label{pro:5}
  分布函数是否是不降的?举出反例或者给出证明。
\end{problem}

\begin{problem}\label{pro:7}
  证明若\(F(x)=\mathbb{P}(\xi < x) \)是连续的,则\(\eta = F(\xi) \)具有\((0,1) \)上的均匀分布。
\end{problem}

\begin{problem}\label{pro:9}
  设\(\xi_n,n \in \mathbb{N}_{+} \)为\(\mathrm{i.i.d.}  \)随机变量,分布为\(\mu \)。
  给定\(A \in \mathcal{B} \),\(\mu(A) >0 \),定义\(\tau = \inf\{k:\xi_k \in A\} \)。
  证明\(\xi_{\tau} \)的分布为\(\frac{\mu(\cdot \cap A)}{\mu(A)} \)。
\end{problem}
\begin{problem}\label{pro:11}
  若\(\mathcal{C}_1,\cdots,\mathcal{C}_n \)为独立的\(\pi \)系,那么\(\sigma(\mathcal{C}_1),\cdots,\sigma(\mathcal{C}_n)\)独立。
\end{problem}
\begin{solution}
  该题的证明需要\(\Omega \in \mathcal{C}_i,1 \leq i \leq n \)。
\end{solution}

\begin{problem}\label{pro:12}
  \begin{enumerate}
    \item 设\(\{A_n\}_{n \geq 1} \) 为独立事件序列,令\(\mathcal{J} = \bigcap_{n=1}^{\infty} \sigma\{A_n,A_{n + 1},\cdots\}. \)
      证明\(\forall A \in \mathcal{J} \),有\(\mathbb{P} (A) =0 \)或\(1 \).
    \item 设 \(\{\xi_n\}_{n \geq 1} \)为独立随机变量,令\(\mathcal{J} = \bigcap_{n=1}^{\infty} \sigma\{\xi_n,\xi_{n + 1},\cdots\} \)。
      证明\(\forall A \in \mathcal{J} \),有\(\mathbb{P}(A)=0 \)或\(1 \).
  \end{enumerate}
\end{problem}

\end{document}
