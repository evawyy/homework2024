%!Mode:: "TeX:UTF-8"
%!TEX encoding = UTF-8 Unicode
%!TEX language = zh
%arara: xelatex
\documentclass{ctexart}
\newif\ifpreface
%\prefacetrue
\input{../../../global/all}
\begin{document}
\large
\setlength{\baselineskip}{1.2em}
\ifpreface
\input{../../../global/preface}
\newgeometry{left=2cm,right=2cm,top=2cm,bottom=2cm}
\else
\newgeometry{left=2cm,right=2cm,top=2cm,bottom=2cm}
\maketitle
\fi
%from_here_to_type

\begin{problem} \label{pro:3.6.1}
  设\(f \)的积分存在,\(\mu \)有限。证明:
  \(\int_{\Omega} f \d \mu = \lim_{n \to \infty} \sum_{i = - \infty}^{\infty} \frac{i}{2^n} \mu \left(\{\frac{i}{2^n} \leq f < \frac{i + 1}{2^n}\}\right) \)。
\end{problem}
\begin{solution}
  令\(A_{n,i}=\{\frac{i}{2^n} \leq f(x) < \frac{i + 1}{2^n}\} \),\(g_n(x)=\sum_{i=-\infty}^{\infty}\frac{i}{2^n}\mathbbm{1}_{A_{n,i}}(x) \)。

  先证\(\lim_{n \to \infty} \int_{\Omega} g_n d \mu = \int_{\Omega} f \d  \mu \)。
  显然有\(g_n(x) \leq f(x)<g_n + \frac{1}{2^n} \)。故\(0 \leq f(x)-g_n(x) < \frac{1}{2^n} \)。从而\(f-g_n \to 0,n \to \infty, \forall x \in \Omega \),且
  \(\int_{\Omega} f-g_n \d \mu \leq \int_{\Omega} \frac{1}{2^n}\d \mu =\frac{1}{2^n} \mu(\Omega)<\infty \),故由LCDT知\(\int_{\Omega} \lim_{n \to \infty }f(x)-g_n(x) \d \mu =\lim_{n \to \infty} \int_{\Omega} f(x)-g_n(x) \d \mu =\int_{\Omega} 0 \d \mu=0 \mu(\Omega)=0\).
  而\(\forall n \), \(g_n=g_n-f + f  \),\(|g_n| \leq |g_n-f| + |f| = |f| + \frac{1}{2^n} \leq |f| + 1\),故\(|g_n| \)积分存在,从而\(g_n \)积分存在。
  由于\(f,g_n-f \)积分存在,那么\(\int_{\Omega} g_n \d \mu = \int_{\Omega} g_n -f \d \mu  + \int_{\Omega} f \d \mu \to 0 + \int_{\Omega} f \d \mu = \int_{\Omega} f \d \mu,n \to \infty  \)。

  下证\( \int_{\Omega} g_n \d \mu = \sum_{i=-\infty}^{\infty} \frac{i}{2^n} \mu(A_{n,i}) \)。
  令\(h_{m,n}= \sum_{i=-m}^{m} \frac{i}{2^n} \mathbbm{1}_{A_{n,i} }(x)\),那么\(h_{m,n} \to g_n,m \to \infty \),
  故\(\int_{\Omega}g_n \d \mu = \int_{\Omega} \lim_{m \to \infty} h_{m,n} \d \mu \)。
  由于\(A_{n,i} \cap A_{n,j}=\varnothing, \forall ij <0  \),那么\(-h_{m,n}^- =\sum_{i=-m}^{0} \frac{i}{2^n} \mathbbm{1}_{A_{n,i}}(x), h^{+ }_{m,n} = \sum_{i=0}^{m} \frac{i}{2^n} \mathbbm{1}_{A_{n,i}}(x), g_n^{-}=-\sum_{i=-\infty}^{0}\frac{i}{2^n}\mathbbm{1}_{A_{n,i}}(x),g_n^{+}=\sum_{i=0}^{\infty} \frac{i}{2^n} \mathbbm{1}_{A_{n,i}}(x)\)。
  又由于\(h_{m + 1,n}^{+} = \sum_{i=0}^{m + 1} \frac{i}{2^n} \mathbbm{1}_{A_{n,i}}(x)=\sum_{i=0}^{m}(\frac{i}{2^n} + \frac{i + 1}{2^n})\mathbbm{1}_{A_{n,i} \cap A_{n,m + 1}}(x) + \frac{m + 1}{2^n} \mathbbm{1}_{A_{n,m + 1} \setminus \bigcup_{i=0}^{m} A_{n,i}}(x) \geq h_{m,n}^{+} \).
  故\(h_{m ,n}^{+} \nearrow_m g_n^{+}\)。同理可知\(h_{m,n}^{-} \nearrow_m g_n^{-} \)。从而\(|h_{m,n}| \leq |g_n| \)
  由于\(g_n \)积分存在,那么\(\int_{\Omega} g_n^{-} \d \mu ,\int_{\Omega} g_n^{+} \d \mu  \)存在,且至少之一有限。从而由非负单调收敛定理知,\(\lim_{m \to \infty } \int_{\Omega} h_{m,n}^{+} \d \mu = \int_{\Omega} \lim_{m \to \infty} h_{m,n}^{+ } \d \mu = \int_{\Omega} g_n^{+} \d \mu  \),
  \(\lim_{m \to \infty }\int_{\Omega} h_{m,n}^{-} \d \mu = \int_{\Omega} \lim_{m \to \infty} h_{m,n}^{-} \d \mu =\int_{\Omega} g_n^{-} \d \mu \)。
  又由于\(\int_{\Omega} h_{m,n}^{+} \d \mu  - \int_{\Omega} h_{m,n}^{-} \d \mu = \int_{\Omega} h_{m, n} \d \mu\),
  故\(\int_{\Omega} g_n \d \mu =\int_{\Omega} g_n^{+ } \d \mu - \int_{\Omega} g_n^{-} \d \mu = \lim_{m \to \infty} \int_{\Omega} h_{m,n}^{+ } \d \mu - \lim_{m \to \infty} \int_{\Omega} h_{m,n}^{-} \d \mu = \lim_{m \to \infty} (\int_{\Omega} h_{m,n}^{+} \d \mu - \int_{\Omega} h_{m,n}^{-} \d \mu) =\lim_{m \to \infty } \int_{\Omega} h_{m,n} \d \mu = \lim_{m \to \infty } \sum_{i=-m}^{m} \frac{i}{2^n} \mu (A_{n,i})=\sum_{i=-\infty}^{\infty} \frac{i}{2^n} \mu(A_{n,i})\)

  从而,\(\int_{\Omega} f \d \mu = \lim_{n \to \infty} \int_{\Omega} g_n \d \mu =\lim_{n \to \infty } \sum_{i=-\infty}^{\infty} \frac{i}{2^n}\mu(A_{n,i})= \lim_{n \to \infty} \sum_{i=-\infty}^{\infty} \frac{i}{2^n} \mu(\{\frac{i}{2^n} \leq f(x) < \frac{i + 1}{2^n}\}) \)。

  对一般的测度\(\mu \)不成立,反例如下:

  取\(\Omega=[1,\infty),\mu \)为勒贝格测度,\(f(x):=-\frac{1}{x^2} \),则易知\(f(x) \)是可积的,故积分存在。
  但对任何\(n \),有\(\mu(\{\frac{-1}{2^n} \leq f < \frac{0}{2^n}\})=\infty \),于是\( \sum_{i = - \infty}^{\infty} \frac{i}{2^n} \mu \left(\{\frac{i}{2^n} \leq f < \frac{i + 1}{2^n}\}\right)=-\infty \),
  故题设不成立。

\end{solution}

\begin{problem}\label{pro:3.6.2}
  设\(f \)为非负可测函数,令:
  \(\overline{\int}_{\Omega} f \d \mu := \inf\left\{\int_{\Omega} g \d \mu:g \geq f,g \text{为简单函数}\right\} \)。
  举例说明\(\overline{\int} \)与\(\int \)未必相同,并解释为何不将积分定义为\(\overline{\int} \)。
\end{problem}
\begin{solution}
  \(\Omega=[1,.\infty),\mu \)为勒贝格测度,\(f=\frac{1}{x^2} \)为非负可测函数。若\(g \)为简单函数且\(g \geq f \),由于\(g \)的值域有限,
  那么\(\exists m \geq 0,x \in \Omega  \),\(m=\min_{x \in \Omega}g(x)=g(x_0) \)。从而\(\int_{\Omega} g \d \mu \geq \int_{\Omega} m \d \mu = m \mu(\Omega)=\infty \)。
  从而\( \overline{\int}_{\Omega} f \d \mu =\infty \)。而\(\int_{\Omega} f \d \mu =1 \),故\(\overline{\int} f \d \mu \neq \int_{\Omega} f \d \mu \)。

  使用\(\int \)而不是\(\overline{\int} \)应该是为了保证所有广义黎曼可积的非负函数都可积。
\end{solution}

\begin{problem}\label{pro:3.6.7}
  设\(\{f_{nm}\}_{n,m \geq 1} \)为一族非负实数。证明\(\liminf_{m \to \infty} \sum_{n = 1}^{\infty}f_{nm} \geq \sum_{n = 1}^{\infty}\liminf_{m \to \infty}f_{nm} \)。
\end{problem}
\begin{solution}
  令\(\Omega=\mathbb{N}_{+},\mu \)为计数测度。记\(g_m(n):=f_{nm} \)。则\(\int_{\Omega} g_m \d \mu =\sum_{n=1}^{\infty} \mu\{n\} g_m(n)=\sum_{n=1}^{\infty}f_{nm} \)。
  由 Fatou 引理知\(\liminf_{m \to \infty} \sum_{n = 1}^{\infty}f_{nm} =\liminf_{m \to \infty} \int_{\Omega} g_m \d \mu \geq \int_{\Omega} \liminf_{m \to \infty} g_m \d \mu =\sum_{n = 1}^{\infty}\liminf_{m \to \infty}f_{nm} \)。
\end{solution}

\begin{problem}\label{pro:3.6.10}
  若\(\xi_n \)依分布收敛于\(\xi \),则\(\mathbb{E} |\xi| \leq \liminf_{n \to \infty}\mathbb{E} |\xi_n| \)。
\end{problem}
\begin{lemma}\label{lem:2}
  \(\xi,\xi_n \)为随机变量,其对应的概率测度分别为\(\mu,\mu_n,n \in \mathbb{N}_{+}  \)在\((E,\mathcal{B}) \)上。
  若\(\xi_n \overset{d}{\to} \xi,n \to \infty  \),那么:
  \begin{enumerate}
    \item \(\liminf_{n}\mu_n((-\infty,x)) \geq \mu((-\infty,x)),\limsup_{n}\mu_n((-\infty,x]) \leq \mu((-\infty,x]) \);
    \item \(F  \in \mathcal{B}\)  为闭集,则\(\limsup_{n} \mu_n(F)  \leq \mu(F)\); \(O \in \mathcal{B} \) 为开集,则\(\liminf_{n} \mu_n(O) \geq \mu(O)\)。
    \item \(A \in \mathcal{B} \),\(\mu(\partial A)=0 \),那么\(\lim_{n} \mu_n(A)=\mu(A) \)。
  \end{enumerate}
\end{lemma}
\begin{proof}
  \begin{enumerate}
    \item 由于\(\mu((-\infty,x)) \)关于\(x \)不降,那么\(\mu((-\infty,x)) \)的不连续点至多可数,记连续点为\(C \)。
      从而,\(\forall x \in \mathbb{R} \),\(\exists x_m \in C,y_m \in C,m \in \mathbb{N}_{+} \)使得\(x_m \searrow_m x,y_m \nearrow_m,x \)。
      那么\(\mu_n((-\infty,x))\geq \mu_n((-\infty,y_m)) \),从而\(\liminf_{n}\mu_n((-\infty,x)) \geq \liminf_{n}\mu_n((-\infty,y_m)) =\mu((-\infty,y_m))\)。
      故\[\liminf_{n} \mu((-\infty,x)) \geq \lim_{m}\mu((-\infty,y_m))=\mu((-\infty,x)) \]。

      又由于\(\mu_n((-\infty,x]) \leq \mu_n((-\infty,x_m)) \),从而\(\limsup_{n}\mu_n((-\infty,x]) \leq \limsup_{n} \mu_n ((-\infty,x_m))=\mu((-\infty,x_m)) \)。
      故\[\limsup_{n}\mu_n((-\infty,x]) \leq \lim_{m} \mu((-\infty,x_m))=\mu((-\infty,x])\]。
    \item 由于\(\forall F \)为闭集,那么\(F^c \)为开集,故\( \limsup_{n} \mu_n(F)=1-\liminf_{n} \mu_n(F^c)\),故只需证明对开集有这个性质即可。
      \(O \in \mathcal{B} \)为开集,那么\(O \)可以表示为可数开区间的不交并。设\(O=\bigcup_{n=1}^{\infty}(a_n,b_n) \),其中\(a_n < b_n ,n \in \mathbb{N}_{+},(a_n,b_n) \cap (a_m,b_m)=\varnothing,n \neq m\)。

      先证\(\forall s,v \in \mathbb{R},s < v \),那么\(\liminf_{n} \mu_n((s,v)) \geq \mu((s,v)) \)。
      由于\(\mu_n((s,v))=\mu_n((-\infty,v))-\mu_n((-\infty,s]) \),故
      \[
        \begin{aligned}
          \liminf_{n}\mu_n((s,v)) &= \liminf_{n}\mu_n((-\infty,v))+ \liminf_{n}(-\mu_n((-\infty,s]))\\
          &=\liminf_{n}\mu_n((-\infty,v)) - \limsup_{n}\mu_n((-\infty,s]) \\
          &\geq \mu((-\infty,v))-\mu((-\infty,s])=\mu((s,v))。
        \end{aligned}
      \]

      由于\(\mu_n(O)=\mu_n (\bigcup_{m \in \mathbb{N}_{+}} (a_m,b_m))=\sum_{m \in \mathbb{N}_{+}} \mu_n((a_m,b_m))\),
      故\[
        \begin{aligned}
          \liminf_{n}\mu_n(O)&=\liminf_{n}\mu_n(\bigcup_{m \in \mathbb{N}_{+}} (a_m,b_m))\\
          &=\liminf_{n}\sum_m \mu_n((a_m,b_m)) \\
          &\overset{\ref{pro:3.6.7}}{\geq } \sum_m \liminf_{n} \mu_n((a_m,b_m))\\
          &\geq \sum_{m} \mu((a_m,b_m))=\mu(\bigcup_{m}(a_m,b_m))=\mu(O)。
        \end{aligned}
      \]
    \item 由于\(A^o \subset A \subset \overline{A} \),那么\(\mu(A^o) \leq \mu(A) \leq \mu(\overline{A}) \)。而\(\partial A=\overline{A}\setminus A^o \),从而\(\mu(\overline{A})=\mu(A^o)+ \mu(\overline{A} \setminus A^o)=\mu(A^o) + \mu(\partial A)=\mu(A^o) \)。
      故\(\mu(A)=\mu(A^o)=\mu(\overline{A}) \)。又由于\(\mu(A^o) \leq \liminf_{n} \mu_n(A^o) \leq \liminf_{n}\mu_n(A) \leq \limsup_n \mu_n(A) \leq \limsup_{n}\mu_n(\overline{A}) \leq \mu(\overline{A})=\mu(A^o)\),
      从而\(\liminf_{n}\mu(A)=\limsup_{n}\mu(A)=\mu(A^o)=\mu(A) \),即\(\lim_{n}\mu_n(A)=\mu(A) \)。
  \end{enumerate}
\end{proof}
\begin{lemma}\label{lem:1}
  若\(\xi_n \overset{d}{\to} \xi\),则对任何有界连续函数\(g:\mathbb{R} \to \mathbb{R} \),有\(\mathbb{E} g(\xi_n) \to \mathbb{E} g(\xi) \)。
\end{lemma}
\begin{proof}
  设\(\xi_n \)的概率分布为\(\mu_n \),\(\xi \)的概率分布为\(\mu \)。
  \(\forall g  \) 为有界连续函数,那么\(\exists a,b \in \mathbb{R} \),满足\(a < g(x) < b ,\forall x \in \mathbb{R}\)。由于\(\mu \)为概率分布,那么\(\mu(\mathbb{R})=\mathbb{P}(\Omega)=1 \).
  故\(\forall k \in \mathbb{N}_{+} \),集合\( \{c \in \mathbb{R}:\mu(\{x:g(x)=c\})> \frac{1}{k}\}\)为有限集,从而\(\bigcup_{k \in \mathbb{N}_{+}}\{c \in \mathbb{R} : \mu(\{x:g(x) = c\}) >\frac{1}{k}\}=\{c \in \mathbb{R} : \mu(\{x:g(x)=c\}) > 0\}:=T\)至多可数。
  从而\(\forall \varepsilon >0, \exists N \in \mathbb{N}_{+}, \exists a_j \in T^c , 0 \leq j \leq N \)满足:
  \begin{itemize}
    \item \(a=a_0 < \cdots < a_N =b \);
    \item \(a_{j}-a_{j-1} < \varepsilon, 1 \leq j \leq N \);
    \item\(\mu(\{x:g(x)=a_j\})=0, 0 \leq j \leq N\),
  \end{itemize}
  令\(A_j:=\{x:a_{j-1} \leq g(x) < a_j\},j=1,\cdots,N \)。显然\(A_i \cap A_j = \varnothing, i \neq j, 0 \leq i,j \leq N \),\(\bigcup_{i=0}^{N} A_i = \mathbb{R} \)。
  由于\(g \)连续,那么\( \partial A_j = \overline{A_j} \setminus A_j^o \subset \{x:a_{j-1} \leq g(x) \leq a_{j}\} \setminus \{x: a_{j-1} < g(x) < g_j\}= \{x:g(x)=a_{j-1}\} \cup \{x:g(x)=a_j\}\)。
  故\(\mu(\partial A_j)=0,0 \leq j \leq N \)。故由引理 \ref{lem:2}知,\(\lim_{n}\mu_n(A_j)=\mu(A_j),1 \leq j \leq N \)。
  令\(f:=\sum_{j=1}^{N}a_{j-1}\mathbbm{1}_{A_j} \),那么\(|f(x)-g(x)|\leq \max_{1 \leq j \leq N} |a_{j}-a_{j-1}| < \varepsilon ,\forall x \in \mathbb{R}\)。因此,
  \[
    \begin{aligned}
      &\left|\mathbb{E} (g(\xi_n))-\mathbb{E}(g(\xi))\right|\\
      =&\left|\int_{\mathbb{R} } g(x) \d \mu_n(x)-\int_{\mathbb{R}} g(x) \d \mu(x)\right|\\
      \leq& \left|\int_{\mathbb{R}} g(x) \d \mu_n(x)-\int_{\mathbb{R}} f(x) \d \mu_n(x)\right| + \left| \int_{\mathbb{R}} f(x) \d \mu_n(x)-\int_{\mathbb{R}} f(x) \d \mu\right| + \left| \int_{\mathbb{R}} f(x) \d \mu - \int_{\mathbb{R}} g(x) \d \mu\right|\\
      \leq& \int_{\mathbb{R}} \left|g(x)-f(x)\right| \d \mu_n(x) + \left|\sum_{j=1}^{N} a_j(\mu_n(A_j)-\mu(A_j))\right| + \int_{\mathbb{R}}\left|f(x)-g(x)\right|\d \mu \\
      \leq & \varepsilon + \sum_{j=1}^{N} \left|a_j\right|\left|\mu_n(A_j)-\mu(A_j)\right| + \varepsilon
    \end{aligned}
  \]
  从而\(\limsup_{n}|\mathbb{E}(g(\xi_n))-\mathbb{E}(g(\xi))| \leq 2\varepsilon + \limsup_{n} \sum_{j=1}^{N} |a_j||\mu_n(A_j)-\mu(A_j)|=2\varepsilon \)。
  又由于\(\varepsilon \)的任意性,故\(\limsup_{n}|\mathbb{E}(g(\xi_n))-\mathbb{E}(g(\xi))|=0 \),从而\(\lim_{n}\mathbb{E}(g(\xi_n))=\mathbb{E}(g(\xi)) \)。
\end{proof}
\begin{solution}
  考虑\(g_m(x):=\min\{m,|x|\} \) 是有界连续函数,那么由引理 \ref{lem:1}可知, \(\lim_{n}\mathbb{E}(g_m(\xi_n)) =\mathbb{E}(g_m(\xi))\)。
  而\(g_m(\xi) \leq |\xi| \),\(g_m(\xi) \nearrow_m |\xi|\),故由非负单调收敛定理可知,\(\lim_{m}\mathbb{E}(g_m(\xi)) =\mathbb{E}(|\xi|)\)。
  由于\(g_m(\xi_n) \leq |\xi_n| \),故\(\liminf_{n}\mathbb{E}(g_m(\xi_n)) \leq \liminf_{n}\mathbb{E}(|\xi_n|) \),即\(\mathbb{E}(g_m(\xi ))=\lim_{n}\mathbb{E}(g_m(\xi_n))\leq \liminf_{n}\mathbb{E}(|\xi_n|) \)。
  从而\(\mathbb{E}(|\xi |)=\lim_{m}\mathbb{E}(g_m(\xi)) \leq \lim_{m} \liminf_{n}\mathbb{E}(|\xi_n|)=\liminf_{n}\mathbb{E}(|\xi_n|) \)。
\end{solution}

\begin{problem}\label{pro:3.6.13}
  设\(\xi \geq 0 \)使\(\mathbb{E} \xi^2 < \infty\)。证明\(\mathbb{P}(\xi >0) \geq \frac{(\mathbb{E} \xi)^2}{\mathbb{E} \xi^2} \)。
\end{problem}
% \begin{solution}
%   令\(\eta =\mathbbm{1}_{\xi >0} \),由于\(\mathbb{P}(\xi >0) =\mathbb{E}(\eta)=\mathbb{E}(\eta^2) \),故\(\mathbb{E}(\xi^2)\mathbb{P}(\xi >0)=\mathbb{E}(\xi^2)\mathbb{E}(\eta^2) \geq (\mathbb{E}(\xi \eta))^2 \)。
%   由于\(\xi \geq 0 \),那么 \(\mathbb{E}(\xi)=\int_{\Omega} \xi(\omega) \mathbbm{1}_{\xi(\omega) \geq 0}(\omega) \d \mathbb{P} =\int_{\Omega} \xi(\omega) \mathbbm{1}_{\xi (\omega) >0} (\omega) \d \mathbb{P} + \int_{\Omega} \xi(\omega) \mathbbm{1}_{\xi(\omega) =0}(\omega) \d \mathbb{P} =\int_{\Omega} \xi(\omega) \mathbbm{1}_{\xi(\omega) >0} (\omega) \d \mathbb{P} =\mathbb{E}(\xi \eta)\),
%   从而\(\mathbb{E}(\xi^2)\mathbb{P}(\xi >0) \geq (\mathbb{E}(\xi))^2 \)。又由于\(0<\mathbb{E}(\xi^2)<\infty \),那么\(\mathbb{P}(\xi>0) \geq \frac{(\mathbb{E}(\xi))^2}{\mathbb{E}(\xi^2)} \).
% \end{solution}

\begin{problem}\label{pro:3.6.15}
  利用 Jensen 不等式证明几何平均值小于代数平均值:
  \(a_1,\cdots,a_n \geq 0 \)及\(\alpha_1,\cdots,\alpha_n \geq 0 \)使\(\sum_{k=1}^{n}\alpha_k = 1 \),有\(\prod_{k = 1}^{n}a_k^{\alpha_k} \leq \sum_{k = 1}^{n}\alpha_k a_k \)。
\end{problem}
\begin{solution}
  若\(\exists k:1 \leq k \leq n \)使得\(a_k=0 \),那么\(\prod_{k=1}^{n}a_k^{\alpha_k}=0 \leq \sum_{k=1}^{n}\alpha_k a_k \)。故只需考虑\(\forall a_k >0,1 \leq k \leq n \)。
  设\(\Omega=\{1,\cdots,n\},\lambda(k)=\alpha_k \),故\(\lambda \)为\((\Omega,2^{\Omega})\) 上的测度。\(X \)为\((\Omega,2^{\Omega}) \)上的随机变量,满足\(X(k)=a_k \)。
  考虑\(\phi=\ln \),由于\(\phi \)是上凸,故\(\mathbb{E}(\phi(X)) \leq \phi(\mathbb{E}(X)) \)。
  从而\(\mathbb{E}(\phi(X))=\int_{\Omega} \ln(X(k)) \d \lambda(k)=\sum_{k=1}^{n}\ln(a_k)\alpha_k \leq \ln(\int_{\Omega} X(k) \d \lambda(k))=\ln(\sum_{k=1}^{n}a_k \alpha_k) \)。
  故\(\ln(\prod_{k=1}^{n}a_k^{\alpha_k}) \leq \ln(\sum_{k=1}^{n}a_k \alpha_k) \)。又由于\(\ln\)单调递增,那么\(\prod_{k=1}^{n}a_k^{\alpha_k} \leq \sum_{k=1}^{n}\alpha_k a_k \)。
\end{solution}
\begin{problem}\label{pro:3.6.26}
  \begin{enumerate}
    \item \label{ite:26.1} 如\(\{f_t\}_{t \in T} \)一致可积,则必积分一致连续;
    \item 当\(\mu \)有限时,一致可积当且仅当积分一致有界且积分一致连续。
  \end{enumerate}
\end{problem}
\begin{solution}
  \begin{enumerate}
    \item 由于\(\{f_t:t \in T\} \)一致可积,故\(\lim_{n \to \infty} \sup_{t \in T} \mu(|f_t| \mathbbm{1}_{|f_t| \geq n})=0 \).
      故\(\forall \varepsilon >0,\exists N>0 \),\(\forall n \geq N \)有\(\sup_{t \in T}\mu(|f_t|\mathbbm{1}_{|f_t| \geq n})<\frac{\varepsilon}{2} \),即\(\forall t \in T,\mu(|f_t| \mathbbm{1}_{|f_t| \geq n}) < \frac{\varepsilon}{2} \)。
      又对于\(\forall A \subset \Omega :\mu(A)<\frac{\varepsilon}{2N} \)都有\(\forall t \in T \),\(\mu(|f_t|\mathbbm{1}_{A})=\mu(|f_t|\mathbbm{1}_{A \cap \{|f_t| \geq N\}}) + \mu(|f_t|\mathbbm{1}_{A \cap \{|f_t|<N\}})\leq \mu(|f_t|\mathbbm{1}_{\{|f_t|\geq N\}}) + \mu(N \mathbbm{1}_{A}) \leq \frac{\varepsilon}{2} + N \mu(A) < \varepsilon\)。
      从而\(\sup_{t \in T} \mu(|f_t|\mathbbm{1}_A) < \varepsilon \)。因此,\(\lim_{\mu(A) \to 0} \sup_{t \in T}\mu(|f_t|\mathbbm{1}_A)=0 \).
    \item ``\(\implies\)'':由于 \ref{ite:26.1}可知,只需证明积分一致可积能推导出一致有界。\(N \)满足\(\sup_{t \in T} \mu(|f_t|\mathbbm{1}_{\{|f_t| \geq N\}})<1 \).
      可知\(\forall t \in T \),\(\mu(|f_t|)=\mu(|f_t|\mathbbm{1}_{\{|f_t| \geq N\}}) + \mu(|f_t|\mathbbm{1}_{\{|f_t| <N\}})<1 + \mu(N \mathbbm{1}_{\{|f_t| < N\}})<1 + N \mu(\Omega)<\infty \)。
      从而\(\sup_{t \in T} \mu(|f_t|)<1 + N \mu(\Omega)<\infty \)。

      ``\(\impliedby\)'':由于\(\sup_{t \in T}\mu(|f_t|)<\infty \),设\(M:=\sup_{t \in T}\mu(|f_t|) \)。由于积分一致连续,即\\
      \(\lim_{\mu(A) \to 0}\sup_{t \in T}\mu(|f_t|\mathbbm{1}_{A}) =0\).
      那么\(\forall \varepsilon >0 \),\(\exists \delta >0 \),\(\forall A \subset \Omega:\mu(A)<\delta \),都有\(\forall t \in T \),\(\mu(|f_t|\mathbbm{1}_A)<\varepsilon \)。
      又由于\(\forall n>\frac{M}{\delta} \),\(\forall t \in T \),\(\mu(|f_t|\geq n)\leq \frac{\mu(|f_t|)}{n}<\frac{M}{\frac{M}{\delta}}=\delta \),故\(\mu(|f_t|\mathbbm{1}_{\{|f_t| \geq n\}}) < \varepsilon \)。
      从而\(\sup_{t \in T}\mu(|f_t|\mathbbm{1}_{\{|f_t| \geq n\}})<\varepsilon \),故\(\lim_{n \to \infty}\sup_{t \in T}\mu(|f_t|\mathbbm{1}_{\{|f_t| \geq n\}})=0 \).
  \end{enumerate}
\end{solution}
\begin{problem}\label{pro:3.6.30}
  对可测函数\(f \),定义本征上确界为:
  \(\norm{f}_{\infty} := \inf \{M: \mu(\{\omega:|f(\omega) |>M\}) = 0\} \)。
  \begin{enumerate}
    \item 证明\(\norm{\cdot}_{\infty} \)满足三角不等式。
    \item 若\(\mu(\Omega)<\infty \),则\(\norm{f}_{\infty}=\lim_{r \to \infty} \norm{f}_{r} \)。
  \end{enumerate}
\end{problem}
\begin{solution}
  \begin{enumerate}
    \item 先证\(\mu(\{x:|f(x)|> \norm{f}_{\infty}\})=0 \),\(\forall f \)可测都成立。由于\(\norm{f}_{\infty} \)的定义可知\(\exists x_n ,n \in \mathbb{N}_{+} \)满足\(\mu(\{x:|f(x)| > x_n\}) =0\),且\(x_n \searrow_n\norm{f}_{\infty} \)
      那么\(\{x:|f(x)|>x_n\}\nearrow_n\bigcup_{n \in \mathbb{N}_{+ }}\{x:|f(x)|>x_n\}=\{x:|f(x)|>\norm{f}_{\infty}\}\),故\(0=\lim_{n \to \infty}\mu(\{x:|f(x)|>x_n\})=\mu(\lim_{n \to \infty} \{x:|f(x)|> x_n\})=\mu(\{x:|f(x)|>\norm{f}_{\infty}\})\)。

      令\(A:=\{x:|f(x)|>\norm{f}_{\infty}\},B:=\{x:|g(x)|>\norm{g}_{\infty}\} \),那么\(\forall x \in A^c \cap B^c \),有\(|f(x)+ g(x)| \leq |f(x)| + |g(x)| \leq \norm{f}_{\infty} + \norm{g}_{\infty} \),故\(\{x:|f(x) + g(x)|> \norm{f}_{\infty} + \norm{g}_{\infty}\} \subset A \cup B \)。
      故\(\mu(\{x:|f(x)+ g(x)|>\norm{f}_{\infty } + \norm{g}_{\infty}\})\leq \mu(A \cup B)\leq \mu(A) + \mu(B)=0 + 0=0 \)。又由\(\norm{f + g}_{\infty} \)的定义知,\(\norm{f + g}_{\infty} \leq \norm{f}_{\infty} + \norm{g}_{\infty} \)。
    \item 令\(A \)如上定义,那么\(\norm{f}_r^r =\int_{\Omega} |f(x)|^r \d \mu(x) =  \int_{A^c} |f(x)|^r \d \mu(x)+\int_{A} |f(x)|^r \d \mu =\int_{A} |f(x)|^r \d \mu \leq \int_{A} \norm{f}_{\infty}^r \d \mu \leq \mu(\Omega) \norm{f}_{\infty}^r\).
      从而\( \norm{f}_r \leq \norm{f}_{\infty}\mu(\Omega)^{\frac{1}{r}}\)。故\(\limsup_{r} \norm{f}_r \leq \limsup_{r} \norm{f}_{\infty} \mu(\Omega)^{\frac{1}{r}}=\norm{f}_{\infty} \)。

      又由于\(\norm{f}_{\infty} \)的定义可知,\(\forall \varepsilon>0 \),令\(A_{\varepsilon}:\{x:|f(x)|>\norm{f}_{\infty} - \varepsilon \} \),那么\(A_{\varepsilon} >0 \).
      从而\(\norm{f}_{r}^r=\int_{\Omega} |f(x)|^r \d \mu(x)=\int_{A_{\varepsilon}} |f(x)|^r \d \mu + \int_{A_{\varepsilon}^c} |f(x)|^r \d \mu \geq \int_{A_{\varepsilon}} |f(x)|^r \d \mu \geq \int_{A_{\varepsilon}}( \norm{f}_{\infty}-\varepsilon )^r\d \mu =(\norm{f}_{\infty}-\varepsilon)^r \mu(A_{\varepsilon})\)。
      故\(\norm{f}_{r} \geq (\norm{f}_{\infty}-\varepsilon) \mu(A_{\varepsilon})^{\frac{1}{r}} \)。故\(\liminf_{r} \norm{f}_r \geq \liminf_{r}(\norm{f}_{\infty}-\varepsilon) \mu(A_{\varepsilon})^{\frac{1}{r}}=\norm{f}_{\infty}-\varepsilon \)。
      又由\(\varepsilon \)的任意性可知,\(\liminf_{r} \norm{f}_{r} \geq \norm{f}_{\infty} \)。

      综上,\(\lim_{r} \norm{f}_r=\norm{f}_{\infty} \)。
  \end{enumerate}
\end{solution}
\begin{problem}\label{pro:3.6.33}
  设随机变量\(\xi \)具有数学期望\(m \)与方差\(\sigma^2 \)。
  \begin{enumerate}
    \item 证明\(\mathbb{P}(\xi - m \geq t) \leq \frac{\sigma^2}{\sigma^2 + t^2}, \forall t \geq 0 \)。
    \item 证明\(\mathbb{P}(|\xi - m| \geq t) \leq \frac{2\sigma^2}{\sigma^2 + t^2} \)。
  \end{enumerate}
\end{problem}
\begin{lemma}\label{lem:3}
  \((\Omega,\mathcal{A},\mathbb{P}) \) 为概率空间,\(\xi \)为随机变量,\(A \in \mathcal{A} \),\(\mathbb{P}(A)>0 \),那么 \[
    \mathbb{E}(\xi^2 \mid A) \geq (\mathbb{E} (\xi \mid A))^2
  \]
\end{lemma}
\begin{proof}
  若\(\mathbb{P}(A) >0 \),那么\(\mathbb{E}(\xi^2 \mid A) =\frac{1}{\mathbb{P}(A)} \int_{A} \xi^2 \d \mathbb{P} \),
  同理可知\(\mathbb{E}(\xi \mid A) =\frac{1}{\mathbb{P}(A)} \int_{A} \xi \d \mathbb{P}\)。
  由于Cauchy-Schwarz 不等式得\((\mathbb{E}(\xi \mid A))^2 =\frac{1}{\mathbb{P}(A)^2} (\int_{\Omega } \xi \mathbbm{1}_A \d \mathbb{P} )^2 \leq  \frac{1}{\mathbb{P}(A)^2} (\int_{\Omega} \xi^2 \d \mathbb{P} \int_{\Omega} \mathbbm{1}_A^2 \d \mathbb{P} )=\mathbb{E}(\xi^2 \mid A)\)。
\end{proof}
\begin{solution}
  \begin{enumerate}
    \item 令\(A:=\{\xi-m \geq t\} \),\(\mu:=\mathbb{P}(A)\)。若\(\mu=0 \),那么\(0=\mathbb{P}(\xi-m \geq t)\leq \frac{\sigma^2}{\sigma^2 + t^2}\)。
      若\(\mu >0 \),那么
      \(x_1=\mathbb{E}(\xi-m \mid A), x_2=\mathbb{E}(\xi-m \mid A^c)\).又由于\(\mathbb{E}(\xi-m)=0 \),那么\(\mathbb{E}(\xi-m)=\mathbb{E}(\xi-m \mid A)\mathbb{P}(A) + \mathbb{E}(\xi-m \mid A^c) \mathbb{P}(A)=x_1 \mu + x_2 (1-\mu)=0\).
      又由于\(x_1=\mathbb{E}(\xi-m \mid A) \geq \mathbb{E} (t \mid A)=t \),故\(x_1 \geq t \)。
      \[
        \begin{aligned}
          \sigma^2 =\mathbb{E}(|\xi-m|^2)&=\mathbb{E}(|\xi-m|^2 \mid A)\mathbb{P}(A) + \mathbb{E}(|\xi-m|^2 \mid A^c)\mathbb{P}(A)\\
          &\overset{\text{引理} \ref{lem:3}}{\geq} (\mathbb{E}(\xi-m \mid A))^2 \mathbb{P}(A) + (\mathbb{E}(\xi-m \mid A^c))^2 \mathbb{P}(A^c)\\
          &=x_1^2 \mu  + x_2^2(1-\mu)
        \end{aligned}
      \]
      \begin{itemize}
        \item 若\(\mu=1 \),那么\(x_1 \mu + x_2 (1-\mu)=0 \),知\(x_1=0 \),那么\(t \leq 0 \). 又由于\(t \geq 0 \),那么\(t =0 \),此时\(\mathbb{P}(\xi-m \geq t)=\mu=1= \frac{\sigma^2}{\sigma^2} \).
        \item 若\(0 < \mu < 1 \),那么\(\sigma^2 \geq \frac{\mu}{1-\mu}x_1^2 \)。又由于\(x_1=\mathbb{E}(\xi-m \mid A) \geq \mathbb{E} (t \mid A)=t \), 故\(\sigma^2 \geq \frac{\mu}{1-\mu}t^2 \).
          从而\(\mu \leq \frac{\sigma^2}{t^2 + \sigma^2} \),即\(\mathbb{P}(\xi-m \geq t) \leq \frac{\sigma^2}{t^2 + \sigma^2} \).
      \end{itemize}
    \item 令\(\eta=-\xi \),那么\(\mathbb{E}(\eta)=-m,\var(\eta)=\sigma^2 \),那么\(\mathbb{P}(\eta + m \geq t) \leq \frac{\sigma^2}{t^2 + \sigma^2} \),故\(\mathbb{P}(-\xi + m \geq t)=\mathbb{P}(\xi-m \leq -t) \leq \frac{\sigma^2}{\sigma^2 + t^2} \).
      故\(\mathbb{P}(|\xi-m| \geq t) \leq \mathbb{P}(\xi-m \geq t) + \mathbb{P}(\xi-m \leq -t) \leq \frac{2 \sigma^2}{\sigma^2 + t^2} \).
  \end{enumerate}
\end{solution}
\end{document}
