%!Mode:: "TeX:UTF-8"
%!TEX encoding = UTF-8 Unicode
%!TEX TS-program = xelatex
\documentclass{ctexart}
\newif\ifpreface
%\prefacetrue
\input{../../../global/all}
\begin{document}
\large
\setlength{\baselineskip}{1.2em}
\ifpreface
  \input{../../../global/preface}
  \newgeometry{left=2cm,right=2cm,top=2cm,bottom=2cm}
\else
  \newgeometry{left=2cm,right=2cm,top=2cm,bottom=2cm}
  \maketitle
\fi
%from_here_to_type
%知识结构、科研能力、研究成果、外语水平以及所应具备的研究生素质和培养潜能方面

知识结构: 在概率方向选修了概率论,测度与概率,概率极限理论,数理统计,随机过程初步,正在修读马氏过程。
在分析方向修读了数学分析,泛函分析,实变函数,复变函数。
在方程方向修读了常微分方程,偏微分方程。
在代数方向修读了高等代数,伽罗瓦理论,近世代数,群表示论,正在修读数论。
在几何方向修读了解析几何,微分几何,拓扑学。

科研能力和研究成果:
\begin{enumerate}
  \item 2021.05: xxxxxx
  \item 2022.05-2023.05: 参加校本基项目并顺利结项,项目名称是:xxxxx,主要研究的是xxxxxx,主要负责:xxxxxx。
  \item 2023.5--: 参加北创项目目前正在准备结项答辩中,项目名称是:xxxx,主要研究的是xxxxxx,主要负责:xxxxx。
\end{enumerate}

外语水平:
\begin{enumerate}
  \item 3月考了托福,获得92分,其中阅读28,听力26,口语,写作19。(口语写作没准备好,正在准备4月继续考。。)
\end{enumerate}

具备的研究生素养:

\end{document}
