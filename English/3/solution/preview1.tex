%!Mode:: "TeX:UTF-8"
%!TEX encoding = UTF-8 Unicode
%arara: xelatex
\documentclass[12pt]{article}
\usepackage{xeCJK}
\usepackage{geometry}
\usepackage{setspace}
\geometry{a4paper, margin=1in}
\setstretch{1.4}

\begin{document}

\begin{center}
\Large \textbf{高阶学术词汇表(AWL 6–10 扩展版)} \\[0.3em]
\normalsize 用于研究生学术英语翻译与写作训练
\end{center}

\section*{I. 理论与研究设计类词汇}
\begin{itemize}
  \item \textbf{coincide} (v.) — 相符;一致;同时发生  
  \textit{collocations:} coincide with results / coincide in time  
  \textit{example:} The findings coincide with earlier research in the field.  
  \textit{note:} 表示“与……一致”时多用介词 \textit{with}。

  \item \textbf{specify} (v.) — 明确说明;具体规定  
  \textit{collocations:} specify conditions / specify parameters  
  \textit{example:} The author specified the assumptions underlying the model.  
  \textit{note:} 常与名词 \textit{requirement, assumption, parameter} 搭配。

  \item \textbf{allocate} (v.) — 分配;配置(资源、时间)  
  \textit{collocations:} allocate resources / allocate funds efficiently  
  \textit{example:} The project aims to allocate resources more efficiently.  

  \item \textbf{constitute} (v.) — 构成;组成;建立  
  \textit{collocations:} constitute evidence / constitute a paradigm shift  
  \textit{example:} Such findings constitute a major shift in the theory.  

  \item \textbf{inhibit} (v.) — 抑制;阻碍;限制  
  \textit{collocations:} inhibit growth / inhibit learning / inhibit motivation  
  \textit{example:} External rewards may inhibit intrinsic motivation.  
\end{itemize}

\section*{II. 数据与分析类词汇}
\begin{itemize}
  \item \textbf{corroborate} (v.) — 证实;确证(理论、假设)  
  \textit{collocations:} corroborate evidence / corroborate hypothesis  
  \textit{example:} The data corroborate the proposed hypothesis.  

  \item \textbf{coherent} (adj.) — 连贯的;一致的  
  \textit{collocations:} coherent argument / coherent strategy  
  \textit{example:} The report presents a coherent framework for analysis.  

  \item \textbf{invoke} (v.) — 援引(理论、原则);唤起  
  \textit{collocations:} invoke a concept / invoke a law / invoke authority  
  \textit{example:} The author invokes Bourdieu’s theory of cultural capital.  

  \item \textbf{trigger} (v.) — 引发;触发  
  \textit{collocations:} trigger changes / trigger a reaction / trigger debate  
  \textit{example:} The crisis triggered extensive political discussions.  

  \item \textbf{offset} (v.) — 抵消;弥补  
  \textit{collocations:} offset the impact / offset the cost / offset decline  
  \textit{example:} The new policy aims to offset negative externalities.  
\end{itemize}

\section*{III. 跨学科与哲学类词汇}
\begin{itemize}
  \item \textbf{paradigm} (n.) — 范式;理论框架;典范  
  \textit{collocations:} paradigm shift / research paradigm / dominant paradigm  
  \textit{example:} Kuhn described how science evolves through paradigm shifts.  

  \item \textbf{scenario} (n.) — 情景;设想;情境  
  \textit{collocations:} best-case scenario / worst-case scenario / research scenario  
  \textit{example:} Participants were presented with several ethical scenarios.  

  \item \textbf{predominant} (adj.) — 占主导地位的;主要的  
  \textit{collocations:} predominant role / predominant factor / predominant theme  
  \textit{example:} Social context plays a predominant role in identity formation.  

  \item \textbf{subordinate} (v./adj.) — 使从属;下级的;次要的  
  \textit{collocations:} subordinate A to B / subordinate clause  
  \textit{example:} Consciousness cannot be subordinated to biology alone.  

  \item \textbf{convene} (v.) — 召集;召开(会议、研讨)  
  \textit{collocations:} convene a conference / convene experts / convene annually  
  \textit{example:} The committee convened to discuss the proposal.  
\end{itemize}

\section*{IV. 学术写作常用抽象名词}
\begin{itemize}
  \item \textbf{framework} (n.) — 框架;体系  
  \textit{example:} The framework provides a basis for cross-disciplinary analysis.  

  \item \textbf{validity} (n.) — 有效性;合理性  
  \textit{example:} The validity of the model was verified through experiments.  

  \item \textbf{integrity} (n.) — 完整性;学术诚实  
  \textit{example:} Research integrity is essential for scientific credibility.  

  \item \textbf{paradox} (n.) — 悖论;自相矛盾的现象  
  \textit{example:} The paradox of tolerance has been widely debated in philosophy.  

  \item \textbf{implication} (n.) — 含义;潜在影响  
  \textit{example:} This finding has broad implications for educational policy.  
\end{itemize}

\section*{V. 复习建议}
\begin{enumerate}
  \item 优先掌握词汇的\textbf{学术搭配}与\textbf{名词化表达}(如 \textit{the implementation of policies})。  
  \item 建议将每个动词造三个例句,涵盖\textbf{社会科学、语言学、自然科学}不同语境。  
  \item 重点背诵具有抽象逻辑含义的动词(如 \textit{constitute, invoke, offset, inhibit})。  
  \item 复习顺序推荐:意义 → 搭配 → 学术例句 → 自己造句。  
\end{enumerate}

\end{document}
