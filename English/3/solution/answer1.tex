%!Mode:: "TeX:UTF-8"
%!TEX encoding = UTF-8 Unicode
%arara: xelatex
\documentclass[12pt]{article}
\usepackage{xeCJK}
\usepackage{geometry}
\usepackage{setspace}
\geometry{a4paper, margin=1in}
\setstretch{1.4}

\begin{document}

\begin{center}
\Large \textbf{研究生学术英语翻译强化练习卷 —— 答案与词汇讲解版(高阶 AWL)}
\end{center}

\section*{第一套答案与讲解}

\subsection*{英译中}
\begin{enumerate}
  \item 该研究的结果与以往研究\textbf{相吻合},这表明语言习得背后存在普遍的认知机制。  
  \textbf{coincide (v.)} — 一致、相符;常与 with 连用。  
  语法提示:名词化结构“suggesting a mechanism”体现学术语体的简洁。

  \item 任何理论模型都必须清楚地\textbf{明确说明}其所基于的假设。  
  \textbf{specify (v.)} — 明确指出、具体说明。搭配:specify requirements / assumptions。

  \item 本项目旨在更有效地\textbf{分配资源},以增强跨学科合作。  
  \textbf{allocate (v.)} — 分配、配置(资源、资金、时间)。

  \item 技术创新往往\textbf{构成}人类交流方式的范式转变。  
  \textbf{constitute (v.)} — 构成、组成;学术写作中常用于“X constitutes Y”。

  \item 研究表明,外部奖励可能\textbf{抑制}参与者的内在动机。  
  \textbf{inhibit (v.)} — 抑制、阻碍;常用于心理学或生物学语境。
\end{enumerate}

\subsection*{中译英}
\begin{enumerate}
  \item The results of the study \textbf{coincide with} previous experiments, suggesting a universal cognitive mechanism for language acquisition.
  \item Technological innovations often \textbf{constitute a paradigm shift} in human communication.
  \item Researchers must \textbf{specify the assumptions} on which their theoretical models are based.
  \item External rewards may \textbf{inhibit intrinsic motivation}.
  \item The project aims to \textbf{allocate resources efficiently} to promote interdisciplinary collaboration.
\end{enumerate}

---

\section*{第二套答案与讲解}

\subsection*{英译中}
\begin{enumerate}
  \item 研究结果\textbf{证实}了环境因素显著影响认知表现的假设。  
  \textbf{corroborate (v.)} — 证实、确证;常用于实证研究语境。

  \item 委员会强调需要一个更\textbf{连贯一致}的国家数字教育战略。  
  \textbf{coherent (adj.)} — 连贯的、一致的(指政策或论述)。

  \item “文化资本”这一概念被\textbf{援引}来解释教育中的不平等。  
  \textbf{invoke (v.)} — 援引(理论、概念)以解释或论证。

  \item 生物多样性的下降可能\textbf{引发}不可逆的生态后果。  
  \textbf{trigger (v.)} — 引发、导致(尤用于因果关系)。

  \item 该政策旨在通过可持续设计来\textbf{抵消}城市扩张的负面影响。  
  \textbf{offset (v.)} — 抵消、弥补;搭配:offset the impact/effects。
\end{enumerate}

\subsection*{中译英}
\begin{enumerate}
  \item The data further \textbf{corroborate} the significant influence of environmental factors on cognitive performance.  
  \item The government emphasized the necessity of a more \textbf{coherent} national strategy for digital education.  
  \item The concept of “cultural capital” has been \textbf{invoked} to explain persistent inequalities in education.  
  \item The decline in biodiversity could \textbf{trigger} irreversible ecological consequences.  
  \item The policy aims to \textbf{offset} the negative effects of urban expansion through sustainable design.
\end{enumerate}

---

\section*{第三套答案与讲解}

\subsection*{英译中}
\begin{enumerate}
  \item 该框架为跨学科话语分析提供了一个\textbf{综合性范式}。  
  \textbf{paradigm (n.)} — 范式、理论框架;常用于哲学与科学研究。

  \item 参与者被要求评估不同的\textbf{情境},这些情境涉及伦理决策。  
  \textbf{scenario (n.)} — 情境、设想的情景(尤其指假设性或研究设计中的场景)。

  \item 报告强调了社会语境在塑造个体身份中的\textbf{主导作用}。  
  \textbf{predominant (adj.)} — 占主导地位的;搭配:predominant factor / role。

  \item 学者们争论意识是否能被\textbf{归属于}纯粹的生物学解释。  
  \textbf{subordinate (v.)} — 使从属、使次要;哲学语境常用于抽象关系。

  \item 即将召开的会议将\textbf{召集}人工智能领域的顶尖专家。  
  \textbf{convene (v.)} — 召集(会议、学术讨论);正式书面语。
\end{enumerate}

\subsection*{中译英}
\begin{enumerate}
  \item The report highlights the \textbf{predominant role} of social context in shaping individual identity.  
  \item The proposed framework provides a \textbf{comprehensive paradigm} for cross-disciplinary discourse analysis.  
  \item Scholars have debated whether consciousness can be \textbf{subordinated to} purely biological explanations.  
  \item The forthcoming conference will \textbf{convene leading experts} in artificial intelligence.  
  \item Participants were asked to assess various \textbf{scenarios} involving ethical decision-making.
\end{enumerate}

---

\section*{总结与复习建议}
\begin{itemize}
  \item 注意高阶学术词汇多为\textbf{抽象名词或动词},掌握其典型搭配比单记含义更重要。  
  \item 学术翻译中常用\textbf{名词化结构}(如 “the implementation of…” 代替 “to implement…”)。  
  \item 翻译时应保持\textbf{语体正式、逻辑连贯、主谓关系清晰},避免过度直译。  
  \item 可将本卷词汇制作成“词根+搭配”卡片,辅助学术写作与翻译。
\end{itemize}

\end{document}
