%!Mode:: "TeX:UTF-8"
%!TEX encoding = UTF-8 Unicode
%arara: xelatex
\documentclass[12pt]{article}
\usepackage[a4paper, margin=1in]{geometry}
\usepackage{xeCJK}
\usepackage{setspace}
\begin{document}

\title{Title Translation Exercises}
\date{}
\maketitle

\section*{Part I: 英译汉}

1. Assessing the prevalence of antibiotic resistance in community health settings: a longitudinal analysis

2. Challenges in implementing data-driven interventions for adolescent mental health across diverse regions

3. Factors associated with vaccine hesitancy among rural populations: evidence from a mixed-method study

4. The impact of online learning environments on students’ academic engagement during the pandemic

5. Exploring gender disparities in STEM career aspirations through cross-cultural comparison


\section*{Part II: 汉译英}

6. 数字医疗技术在慢性病管理中的应用与挑战

7. 高等教育国际化背景下教师专业发展策略研究

8. 基于大数据的城市空气质量动态监测:方法与实证

9. 老龄社会中长期护理服务体系建设的政策建议

10. 人工智能在促进可持续农业中的作用与前景


\section*{Part I: 英译汉}
\textbf{1.} 评估社区卫生环境中抗生素耐药性的流行率:一项纵向分析 \\
\textit{解析:} prevalence=流行率;in community health settings 表地点;longitudinal analysis=纵向分析。

\textbf{2.} 在不同地区开展数据驱动的青少年心理健康干预所面临的挑战 \\
\textit{解析:} implementing interventions=实施干预;across diverse regions=跨不同地区。

\textbf{3.} 农村人群疫苗犹豫相关因素:来自一项混合方法研究的证据 \\
\textit{解析:} associated with=与……相关;mixed-method study=混合方法研究。

\textbf{4.} 疫情期间在线学习环境对学生学术参与度的影响 \\
\textit{解析:} impact on=对……的影响;academic engagement=学习投入。

\textbf{5.} 通过跨文化比较探讨 STEM 职业志向中的性别差异 \\
\textit{解析:} through cross-cultural comparison=通过跨文化比较;STEM 保持大写不翻译。


\section*{Part II: 汉译英}

\textbf{6.} Applications and challenges of digital health technologies in chronic disease management \\
\textit{解析:} in chronic disease management:介词 in 表领域;applications and challenges 并列结构自然。

\textbf{7.} A study on strategies for teacher professional development in the context of higher education internationalization \\
\textit{解析:} in the context of=在……背景下;professional development=标准术语。

\textbf{8.} Dynamic monitoring of urban air quality based on big data: methods and empirical evidence \\
\textit{解析:} based on=基于;empirical evidence=实证证据。

\textbf{9.} Policy recommendations for developing long-term care service systems in an aging society \\
\textit{解析:} in an aging society:介词 in 表范围;recommendations for=关于……的建议。

\textbf{10.} The role and prospects of artificial intelligence in promoting sustainable agriculture \\
\textit{解析:} in promoting=在……中发挥作用;prospects 学术性强于 future。


\section*{英译汉}

1. Integrating climate risk assessment into urban infrastructure planning: a comparative policy analysis

2. The effectiveness of mobile health platforms in improving medication adherence among diabetic patients

3. Governance challenges in managing cross-border data flows in the digital economy

4. Social determinants of mental well-being among migrant workers: insights from a multi-country survey

5. Machine learning approaches for predicting antimicrobial resistance patterns in clinical settings


\section*{汉译英}

6. 高校在线课程质量保障体系的国际比较研究

7. 面向老年人的智能穿戴设备使用障碍及其影响因素分析

8. 人工智能赋能公共交通系统优化的路径研究

9. 乡村振兴背景下农业绿色转型的动力机制

10. 基于跨学科视角的全球粮食安全风险评估

\section*{英译汉(含解析)}

\textbf{1.} 将气候风险评估纳入城市基础设施规划:一项比较政策分析  
\textit{解析:} integrating ... into=将……纳入;urban infrastructure=城市基础设施;comparative policy analysis=比较政策分析。

\textbf{2.} 移动健康平台在提高糖尿病患者用药依从性方面的有效性  
\textit{解析:} medication adherence=用药依从性;effectiveness=有效性。

\textbf{3.} 管理数字经济中的跨境数据流动的治理挑战  
\textit{解析:} cross-border data flows=跨境数据流动;governance challenges=治理挑战。

\textbf{4.} 移民工人心理健康的社会决定因素:来自多国调查的见解  
\textit{解析:} social determinants=社会决定因素;mental well-being=心理健康;multi-country survey=多国调查。

\textbf{5.} 预测临床环境中抗微生物耐药模式的机器学习方法  
\textit{解析:} antimicrobial resistance=抗微生物耐药;patterns=模式。


\section*{汉译英(含解析)}

\textbf{6.} An international comparative study on quality assurance systems for online courses in higher education  
\textit{解析:} quality assurance system=质量保障体系;international comparative study=国际比较研究。

\textbf{7.} Analysis of usage barriers and influencing factors of smart wearable devices for older adults  
\textit{解析:} usage barriers=使用障碍;influencing factors=影响因素。

\textbf{8.} Strategies for optimizing public transportation systems empowered by artificial intelligence  
\textit{解析:} empowered by=由……赋能;optimizing public transportation=优化公共交通。

\textbf{9.} Driving mechanisms of agricultural green transformation under the background of rural revitalization  
\textit{解析:} green transformation=绿色转型;driving mechanisms=动力机制。

\textbf{10.} Global food security risk assessment from an interdisciplinary perspective  
\textit{解析:} interdisciplinary perspective=跨学科视角;risk assessment=风险评估。


\section*{英译汉}

1. Digital governance strategies for strengthening cybersecurity resilience  
2. Socioeconomic impacts of telemedicine adoption in remote communities  
3. Modeling renewable energy storage efficiency under climate variability  
4. The role of parental engagement in early childhood literacy development  
5. Ethical challenges in deploying autonomous vehicles in urban environments  
6. Evaluating public attitudes toward gene-edited crops: a survey analysis  
7. Cross-cultural differences in workplace motivation and productivity  
8. Assessing the long-term environmental costs of plastic waste management  
9. The effectiveness of peer-supported learning in higher education  
10. Artificial intelligence in predicting global migration patterns

\section*{汉译英}

11. 大城市住房可负担性的空间分布研究  
12. 气候变化背景下水资源调配的风险管理  
13. 低碳交通体系构建的政策路径研究  
14. 智慧城市建设中的隐私保护问题分析  
15. 人口老龄化对劳动市场结构的影响  
16. 社会媒体使用对青少年身份认同的影响  
17. 公共卫生突发事件中的社区动员策略  
18. 数字鸿沟与教育公平的多维度分析  
19. 海洋生态系统退化的驱动因素研究  
20. 城市绿地配置优化的多目标模型研究


\textbf{1. Digital governance strategies for strengthening cybersecurity resilience} \\
数字治理策略以增强网络安全韧性。 \\
\textit{解析:}digital governance=数字治理;resilience=“韧性”,网络安全语境中常译为“安全韧性”。

\bigskip

\textbf{2. Socioeconomic impacts of telemedicine adoption in remote communities} \\
远程社区采用远程医疗的社会经济影响。 \\
\textit{解析:}telemedicine=远程医疗;remote communities=偏远社区。

\bigskip

\textbf{3. Modeling renewable energy storage efficiency under climate variability} \\
气候变率条件下可再生能源储能效率的建模研究。 \\
\textit{解析:}under climate variability 表示“在气候变率之下”;modeling=建模。

\bigskip

\textbf{4. The role of parental engagement in early childhood literacy development} \\
父母参与在幼儿读写能力发展中的作用。 \\
\textit{解析:}literacy=读写能力;parental engagement=父母参与。

\bigskip

\textbf{5. Ethical challenges in deploying autonomous vehicles in urban environments} \\
在城市环境中部署自动驾驶汽车所面临的伦理挑战。 \\
\textit{解析:}deploy=部署;autonomous vehicles=自动驾驶汽车。

\bigskip

\textbf{6. Evaluating public attitudes toward gene-edited crops: a survey analysis} \\
基因编辑作物的公众态度评估:一项调查分析。 \\
\textit{解析:}gene-edited crops=基因编辑作物;survey analysis=调查分析。

\bigskip

\textbf{7. Cross-cultural differences in workplace motivation and productivity} \\
工作场所动机与生产力的跨文化差异。 \\
\textit{解析:}cross-cultural differences=跨文化差异。

\bigskip

\textbf{8. Assessing the long-term environmental costs of plastic waste management} \\
塑料废弃物管理的长期环境成本评估。 \\
\textit{解析:}environmental costs=环境成本;long-term=长期的。

\bigskip

\textbf{9. The effectiveness of peer-supported learning in higher education} \\
高校同伴支持学习的有效性。 \\
\textit{解析:}peer-supported learning=同伴支持学习。

\bigskip

\textbf{10. Artificial intelligence in predicting global migration patterns} \\
人工智能在预测全球迁移模式中的应用。 \\
\textit{解析:}migration patterns=迁移模式;predicting=预测。

\bigskip

\section*{汉译英(答案与解析)}

\textbf{11. 大城市住房可负担性的空间分布研究} \\
A study on the spatial distribution of housing affordability in large cities. \\
\textit{解析:}housing affordability=住房可负担性;spatial distribution=空间分布。

\bigskip

\textbf{12. 气候变化背景下水资源调配的风险管理} \\
Risk management of water resource allocation under climate change. \\
\textit{解析:}allocation=调配、分配;under climate change=在气候变化背景下。

\bigskip

\textbf{13. 低碳交通体系构建的政策路径研究} \\
A policy pathway study for developing a low-carbon transportation system. \\
\textit{解析:}policy pathway=政策路径;low-carbon transportation system=低碳交通体系。

\bigskip

\textbf{14. 智慧城市建设中的隐私保护问题分析} \\
An analysis of privacy protection issues in smart city development. \\
\textit{解析:}privacy protection=隐私保护;issues=问题。

\bigskip

\textbf{15. 人口老龄化对劳动市场结构的影响} \\
The impact of population aging on the structure of the labor market. \\
\textit{解析:}population aging=人口老龄化。

\bigskip

\textbf{16. 社会媒体使用对青少年身份认同的影响} \\
The influence of social media use on adolescents' identity formation. \\
\textit{解析:}identity formation=身份认同形成;adolescents=青少年。

\bigskip

\textbf{17. 公共卫生突发事件中的社区动员策略} \\
Community mobilization strategies in public health emergencies. \\
\textit{解析:}community mobilization=社区动员;emergencies=突发事件。

\bigskip

\textbf{18. 数字鸿沟与教育公平的多维度分析} \\
A multidimensional analysis of the digital divide and educational equity. \\
\textit{解析:}digital divide=数字鸿沟;educational equity=教育公平。

\bigskip

\textbf{19. 海洋生态系统退化的驱动因素研究} \\
A study on the driving factors of marine ecosystem degradation. \\
\textit{解析:}driving factors=驱动因素;degradation=退化。

\bigskip

\textbf{20. 城市绿地配置优化的多目标模型研究} \\
A multi-objective modeling study on the optimization of urban green space allocation. \\
\textit{解析:}multi-objective=多目标;allocation=配置。
\section*{英译汉}

1. Evaluating the sustainability of circular supply chains in global markets  
2. The influence of bilingual education on cognitive flexibility in children  
3. Data-driven approaches to forecasting infectious disease outbreaks  
4. The economic implications of digital taxation in emerging economies  
5. Behavioral determinants of urban recycling participation  
6. Trust-building mechanisms in online collaborative learning platforms  
7. Economic forecasting using hybrid neural network models  
8. Gender disparities in access to STEM higher education  
9. The role of public–private partnerships in disaster risk reduction  
10. Long-term socioeconomic outcomes of early childhood intervention programs  
11. Urban heat island mitigation strategies based on green infrastructure  
12. Big-data-supported evaluation of traffic congestion policies  
13. Psychological resilience among healthcare workers during pandemic crises  
14. Assessing renewable energy investment risks in developing countries  
15. Multilingual sentiment analysis using transformer-based models

\section*{汉译英}

16. 跨境电商发展对传统贸易模式的重塑  
17. 数字金融普惠性的区域差异及成因分析  
18. 高校科研绩效评价体系的优化策略  
19. 智慧医疗体系建设中的数据互操作性研究  
20. 城市韧性提升中的社区参与机制  
21. 环境规制对制造业绿色创新的影响  
22. 碳中和目标下产业结构调整的路径研究  
23. 粮食安全治理体系的全球比较研究  
24. 高铁网络对区域经济一体化的促进作用  
25. 社区老年服务设施供给的空间公平性  
26. 海外华人文化认同的跨代际传递研究  
27. 人工智能推动文化创意产业发展的模式研究  
28. 新媒体平台上的健康传播行为分析  
29. 农村数字基础设施建设的绩效评价  
30. 城市空气污染治理的多源数据融合方法


\section*{英译汉(答案与解析)}

\textbf{1. Evaluating the sustainability of circular supply chains in global markets} \\
评估全球市场中循环供应链的可持续性。 \\
\textit{解析:}circular supply chains=循环供应链;sustainability=可持续性。

\bigskip

\textbf{2. The influence of bilingual education on cognitive flexibility in children} \\
双语教育对儿童认知灵活性的影响。 \\
\textit{解析:}cognitive flexibility=认知灵活性。

\bigskip

\textbf{3. Data-driven approaches to forecasting infectious disease outbreaks} \\
用于预测传染病暴发的数据驱动方法。 \\
\textit{解析:}data-driven=数据驱动;outbreak=暴发。

\bigskip

\textbf{4. The economic implications of digital taxation in emerging economies} \\
数字税在新兴经济体中的经济影响。 \\
\textit{解析:}economic implications=经济影响;digital taxation=数字税。

\bigskip

\textbf{5. Behavioral determinants of urban recycling participation} \\
城市居民参与垃圾回收的行为决定因素。 \\
\textit{解析:}behavioral determinants=行为决定因素。

\bigskip

\textbf{6. Trust-building mechanisms in online collaborative learning platforms} \\
在线协作学习平台中的信任构建机制。 \\
\textit{解析:}trust-building mechanisms=信任构建机制。

\bigskip

\textbf{7. Economic forecasting using hybrid neural network models} \\
使用混合神经网络模型进行经济预测。 \\
\textit{解析:}hybrid neural network models=混合神经网络模型。

\bigskip

\textbf{8. Gender disparities in access to STEM higher education} \\
在获得 STEM 高等教育方面的性别差异。 \\
\textit{解析:}gender disparities=性别差异;access to=获得(教育/资源)。

\bigskip

\textbf{9. The role of public–private partnerships in disaster risk reduction} \\
公私合作伙伴关系在减轻灾害风险中的作用。 \\
\textit{解析:}public–private partnerships=PPP 公私合作模式。

\bigskip

\textbf{10. Long-term socioeconomic outcomes of early childhood intervention programs} \\
幼儿干预项目的长期社会经济效果。 \\
\textit{解析:}intervention programs=干预项目;outcomes=结果/效果。

\bigskip

\textbf{11. Urban heat island mitigation strategies based on green infrastructure} \\
基于绿色基础设施的城市热岛效应缓解策略。 \\
\textit{解析:}urban heat island=城市热岛;mitigation=缓解。

\bigskip

\textbf{12. Big-data-supported evaluation of traffic congestion policies} \\
基于大数据的交通拥堵治理政策评估。 \\
\textit{解析:}big-data-supported=大数据支持的。

\bigskip

\textbf{13. Psychological resilience among healthcare workers during pandemic crises} \\
疫情危机期间医护人员的心理韧性。 \\
\textit{解析:}psychological resilience=心理韧性。

\bigskip

\textbf{14. Assessing renewable energy investment risks in developing countries} \\
发展中国家可再生能源投资风险评估。 \\
\textit{解析:}investment risks=投资风险。

\bigskip

\textbf{15. Multilingual sentiment analysis using transformer-based models} \\
使用基于 Transformer 的模型进行多语种情感分析。 \\
\textit{解析:}sentiment analysis=情感分析;transformer-based=基于 Transformer。

\bigskip

\section*{汉译英(答案与解析)}

\textbf{16. 跨境电商发展对传统贸易模式的重塑} \\
The reshaping of traditional trade patterns by the development of cross-border e-commerce. \\
\textit{解析:}reshaping=重塑;cross-border e-commerce=跨境电商。

\bigskip

\textbf{17. 数字金融普惠性的区域差异及成因分析} \\
An analysis of regional disparities and underlying causes in the inclusiveness of digital finance. \\
\textit{解析:}inclusiveness=普惠性;disparities=差异。

\bigskip

\textbf{18. 高校科研绩效评价体系的优化策略} \\
Optimization strategies for performance evaluation systems of scientific research in universities. \\
\textit{解析:}performance evaluation system=绩效评价体系。

\bigskip

\textbf{19. 智慧医疗体系建设中的数据互操作性研究} \\
A study on data interoperability in the construction of smart healthcare systems. \\
\textit{解析:}interoperability=互操作性。

\bigskip

\textbf{20. 城市韧性提升中的社区参与机制} \\
Community participation mechanisms for enhancing urban resilience. \\
\textit{解析:}urban resilience=城市韧性。

\bigskip

\textbf{21. 环境规制对制造业绿色创新的影响} \\
The impact of environmental regulation on green innovation in the manufacturing industry. \\
\textit{解析:}environmental regulation=环境规制。

\bigskip

\textbf{22. 碳中和目标下产业结构调整的路径研究} \\
A study on pathways for industrial structure adjustment under the carbon neutrality goal. \\
\textit{解析:}carbon neutrality=碳中和。

\bigskip

\textbf{23. 粮食安全治理体系的全球比较研究} \\
A global comparative study of food security governance systems. \\
\textit{解析:}food security governance=粮食安全治理。

\bigskip

\textbf{24. 高铁网络对区域经济一体化的促进作用} \\
The promoting role of high-speed rail networks in regional economic integration. \\
\textit{解析:}regional economic integration=区域经济一体化。

\bigskip

\textbf{25. 社区老年服务设施供给的空间公平性} \\
Spatial equity in the provision of elderly service facilities in communities. \\
\textit{解析:}spatial equity=空间公平性;provision=供给。

\bigskip

\textbf{26. 海外华人文化认同的跨代际传递研究} \\
A study on intergenerational transmission of cultural identity among overseas Chinese. \\
\textit{解析:}intergenerational transmission=跨代际传递。

\bigskip

\textbf{27. 人工智能推动文化创意产业发展的模式研究} \\
A study on models of AI-driven development in the cultural and creative industries. \\
\textit{解析:}AI-driven=人工智能推动。

\bigskip

\textbf{28. 新媒体平台上的健康传播行为分析} \\
An analysis of health communication behaviors on new media platforms. \\
\textit{解析:}health communication=健康传播。

\bigskip

\textbf{29. 农村数字基础设施建设的绩效评价} \\
Performance evaluation of digital infrastructure construction in rural areas. \\
\textit{解析:}digital infrastructure=数字基础设施。

\bigskip

\textbf{30. 城市空气污染治理的多源数据融合方法} \\
Multi-source data fusion methods for urban air pollution control. \\
\textit{解析:}multi-source data fusion=多源数据融合。
\section*{一、英译中(Yield)}

\begin{enumerate}[1.]

\item The farmer’s land yields more corn after adopting the new irrigation system. \\
\textbf{翻译:} 这位农民在采用新的灌溉系统后,土地产出了更多的玉米。\\
\textbf{解析:} yield = produce / bring forth(“产出、产生”)。

\item Drivers must yield to pedestrians at the crosswalk. \\
\textbf{翻译:} 司机必须在人行横道处礼让行人。\\
\textbf{解析:} yield = give way(“礼让、让行”),交通场景高频考义。

\item The research experiment yielded unexpected results. \\
\textbf{翻译:} 该研究实验产生了意想不到的结果。\\
\textbf{解析:} yield = produce results(产生结果),学术文常见。

\item Under pressure, he finally yielded to the committee’s decision. \\
\textbf{翻译:} 在压力下,他最终屈服于委员会的决定。\\
\textbf{解析:} yield to = give in to(“屈服、让步”)。

\item The investment is expected to yield a 7\% annual return. \\
\textbf{翻译:} 这项投资预计每年可带来 7\% 的回报。\\
\textbf{解析:} yield = bring financial returns(“带来收益”)。

\end{enumerate}

\section*{二、中译英(Yield)}

\begin{enumerate}[6.]

\item 请在狭窄路段对会车的车辆礼让。 \\
\textbf{翻译:} Please yield to oncoming vehicles on narrow roads.\\
\textbf{解析:} 交通用语,yield to = 让行。

\item 新技术已经开始为公司带来更高的收益。 \\
\textbf{翻译:} The new technology has begun to yield higher profits for the company.\\
\textbf{解析:} yield = generate profits。

\item 实验最终没有产出任何可靠的数据。 \\
\textbf{翻译:} The experiment eventually did not yield any reliable data.\\
\textbf{解析:} 科研语境常见搭配:yield data/results。

\item 在巨大的压力下,他仍然拒绝屈服。 \\
\textbf{翻译:} He refused to yield even under enormous pressure.\\
\textbf{解析:} yield = give in(屈服)。

\item 这种作物每年的产量都很高。 \\
\textbf{翻译:} This crop yields a high output every year.\\
\textbf{解析:} yield(动词)= produce(产量)。

\end{enumerate}

\end{document}
