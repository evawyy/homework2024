%!Mode:: "TeX:UTF-8"
%!TEX encoding = UTF-8 Unicode
%arara: xelatex
\documentclass[12pt]{article}
\usepackage[a4paper, margin=1in]{geometry}
\usepackage{xeCJK}
\usepackage{setspace}
\usepackage{color}
% \newcommand{\prep}[1]{\textcolor{red}{#1}} % 高亮介词
\begin{document}

\title{Title Translation Exercises}
\date{}
\maketitle
% 介词的使用
% 新冠肺炎流行后医疗卫生专家培训面临的挑战和机遇
% Challenges and opportunities for educating health professionals after the Covid-19 pandemic
% 对高校科研组织改革与创新的几点建议
% Suggestions for Reform and Innovation of Scientific Research Organization in Higher Learning Institutions
% 应用自动解释算法来评估胸部X射线在监狱中大规模结核病筛查:横断面研究
% Evaluation of chest X-ray with automated interpretation algorithms for mass tuberculosis screening in prisons: A cross-sectional study
% 使用常规临床和认知测量对多中心代表性不足的样本进行阿尔茨海默病和额颞叶痴呆的分类:横断面观察性研究
% Classification of Alzheimer's disease and frontotemporal dementia using routine clinical and cognitive measures across multicentric underrepresented samples: A cross sectional observational study
% 在线实践社区在促进科学教师社会技术资本方面所发挥的作用
% The Role of Online Communities of Practice in Promoting Sociotechnical Capital among Science Teachers
% 中朝双边经贸关系现状与前景
% Status and Outlook of Sino-DPRK Bilateral Trade and Economic Relation
% 中美贸易互补性的实证研究
% Empirical Analysis on the Trade Complementarity of Sino-U.S. Trade
% 日韩两国建立碳足迹标签制度的经验及启示
% Mutual Experiences and Revelations of Establishing Carbon Footprint Labeling in Japan and Korea (revelation: an act of revealing to view or making known)
% 赞比亚艾滋病毒流行普遍检测和治疗的预测结果
% Projected outcomes of universal testing and treatment in an HIV epidemicin Zambia
% 东亚经济一体化和跨太平洋伙伴关系协议一中日之间的博弈
% The Economic Integration in East Asia and TPP: A Game Between China and Japan
% 为可持续发展创造临界点的政治智慧
% The political wisdom of enabling tipping points for sustainable development
% 文化遗产保护中的人工智能伦理:跨文化视角
% Al Ethics in Cultural Heritage Preservation: A Cross-Cultural Perspective
% 循环经济与生态系统韧性:可持续发展的跨学科分析
% Circular Economy and Ecosystem Resilience: An Interdisciplinary Analysis of Sustainable Development
% 新冠疫情背景下可再生能源智能家居的能源管理方案分析
% Analysis of energy management schemes for renewable-energy-based smart homes against the backdrop of COVID-19
% 理工科教育中的混合式学习:教育技术与认知科学的融合应用
% Blended Learning in STEM Education: Integrating Educational Technology and Cognitive Science
% 数字时代的数据隐私监管:法律、技术与伦理的跨学科研究
% Data Privacy Regulation in the Digital Age: A Cross-Disciplinary Study of Law, Technology and Ethics
% 代表可持续基础设施发展的多学科决策基本原理的本体论
% Ontology representing multidisciplinary decision-making rationales for sustainable infrastructure developments
% 西方发达国家残疾人社会保障的成功经验对我国的启示
% The Enlightenment of Experiences regarding Social Security for the Disabled in Western Developed Countries to China
% 通过立法保障妇女平等权-基于国内和国际比较的视角
% Legislative Protection for Gender Equality: An International and Domestic Comparative
% 争议与纠结:经济法中的法律责任研究述评
% Controversy and Dilemma: Legal Liability in Economic Law
% 为何要技术革新?如何进行革新?
% The Why and How of Technological Innovation
% 新出台标准可能加快抗癌药物研发步伐
% New Yardstick Could Speed Access to Cancer Drugs
% 考古学视角下的日本简史
% Archaeology: A Potted History of Japan
% 用什么准则来指导糖尿病的诊断、治疗和预防?
% Guldelines for Dlagnosis, Treatment, and Prevention of Diabetes
% 美国医院中关于早产儿养护的咨询实践所发生的变化
% Variations among US hospitals in counseling practices regarding prematurely born infants
% 通过回顾预测到近十年来全球变暖节奏变慢
% Retrospective prediction of the global warming slowdown in the past decade
% 未来气候变化背景下来自陆地生物圈中有关温室气体的多重反馈
% Multiple Greenhouse-gas Feedbacks from the Land Biosphere under Future Climate Change Scenarios
% 在最差情况及最佳情况下报童问题的非分布研究
% The Distribution-free Newsboy Problem under the Worst-case and Best-case Scenarios
% 叙事认知:文学研究、神经科学与语言学的交互探索
% Narrative Cognition: Interactions Between Literary Studies, Neuroscience and Linguistics
% 巴西发现入侵性的蓑鲉
% Invasive Lionfish Discovered in Brazil
% 3D模拟黑洞碰撞被誉为迄今为止最真实的一次
% 3D Simulations of Colliding Black Holes Hailed as Most Realistic Yet (hail: greet with enthusiastic approval;acclaim赞扬;称颂)
% 科学家们对安乐死所涉及的道德问题产生意见分歧
% Ethics of Euthanasia Divides Scientists


\section*{医学与健康 Medical \& Health}
\begin{enumerate}

\item 新冠肺炎流行后医疗卫生专家培训面临的挑战和机遇\\
\textbf{Challenges and opportunities \textcolor{red}{for} educating health professionals \textcolor{red}{after} the Covid-19 pandemic}\\
\textit{点评:\textcolor{red}{for} 表对象;\textcolor{red}{after} 表时间背景。}

\item 应用自动解释算法来评估胸部X射线在监狱中大规模结核病筛查:横断面研究\\
\textbf{Evaluation of chest X-ray with automated interpretation algorithms \textcolor{red}{for} mass tuberculosis screening \textcolor{red}{in} prisons: A cross-sectional study}\\
\textit{点评:\textcolor{red}{for} 表用途;\textcolor{red}{in} 表地点。}

\item 使用常规临床和认知测量对多中心代表性不足的样本进行阿尔茨海默病和额颞叶痴呆的分类:横断面观察性研究\\
\textbf{Classification of Alzheimer's disease and frontotemporal dementia using routine clinical and cognitive measures \textcolor{red}{across} multicentric underrepresented samples: A cross-sectional observational study}\\
\textit{点评:\textcolor{red}{across} 表范围跨越。}

\item 赞比亚艾滋病毒流行普遍检测和治疗的预测结果\\
\textbf{Projected outcomes \textcolor{red}{of} universal testing and treatment \textcolor{red}{in} an HIV epidemic \textcolor{red}{in} Zambia}\\
\textit{点评:\textcolor{red}{of} 所属;\textcolor{red}{in} 表发生地点。}

\item 用什么准则来指导糖尿病的诊断、治疗和预防?\\
\textbf{Guidelines \textcolor{red}{for} Diagnosis, Treatment, and Prevention \textcolor{red}{of} Diabetes}\\
\textit{点评:\textcolor{red}{for} 表用途;\textcolor{red}{of} 所属对象。}

\item 美国医院中关于早产儿养护的咨询实践所发生的变化\\
\textbf{Variations \textcolor{red}{among} US hospitals \textcolor{red}{in} counseling practices \textcolor{red}{regarding} prematurely born infants}\\
\textit{点评:\textcolor{red}{among} 范围;\textcolor{red}{in} 范畴;\textcolor{red}{regarding} 关于。}

\item 通过回顾预测到近十年来全球变暖节奏变慢\\
\textbf{Retrospective prediction \textcolor{red}{of} the global warming slowdown \textcolor{red}{in} the past decade}\\
\textit{点评:\textcolor{red}{of} 所属;\textcolor{red}{in} 时间范围。}

\item 未来气候变化背景下来自陆地生物圈中有关温室气体的多重反馈\\
\textbf{Multiple Greenhouse-gas Feedbacks \textcolor{red}{from} the Land Biosphere \textcolor{red}{under} Future Climate Change Scenarios}\\
\textit{点评:\textcolor{red}{from} 来源;\textcolor{red}{under} 背景条件。}

\end{enumerate}



\section*{教育与社会科学 Education \& Social Science}
\begin{enumerate}

\item 对高校科研组织改革与创新的几点建议\\
\textbf{Suggestions \textcolor{red}{for} Reform and Innovation \textcolor{red}{of} Scientific Research Organization \textcolor{red}{in} Higher Learning Institutions}\\
\textit{点评:\textcolor{red}{for} 对象;\textcolor{red}{of} 所属;\textcolor{red}{in} 范围。}

\item 在线实践社区在促进科学教师社会技术资本方面所发挥的作用\\
\textbf{The Role \textcolor{red}{of} Online Communities of Practice \textcolor{red}{in} Promoting Sociotechnical Capital \textcolor{red}{among} Science Teachers}\\
\textit{点评:\textcolor{red}{of} 所属;\textcolor{red}{in} 范畴;\textcolor{red}{among} 群体范围。}

\item 理工科教育中的混合式学习:教育技术与认知科学的融合应用\\
\textbf{Blended Learning \textcolor{red}{in} STEM Education: Integrating Educational Technology and Cognitive Science}\\
\textit{点评:\textcolor{red}{in} 表范围。}

\item 文化遗产保护中的人工智能伦理:跨文化视角\\
\textbf{AI Ethics \textcolor{red}{in} Cultural Heritage Preservation: A Cross-Cultural Perspective}\\
\textit{点评:\textcolor{red}{in} 表适用范围。}

\item 叙事认知:文学研究、神经科学与语言学的交互探索\\
\textbf{Narrative Cognition: Interactions \textcolor{red}{between} Literary Studies, Neuroscience and Linguistics}\\
\textit{点评:\textcolor{red}{between} 表多个领域间的交互。}

\item 争议与纠结:经济法中的法律责任研究述评\\
\textbf{Controversy and Dilemma: Legal Liability \textcolor{red}{in} Economic Law}\\
\textit{点评:\textcolor{red}{in} 表范围领域。}

\item 为何要技术革新?如何进行革新?\\
\textbf{The Why and How \textcolor{red}{of} Technological Innovation}\\
\textit{点评:\textcolor{red}{of} 所属。}

\item 通过立法保障妇女平等权——基于国内和国际比较的视角\\
\textbf{Legislative Protection \textcolor{red}{for} Gender Equality: An International and Domestic Comparative}\\
\textit{点评:\textcolor{red}{for} 表对象。}

\item 西方发达国家残疾人社会保障的成功经验对我国的启示\\
\textbf{The Enlightenment \textcolor{red}{of} Experiences regarding Social Security for the Disabled \textcolor{red}{in} Western Developed Countries \textcolor{red}{to} China}\\
\textit{点评:\textcolor{red}{of} 所属;\textcolor{red}{in} 地域;\textcolor{red}{to} 指向对象。}

\end{enumerate}



\section*{经济与国际关系 Economics \& International Relations}
\begin{enumerate}

\item 中朝双边经贸关系现状与前景\\
\textbf{Status and Outlook \textcolor{red}{of} Sino-DPRK Bilateral Trade and Economic Relation}\\
\textit{点评:\textcolor{red}{of} 所属关系。}

\item 中美贸易互补性的实证研究\\
\textbf{Empirical Analysis \textcolor{red}{on} the Trade Complementarity \textcolor{red}{of} Sino-U.S. Trade}\\
\textit{点评:\textcolor{red}{on} 表主题;\textcolor{red}{of} 所属。}

\item 日韩两国建立碳足迹标签制度的经验及启示\\
\textbf{Mutual Experiences and Revelations \textcolor{red}{of} Establishing Carbon Footprint Labeling \textcolor{red}{in} Japan and Korea}\\
\textit{点评:\textcolor{red}{of} 所属;\textcolor{red}{in} 地点范围。}

\item 东亚经济一体化和跨太平洋伙伴关系协议——中日之间的博弈\\
\textbf{The Economic Integration \textcolor{red}{in} East Asia and TPP: A Game \textcolor{red}{between} China and Japan}\\
\textit{点评:\textcolor{red}{in} 范围;\textcolor{red}{between} 双方博弈。}

\item 为可持续发展创造临界点的政治智慧\\
\textbf{The political wisdom \textcolor{red}{of} enabling tipping points \textcolor{red}{for} sustainable development}\\
\textit{点评:\textcolor{red}{of} 所属;\textcolor{red}{for} 目的。}

\item 循环经济与生态系统韧性:可持续发展的跨学科分析\\
\textbf{Circular Economy and Ecosystem Resilience: An Interdisciplinary Analysis \textcolor{red}{of} Sustainable Development}\\
\textit{点评:\textcolor{red}{of} 所属。}

\item 代表可持续基础设施发展的多学科决策基本原理的本体论\\
\textbf{Ontology representing multidisciplinary decision-making rationales \textcolor{red}{for} sustainable infrastructure developments}\\
\textit{点评:\textcolor{red}{for} 表目的对象。}

\item 数字时代的数据隐私监管:法律、技术与伦理的跨学科研究\\
\textbf{Data Privacy Regulation \textcolor{red}{in} the Digital Age: A Cross-Disciplinary Study \textcolor{red}{of} Law, Technology and Ethics}\\
\textit{点评:\textcolor{red}{in} 表时代背景;\textcolor{red}{of} 构成要素。}

\end{enumerate}



\section*{自然科学 Natural Science}
\begin{enumerate}

\item 考古学视角下的日本简史\\
\textbf{Archaeology: A Potted History \textcolor{red}{of} Japan}\\
\textit{点评:\textcolor{red}{of} 表对象。}

\item 3D模拟黑洞碰撞被誉为迄今为止最真实的一次\\
\textbf{3D Simulations \textcolor{red}{of} Colliding Black Holes Hailed \textcolor{red}{as} Most Realistic Yet}\\
\textit{点评:\textcolor{red}{of} 所属;\textcolor{red}{as} 表评判身份。}

\item 巴西发现入侵性的蓑鲉\\
\textbf{Invasive Lionfish Discovered \textcolor{red}{in} Brazil}\\
\textit{点评:\textcolor{red}{in} 地点。}

\item 在最差情况及最佳情况下报童问题的非分布研究\\
\textbf{The Distribution-free Newsboy Problem \textcolor{red}{under} the Worst-case and Best-case Scenarios}\\
\textit{点评:\textcolor{red}{under} 条件范围。}

\end{enumerate}



\section*{伦理与社会 Ethics \& Society}
\begin{enumerate}

\item 新出台标准可能加快抗癌药物研发步伐\\
\textbf{New Yardstick Could Speed Access \textcolor{red}{to} Cancer Drugs}\\
\textit{点评:\textcolor{red}{to} 表方向。}

\item 科学家们对安乐死所涉及的道德问题产生意见分歧\\
\textbf{Ethics \textcolor{red}{of} Euthanasia Divides Scientists}\\
\textit{点评:\textcolor{red}{of} 所属。}

\end{enumerate}
\end{document}
