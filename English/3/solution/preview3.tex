%!Mode:: "TeX:UTF-8"
%!TEX encoding = UTF-8 Unicode
%arara: xelatex
\documentclass[12pt]{article}
\usepackage{xeCJK}
\usepackage{geometry}
\usepackage{booktabs}
\usepackage{setspace}
\usepackage{array}
\usepackage{titlesec}

\geometry{a4paper, margin=1in}
\setstretch{1.25}
\setCJKmainfont{SimSun}
\titleformat{\section}{\large\bfseries}{\thesection.}{1em}{}
\titleformat{\subsection}{\bfseries}{\thesubsection.}{1em}{}

\begin{document}

\begin{center}
  \Large \textbf{学术英语翻译综合练习(二)· 预习部分}\\[0.5em]
  \large (Pre-study Section — Key Vocabulary and Collocations)
\end{center}

\section{核心词汇(AWL精选)}

\begin{center}
  \begin{tabular}{p{3cm} p{2cm} p{4cm} p{6cm}}
    \toprule
    \textbf{英文单词} & \textbf{词性} & \textbf{中文释义} & \textbf{常见搭配} \\
    \midrule
    allocate & v. & 分配;指派 & allocate resources / time / funding \\
    approach & n./v. & 方法;途径;接近 & theoretical approach / adopt an approach \\
    assess & v. & 评估;估计 & assess the impact / assess performance \\
    conceptualize & v. & 概念化;使形成概念 & conceptualize knowledge / conceptual framework \\
    constraint & n. & 限制;约束 & financial constraints / social constraints \\
    derive & v. & 获得;源自 & derive from / derive meaning \\
    discrete & adj. & 分离的;离散的 & discrete categories / discrete variables \\
    enhance & v. & 提高;强化 & enhance performance / enhance understanding \\
    establish & v. & 建立;确立 & establish a model / establish credibility \\
    framework & n. & 框架;结构 & theoretical framework / analytical framework \\
    generate & v. & 产生;引发 & generate data / generate interest \\
    hypothesis & n. & 假设 & formulate a hypothesis / test the hypothesis \\
    implication & n. & 含义;影响 & practical implications / policy implications \\
    inherent & adj. & 固有的;内在的 & inherent limitations / inherent value \\
    integrate & v. & 融合;结合 & integrate theory and practice / integrate data \\
    justify & v. & 证明……正当 & justify a claim / justify the conclusion \\
    notion & n. & 概念;观念 & abstract notion / challenge the notion \\
    paradigm & n. & 范式;典范 & paradigm shift / scientific paradigm \\
    precede & v. & 先于;在……之前 & events that precede the reform \\
    subsequent & adj. & 随后的 & subsequent analysis / subsequent discussion \\
    underlie & v. & 构成……的基础 & principles that underlie the theory \\
    \bottomrule
  \end{tabular}
\end{center}

\section{学术常用搭配表达(Academic Collocations)}

\begin{center}
  \begin{tabular}{p{8cm} p{8cm}}
    \toprule
    \textbf{英文表达} & \textbf{中文含义} \\
    \midrule
    play a crucial role in \dots & 在……中起关键作用 \\
    be attributed to \dots & 归因于…… \\
    in accordance with \dots & 与……一致 \\
    to a considerable extent & 在相当程度上 \\
    in the context of \dots & 在……的背景下 \\
    provide empirical evidence for \dots & 为……提供实证依据 \\
    account for the phenomenon & 解释这一现象 \\
    be subject to change & 可能会变化 \\
    pose a challenge to \dots & 对……提出挑战 \\
    shed light on \dots & 阐明;揭示 \\
    \bottomrule
  \end{tabular}
\end{center}

\section{学习建议}

\begin{itemize}
  \item \textbf{重点记搭配,而非单词本身。} 例如:\textit{justify a conclusion} 比单记 \textit{justify} 更贴近考试语境。
  \item \textbf{注意学术语体表达。} 如:
    \begin{itemize}
      \item enhance → 提升
      \item derive from → 源自
      \item inherent → 内在的
    \end{itemize}
  \item \textbf{中译英时主动使用这些词。} 即使能用更简单的表达,也尽量替换为学术词,以提升译文的学术风格。
\end{itemize}

\end{document}
