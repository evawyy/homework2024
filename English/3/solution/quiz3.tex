%!Mode:: "TeX:UTF-8"
%!TEX encoding = UTF-8 Unicode
%arara: xelatex
\documentclass[12pt]{article}
\usepackage[a4paper, margin=1in]{geometry}
\usepackage{xeCJK}
\usepackage{setspace}
% \setCJKmainfont{SimSun}
% \setstretch{1.4}
\begin{document}

\title{Academic English Translation — Advanced AWL Practice}
\author{Designed for 王胤雅}
\date{}
\maketitle

\section*{第一套}

\subsection*{中译英}
1. 政府必须制定可持续的政策,以减轻气候变化对弱势群体的影响。
2. 研究表明,文化差异会显著影响学习策略的选择与应用。
3. 数据的解释应基于透明且系统化的分析框架。
4. 该研究试图将抽象的理论模型与具体的社会实践相结合。
5. 结果显示,个体差异可能会调节变量之间的因果关系。

\subsection*{英译中}
1. The rapid expansion of digital platforms has transformed the mechanisms of social interaction.
2. The validity of the experiment depends on the accuracy of data collection and interpretation.
3. Scholars have emphasized the importance of contextualizing knowledge in diverse environments.
4. The research was conducted under the assumption that human cognition is context-dependent.
5. The proposal aims to enhance institutional accountability through transparent governance.

\newpage
\section*{第二套}

\subsection*{中译英}
1. 在经济全球化的背景下,劳动力市场的结构性变化日益明显。
2. 教育公平的实现需要跨部门的协调与长期的社会投入。
3. 科学家利用统计模型预测能源消耗的未来趋势。
4. 在伦理层面,该实验的实施需经过严格的审查与批准。
5. 语言政策的制定反映了政府对文化多样性的态度。

\subsection*{英译中}
1. The concept of sustainability has been widely adopted across multiple disciplines.
2. The findings demonstrate a positive correlation between innovation and economic growth.
3. Participants were randomly assigned to minimize potential bias.
4. The study examines how individual motivation mediates academic performance.
5. The project was terminated due to insufficient financial resources.

\newpage
\section*{第三套}

\subsection*{中译英}
1. 本研究的理论意义在于提出了一种新的解释框架,用以整合现有成果。
2. 在社会语言学研究中,语码转换被视为身份建构的重要手段。
3. 该论文对先前文献进行了批判性综述,并指出其中的逻辑漏洞。
4. 研究者采用多维度的分析方法,以提高结论的普遍性和说服力。
5. 随着信息技术的发展,知识的传播方式正经历深刻变革。

\subsection*{英译中}
1. The investigation highlights the significance of longitudinal data in behavioral studies.
2. The hypothesis was rejected because it failed to account for confounding variables.
3. Policy interventions must be adaptive to dynamic socio-economic contexts.
4. The author contends that scientific objectivity is influenced by ideological assumptions.
5. The results were consistent with previous empirical observations.

\newpage
\section*{第四套}

\subsection*{中译英}
1. 可再生能源的创新不仅关乎技术进步,也体现了全球治理的责任。
2. 研究者通过比较不同样本的统计参数,验证了假设的稳健性。
3. 在教育心理学中,动机被认为是学习成果的重要预测指标。
4. 随着社会结构的演变,传统价值观正在被重新定义。
5. 论文强调了在复杂系统研究中多变量分析的重要性。

\subsection*{英译中}
1. The research integrates qualitative and quantitative methodologies to enhance analytical rigor.
2. The data were processed using a computational model developed specifically for this study.
3. Ethical dilemmas often arise when theoretical ideals conflict with practical constraints.
4. The phenomenon was observed across distinct demographic segments.
5. The framework provides a coherent explanation for the interaction between policy and behavior.

\end{document}
