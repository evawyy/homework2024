%!Mode:: "TeX:UTF-8"
%!TEX encoding = UTF-8 Unicode
%arara: xelatex

\documentclass[12pt]{article}
\usepackage[a4paper, margin=1in]{geometry}
\usepackage{xeCJK}
\setCJKmainfont{SimSun}
\usepackage{setspace}
\setstretch{1.35}
\begin{document}

\title{研究生学术英语翻译全覆盖训练·AWL强化版}
\author{Designed for 王胤雅}
\date{}
\maketitle

\section*{第一套}

\subsection*{中译英}
1. The government must formulate sustainable policies to mitigate the impact of climate change on vulnerable populations.
2. Research indicates that cultural differences significantly influence the selection and application of learning strategies.
3. The interpretation of data should be based on a transparent and systematic analytical framework.
4. The study attempts to integrate abstract theoretical models with concrete social practices.
5. The results reveal that individual differences may moderate the causal relationship between variables.

\subsection*{英译中}
1. 数字平台的快速扩张已经改变了社会互动的机制。
2. 实验的有效性取决于数据收集与解释的准确性。
3. 学者们强调了在多元环境中情境化知识的重要性。
4. 研究是在假设人类认知依赖情境的前提下进行的。
5. 该提案旨在通过透明的治理提高机构问责性。

\textbf{考点说明:}
- 高频AWL词:formulate, mitigate, interpret, integrate, moderate, assumption, accountability
- 技巧:中译英时善用动词的学术替换(如 make → formulate, lessen → mitigate)。
- 误区:不要将 “社会实践” 直译为 *social practiceS* 时漏掉复数;此处强调多样场景。

---

\section*{第二套}

\subsection*{中译英}
1. In the context of economic globalization, structural changes in the labor market have become increasingly evident.
2. The realization of educational equity requires cross-sectoral coordination and long-term social investment.
3. Scientists used statistical models to predict future trends in energy consumption.
4. On the ethical level, the implementation of the experiment must undergo rigorous review and approval.
5. The formulation of language policies reflects the government’s attitude toward cultural diversity.

\subsection*{英译中}
1. 可持续性这一概念已在多个学科中被广泛采用。
2. 研究结果表明,创新与经济增长之间存在正相关。
3. 为最小化潜在偏差,参与者被随机分配。
4. 本研究探讨了个体动机如何在学业表现中起中介作用。
5. 该项目因财政资源不足而终止。

\textbf{考点说明:}
- 高频AWL词:coordination, investment, predict, implementation, diversity, correlation, bias, mediate, terminate
- 技巧:中译英中注意学术搭配 “require + noun (coordination/investment)”;
英译中要识别逻辑词 “to minimize bias”、“due to insufficient resources”。

---

\section*{第三套}

\subsection*{中译英}
1. The theoretical significance of this research lies in proposing a new explanatory framework to integrate existing findings.
2. In sociolinguistic studies, code-switching is regarded as an important means of identity construction.
3. The paper presents a critical review of previous literature and points out its logical inconsistencies.
4. Researchers adopted a multi-dimensional analytical approach to improve the generalizability and persuasiveness of the conclusions.
5. With the advancement of information technology, the dissemination of knowledge is undergoing profound transformation.

\subsection*{英译中}
1. 该研究强调了纵向数据在行为研究中的重要性。
2. 该假设被否决,因为它未能考虑混杂变量的影响。
3. 政策干预必须能够适应动态的社会经济环境。
4. 作者主张科学的客观性受到意识形态假设的影响。
5. 研究结果与先前的实证观察一致。

\textbf{考点说明:}
- 高频AWL词:framework, inconsistencies, generalizability, dissemination, confounding, intervention, ideology, longitudinal
- 技巧:句式名词化,如“identity construction”、“critical review”;
- 误区:confounding variable ≠ confusing variable,要区分专业术语。

---

\section*{第四套}

\subsection*{中译英}
1. Innovation in renewable energy concerns not only technological progress but also the responsibility of global governance.
2. Researchers verified the robustness of the hypothesis by comparing statistical parameters across different samples.
3. In educational psychology, motivation is regarded as a key predictor of learning outcomes.
4. As social structures evolve, traditional values are being redefined.
5. The paper emphasizes the importance of multivariate analysis in complex system research.

\subsection*{英译中}
1. 该研究结合了定性与定量方法,以提高分析的严谨性。
2. 数据通过为本研究专门开发的计算模型进行处理。
3. 当理论理想与实际约束发生冲突时,往往会出现伦理困境。
4. 这一现象在不同的人口群体中均被观察到。
5. 该框架为政策与行为之间的互动提供了连贯的解释。

\textbf{考点说明:}
- 高频AWL词:robustness, predictor, redefine, multivariate, qualitative, quantitative, dilemma, coherent
- 技巧:跨学科表达能力(教育心理学 + 政策分析 + 科学研究)
- 建议重点记忆搭配:
- *predictor of outcomes* → 结果的预测因子
- *multivariate analysis* → 多变量分析
- *ethical dilemma* → 伦理困境

---

\section*{词汇总结建议}

为备考使用建议如下:
1. 将文中粗体学术词汇整理为个人词表,标注中文含义。
2. 每个词至少写出两个固定搭配或例句,如:
- formulate a policy / formulate a theory
- mitigate negative impacts
3. 尝试反向练习:看到中文提示,写出对应的学术表达。
4. 周期性复习 Sublist 1–10,注意学科语境差异(如社会学 vs 工程)。

\end{document}
