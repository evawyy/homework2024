%!Mode:: "TeX:UTF-8"
%!TEX encoding = UTF-8 Unicode
%arara: xelatex
\documentclass{ctexart}
\newif\ifpreface
%\prefacetrue
\usepackage{tabularray}
\usepackage{titlesec}
\usepackage{setspace}
\input{../../../global/all}
\begin{document}
\large
\setlength{\baselineskip}{1.2em}
\ifpreface
\input{../../../global/preface}
\newgeometry{left=2cm,right=2cm,top=2cm,bottom=2cm}
\else
\newgeometry{left=2cm,right=2cm,top=2cm,bottom=2cm}
\maketitle
\fi
%from_here_to_type

\title{学科英文大全及易混淆学科词汇}
\author{}
\date{}

\section*{一、自然科学 Natural Sciences}
\begin{itemize}
  \item Mathematics 数学
  \item Physics 物理
  \item Chemistry 化学
  \item Biology 生物
  \item Geology 地质学
  \item Astronomy 天文学
  \item Ecology 生态学
  \item Environmental Science 环境科学
  \item Oceanography 海洋学
  \item Meteorology 气象学
\end{itemize}

\section*{二、工程与技术 Engineering \& Technology}
\begin{itemize}
  \item Computer Science 计算机科学
  \item Software Engineering 软件工程
  \item Information Technology (IT) 信息技术
  \item Electrical Engineering 电气工程
  \item Mechanical Engineering 机械工程
  \item Civil Engineering 土木工程
  \item Chemical Engineering 化学工程
  \item Biomedical Engineering 生物医学工程
  \item Aerospace Engineering 航空航天工程
  \item Materials Science 材料科学
  \item Industrial Engineering 工业工程
\end{itemize}

\section*{三、医学与健康科学 Medical \& Health Sciences}
\begin{itemize}
  \item Medicine 医学
  \item Nursing 护理学
  \item Pharmacy 药学
  \item Dentistry 牙医学
  \item Public Health 公共卫生
  \item Psychology 心理学
  \item Neuroscience 神经科学
  \item Rehabilitation Science 康复科学
  \item Nutrition Science 营养学
\end{itemize}

\section*{四、社会科学 Social Sciences}
\begin{itemize}
  \item Economics 经济学
  \item Sociology 社会学
  \item Political Science 政治学
  \item Anthropology 人类学
  \item Education 教育学
  \item Law / Legal Studies 法学
  \item International Relations 国际关系
  \item Public Administration 公共管理
  \item Linguistics 语言学
  \item Communication Studies 传播学
\end{itemize}

\section*{五、人文学科 Humanities}
\begin{itemize}
  \item History 历史
  \item Philosophy 哲学
  \item Literature 文学
  \item Art History 艺术史
  \item Religious Studies 宗教学
  \item Cultural Studies 文化研究
  \item Classics 古典学
\end{itemize}

\section*{六、艺术与创意 Arts \& Creative Studies}
\begin{itemize}
  \item Fine Arts 美术
  \item Design 设计
  \item Music 音乐
  \item Dance 舞蹈
  \item Film Studies 电影学
  \item Theater / Drama 戏剧
  \item Architecture 建筑学
\end{itemize}

\section*{七、商科 Business \& Management}
\begin{itemize}
  \item Business Administration 工商管理
  \item Accounting 会计
  \item Finance 金融
  \item Marketing 市场营销
  \item Management 管理学
  \item Human Resource Management 人力资源管理
  \item Logistics / Supply Chain Management 物流/供应链管理
  \item International Business 国际商务
\end{itemize}

\section*{八、交叉领域 Interdisciplinary Fields}
\begin{itemize}
  \item Data Science 数据科学
  \item Artificial Intelligence 人工智能
  \item Cognitive Science 认知科学
  \item Statistics 统计学
  \item Biotechnology 生物技术
  \item Environmental Engineering 环境工程
  \item Urban Studies 城市研究
  \item Gender Studies 性别研究
\end{itemize}

\newpage
\section*{易混淆学科英文词汇整理}

\subsection*{1. Psychology(心理学)}
\begin{tblr}{vlines,hlines}
  单词 & 中文 & 区别 \\
  psychology & 心理学 & 研究正常人的心理活动 \\
  psychiatry & 精神病学 & 医学领域,可开药和治疗精神疾病 \\
  psychic & 通灵的 & 完全不属于科学领域 \\
\end{tblr}
%
\subsection*{2. Linguistics(语言学)}
\begin{tblr}{hlines,vlines}
  linguistics & 语言学 & 学科本身 \\
  linguist & 语言学家/多语者 & 指人 \\
\end{tblr}

\subsection*{3. Mathematics(数学)}
\begin{tblr}{hlines,vlines}
  mathematics & 数学 & 学科 \\
  mathematician & 数学家 & 人 \\
  matrix & 矩阵 & 常被听力混淆为数学家(mathematician) \\
\end{tblr}

\subsection*{4. Statistics(统计学)}
\begin{tblr}{hlines,vlines}
  statistics & 统计学 & 学科,复数形式 \\
  statistic & 一条统计数据 & 单数名词 \\
\end{tblr}

\subsection*{5. Geology vs Geography}
\begin{tblr}{hlines,vlines}
  geology & 地质学 & 研究岩石与地球结构 \\
  geography & 地理学 & 研究地形、人口、位置 \\
\end{tblr}

\subsection*{6. Astronomy vs Astrology}
\begin{tblr}{hlines,vlines}
  astronomy & 天文学 & 科学领域 \\
  astrology & 占星术 & 伪科学/文化活动 \\
\end{tblr}

\subsection*{7. Pharmacy(药学)}
\begin{tblr}{hlines,vlines}
  pharmacy & 药学/药店 & 学科或地点 \\
  pharmacist & 药剂师 & 职业 \\
  farm & 农场 & 听力易混 \\
\end{tblr}

\subsection*{8. Sociology(社会学)}
\begin{tblr}{hlines,vlines}
  sociology & 社会学 & 学科 \\
  social & 社会的 & 形容词 \\
  socialize & 社交 & 动词 \\
\end{tblr}

\subsection*{9. Philosophy(哲学)}
\begin{tblr}{hlines,vlines}
  philosophy & 哲学 & 学科 \\
  philistine & 庸俗的人 & 意义完全不同,但拼写相似 \\
\end{tblr}

\subsection*{10. Architecture vs Archaeology}
\begin{tblr}{hlines,vlines}
  architecture & 建筑学 & 设计建筑物 \\
  archaeology & 考古学 & 研究古代遗迹 \\
\end{tblr}
\section*{科研/学术常用“缺乏”英文表达(含例句)}

\begin{tblr}{
    colspec = {Q[2.5cm] Q[2cm] Q[3.5cm] Q[6cm]},
    hlines, vlines
  }
  英文 & 词性 & 中文及使用场景 & 例句 \\
  lack & 名词 / 动词 & 缺乏,通用,可用于证据、经验、资源等 & There is a lack of data to support the hypothesis. (缺乏数据来支持假设) \\
  shortage & 名词 & 缺乏,多用于资源、物资、能源等 & The region suffers from a shortage of clean water. (该地区缺乏清洁水源) \\
  deficiency & 名词 & 不足或缺陷,多用于营养、健康或性能方面 & Vitamin D deficiency can lead to bone problems. (缺乏维生素D可能导致骨骼问题) \\
  insufficient & 形容词 & 不足、不够,常用于数量、能力或条件不足 & The evidence is insufficient to draw conclusions. (证据不足以得出结论) \\
  absence & 名词 & 缺少、没有,强调某物完全不存在 & The study notes the absence of previous research on this topic. (研究指出此前没有相关研究) \\
  dearth & 名词 & 缺乏、稀缺,书面或正式语气,多用于文学或学术写作 & There is a dearth of literature on this rare disease. (关于这种罕见疾病的文献非常少) \\
\end{tblr}
\end{document}
