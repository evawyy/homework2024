%!Mode:: "TeX:UTF-8"
%!TEX encoding = UTF-8 Unicode
%arara: xelatex
\documentclass{ctexart}
\newif\ifpreface
%\prefacetrue
\input{../../../global/all}
\begin{document}
\large
\setlength{\baselineskip}{1.2em}
\ifpreface
    \input{../../../global/preface}
\newgeometry{left=2cm,right=2cm,top=2cm,bottom=2cm}
\else
\newgeometry{left=2cm,right=2cm,top=2cm,bottom=2cm}
\maketitle
\fi
%from_here_to_type

\begin{problem} 
  \begin{itemize}
    \item   Consider two matrices \(A \) and \(B \) of dimension \(n \times n \), whose diagonal 
      elements are all nonzeros. Let \(E_X \) denote the set of edges in the 
      adjacency graph of a matrix \(X \) (that is the set of pairs \((i,j) \) 
      such \(  X_{ij} \neq 0\)), then show that \[
        E_{AB} \supset E_A \supset E_B
      \]
    \item Given extreme examples when \(|E_{AB}| =n^2 \) while \(E_A \cup E_B \) is of 
      order \(n \). What practical implications does this have on ways to store 
      products of sparse matrices? (Is this better to store \(AB \) or pairs \(A,B \) 
      separately? Consider both the computation cost for performing matrix-vector 
      products and the cost of memory.)
  \end{itemize}
\end{problem}
\begin{solution}
 \begin{enumerate}
   \item According to the definition of \(E_X \), we can get that \( E_X  \) 
 \end{enumerate}
  
\end{solution}

\begin{problem} 
  You are given an \(8 \) matrix which has the following pattern: 
  \begin{itemize}
    \item Show the adjacency graph of \(A \);
    \item Find the Cuthill Mc Kee ordering for the matrix (break ties by 
      giving priority to the node with lowest index). Show the graph of 
      the matrix permuted according to the Cuthill-Mc Kee ordering.
    \item What is the Reverse Cuthill Mc Kee ordering for this case? Show the matrix 
      reordered according to the reverse Cuthill Mc Kee ordering.
    \item Find a multicoloring of the graph using the greedy multicolor 
      algorithm. What is the minimum number of colors required for multicoloring 
      the graph? 
    \item Consider the variation of the Cuthill Mc Kee ordering in which the 
      first level consists \(L_0 \) several vertices instead on only one vertex. 
      Find the Cuthill Mc Kee ordering with this variant with the starting level 
      \(L_0 = \{1,8\} \).
  \end{itemize}
\end{problem}
  

\end{document}
