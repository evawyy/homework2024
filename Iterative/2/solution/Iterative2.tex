%!Mode:: "TeX:UTF-8"
%!TEX encoding = UTF-8 Unicode
%arara: xelatex
\documentclass{ctexart}
\newif\ifpreface
%\prefacetrue
\input{../../../global/all}
\begin{document}
\large
\setlength{\baselineskip}{1.2em}
\ifpreface
    \input{../../../global/preface}
\newgeometry{left=2cm,right=2cm,top=2cm,bottom=2cm}
\else
\newgeometry{left=2cm,right=2cm,top=2cm,bottom=2cm}
\maketitle
\fi
%from_here_to_type

\begin{problem} 
  \begin{enumerate}
    \item Prove that the iteration matrix \( G_{\omega} \) of SSOR, as defined by 
  \[
G_{\omega} = (D - \omega F)^{-1} (\omega E + (1 - \omega) D)
              (D - \omega E)^{-1} (\omega F + (1 - \omega) D).
  \]
  can be expressed as  
\[
G_{\omega} = I - \omega(2 - \omega)(D - \omega F)^{-1} D (D - \omega E)^{-1} A.
\]
\item Deduce the expression below for the preconditioning matrix associated with the SSOR iteration.
  \[
M_{SSOR} = \frac{1}{\omega (2 - \omega)} (D - \omega E) D^{-1} (D - \omega F)
\]
  \end{enumerate}
  
\end{problem}
\begin{solution}
  \begin{enumerate}
    \item 由于\(\omega E + (1-\omega )D = -(D-\omega E)+ (2-\omega )D \),\(\omega F + (1-\omega )D = -(D-\omega F)+ (2-\omega )D  \),考虑:
  \[
    \begin{aligned}
      G_{\Omega}&=(D- \omega F)^{-1}(\omega E + (1-\omega )D)(D -\omega E)^{-1}(\omega F + (1-\omega)D)\\ 
      &=(D- \omega F)^{-1}(-(D-\omega E)+ (2-\omega )D )(D -\omega E)^{-1}(-(D-\omega F)+ (2-\omega )D)\\ 
      &=(D- \omega F)^{-1}(-I+ (2-\omega )D (D -\omega E)^{-1})(-(D-\omega F)+ (2-\omega )D)\\ 
      &=I -(2-\omega)(D-\omega F)^{-1}D -(2-\omega)(D-\omega F)^{-1}D(D-\omega E)^{-1}(D-\omega F)\\ 
      &+ (2-\omega )^2(D-\omega F)^{-1}D(D-\omega E)^{-1}D\\ 
      &=I - (2-\omega)(D-\omega F)^{-1}D(D-\omega E)^{-1}(D-\omega F + (2-\omega)D + D - \omega E )\\ 
      &=I - \omega (2-\omega)(D-\omega F)^{-1}D(D-\omega E)^{-1}(D-E-F)\\
      &=I - \omega(2 - \omega)(D - \omega F)^{-1} D (D - \omega E)^{-1} A.
    \end{aligned}
  \]
\item \(I-G_{\omega}=M_{SSOR}A \) 
  \end{enumerate}
\end{solution}

 \begin{problem}
Let \(A\) be a matrix with a positive diagonal \(D\).
\begin{enumerate}[label=(\alph*)]
  \item Obtain an expression equivalent to that of (4.13) for \(G_{\omega}\), 
  but which involves the matrices
  \[
    S_E \equiv D^{-1/2} E D^{-1/2}
    \quad \text{and} \quad
    S_F \equiv D^{-1/2} F D^{-1/2}.
  \]

  \item Show that
  \[
    D^{1/2} G_{\omega} D^{-1/2}
    = (I - \omega S_F)^{-1} (I - \omega S_E)^{-1}
      (\omega S_E + (1 - \omega) I)
      (\omega S_F + (1 - \omega) I).
  \]

  \item Now assume that, in addition to having a positive diagonal, 
  \(A\) is symmetric.  
  Prove that the eigenvalues of the SSOR iteration matrix \(G_{\omega}\) 
  are real and nonnegative.
\end{enumerate}
\end{problem}
\begin{solution}
\begin{enumerate}
  \item  \( D \) 的对角元均为正,因此 \( D^{1/2} \) 与 \( D^{-1/2} \) 均存在。

注意以下恒等式:
\[
\begin{aligned}
D - \omega F &= D^{1/2} (I - \omega S_F) D^{1/2}, \\
D - \omega E &= D^{1/2} (I - \omega S_E) D^{1/2}, \\
\omega E + (1 - \omega) D &= D^{1/2} (\omega S_E + (1 - \omega) I) D^{1/2}, \\
\omega F + (1 - \omega) D &= D^{1/2} (\omega S_F + (1 - \omega) I) D^{1/2}.
\end{aligned}
\]

因此它们的逆可以写作:
\[
\begin{aligned}
(D - \omega F)^{-1} &= D^{-1/2} (I - \omega S_F)^{-1} D^{-1/2}, \\
(D - \omega E)^{-1} &= D^{-1/2} (I - \omega S_E)^{-1} D^{-1/2}.
\end{aligned}
\]

代入\(G_{\omega } \)得:
\[
\begin{aligned}
G_{\omega}&= 
  (D^{-1/2} (I - \omega S_F)^{-1} D^{-1/2})(D^{1/2} (\omega S_E + (1 - \omega) I) D^{1/2})\\
  &=(D^{-1/2} (I - \omega S_E)^{-1} D^{-1/2})(D^{1/2} (\omega S_F + (1 - \omega) I) D^{1/2})\\
  &=D^{-1/2}(I - \omega S_F)^{-1}(\omega S_E + (1 - \omega) I)(I - \omega S_E)^{-1}(\omega S_F + (1 - \omega) I)D^{1/2}
\end{aligned}
\]
这一形式仅依赖于矩阵 \( S_E, S_F \),
并且是 (4.13) 的等价表达。
\item  由上一小问的结果可立马得到 \[
    D^{1/2} G_{\omega} D^{-1/2}
    = (I - \omega S_F)^{-1} (I - \omega S_E)^{-1}
      (\omega S_E + (1 - \omega) I)
      (\omega S_F + (1 - \omega) I).
  \]
\item 由于\(A \)对称,那么\(E=F^T \),从而\(S_F=S_E^T \)。
  由第二问的结论可知,\[
\begin{aligned}
  B:&=D^{\frac{1}{2}}G_{\omega}D^{\frac{1}{2}}= (I - \omega S_E^T)^{-1} (I - \omega S_E)^{-1}\\
      &(\omega S_E + (1 - \omega) I)(\omega S_E^T + (1 - \omega) I)(I-\omega S_E^T)^{-1}(I-\omega S_E^T)\\ 

\end{aligned}
    \]
\end{enumerate}
  
\end{solution}
 
\begin{problem} 
A matrix of the form 
\[
B = 
\begin{pmatrix}
0 & E & 0 \\
0 & 0 & F \\
H & 0 & 0
\end{pmatrix}
\]
is called a three-cyclic matrix.
\begin{enumerate}[label=(\alph*)]
  \item What are the eigenvalues of \( B \)? (Express them in terms of eigenvalues of a certain matrix which depends on \( E, F, \) and \( H \).)
  
  \item Assume that a matrix \( A \) has the form \( A = D + B \), where \( D \) is a nonsingular diagonal matrix, and \( B \) is three-cyclic.  
  How can the eigenvalues of the Jacobi iteration matrix be related to those of the Gauss--Seidel iteration matrix?  
  How does the asymptotic convergence rate of the Gauss--Seidel iteration compare with that of the Jacobi iteration matrix in this case?
  
  \item Answer the same questions as in (b) for the case when SOR replaces the Gauss--Seidel iteration.
  \item    A matrix of the form
\[
B =
\begin{pmatrix}
0      & E_1    & 0      & \cdots & 0      \\
0      & 0      & E_2    & \ddots & \vdots \\
\vdots & \ddots & \ddots & \ddots & 0      \\
0      & \cdots & 0      & 0      & E_{p-1} \\
E_p    & 0      & \cdots & 0      & 0
\end{pmatrix}
\]
is called a \(p\)-cyclic matrix, where the blocks \(E_1,\dots,E_p\) are conformable
(for instance, each \(E_i\) is an \(m_i\times m_{i+1}\) block with \(m_{p+1}=m_1\)).
\end{enumerate}
\end{problem}
  \begin{problem}
Consider the linear system \(Ax=b\), where \(A\) is symmetric positive definite.  
Consider a projection step with \(K=L=\operatorname{span}\{v\}\), where \(v\) is some nonzero vector.  
Let \(x_{\text{new}}\) be the new iterate after one projection step from \(x\), and let
\[
d = A^{-1}b - x,\qquad d_{\text{new}} = A^{-1}b - x_{\text{new}}.
\]

\begin{enumerate}[label=(\alph*)]
  \item Show that
  \[
    (A d_{\text{new}},\, d_{\text{new}})
    = (A d,\, d) - \frac{(r, v)^2}{(A v, v)},
  \]
  where \(r=b-Ax\). Does this equality establish convergence of the algorithm?
  
  \item (Gastinel's method.) In Gastinel's method the vector \(v\) is selected so that
  \((v,r)=\|r\|_1\); for example one can take \(v\) with components \(v_i=\operatorname{sign}(e_i^T r)\),
  where \(r=b-Ax\) is the current residual. Show that
  \[
    \|d_{\text{new}}\|_A \le \Bigl(1-\frac{1}{n\kappa(A)}\Bigr)^{1/2}\,\|d\|_A,
  \]
  where \(\kappa(A)\) is the spectral condition number of \(A\). Does this prove that the algorithm converges?
  
  \item Compare the cost of one step of this method with that of cyclic Gauss--Seidel (see Example 5.1)
  and with that of an “optimal” Gauss--Seidel in which at each step \(K=L=\operatorname{span}\{e_i\}\)
  and \(i\) is chosen to be the index of largest magnitude in the current residual vector.
\end{enumerate}
\end{problem}


\begin{problem}
Consider the iteration
\[
x_{k+1} = x_k + \alpha_k d_k,
\]
where \(d_k\) is a nonzero vector called the search direction and \(\alpha_k\) is a scalar.  
Assume a method that chooses \(\alpha_k\) so that the residual \(\|r_{k+1}\|_2\) is minimized, where \(r_k=b-Ax_k\).

\begin{enumerate}[label=(\alph*)]
  \item Determine \(\alpha_k\) that minimizes \(\|r_{k+1}\|_2\).
  
  \item Show that the residual vector \(r_{k+1}\) obtained in this manner is orthogonal to \(A d_k\).
  
  \item Show that the residuals satisfy
  \[
    \|r_{k+1}\|_2 \le \|r_k\|_2 \,\sin\angle(r_k, A d_k).
  \]
  
  \item Assume that at each step \(k\) we have \((r_k, A d_k)\neq 0\). Will the method always converge?
  
  \item Now assume \(A\) is symmetric positive definite and choose \(d_k \equiv r_k\) at each step. Prove that the method converges for any initial guess \(x_0\).
\end{enumerate}
\end{problem}

\end{document}
