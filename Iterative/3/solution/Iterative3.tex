%!Mode:: "TeX:UTF-8"
%!TEX encoding = UTF-8 Unicode
%!TeX language =    en
%arara: xelatex
\documentclass{ctexart}
\newif\ifpreface
%\prefacetrue
\input{../../../global/all}
\begin{document}
\large
\setlength{\baselineskip}{1.2em}
\ifpreface
\input{../../../global/preface}
\newgeometry{left=2cm,right=2cm,top=2cm,bottom=2cm}
\else
\newgeometry{left=2cm,right=2cm,top=2cm,bottom=2cm}
\maketitle
\fi
%from_here_to_type
\begin{problem}
  In the Householder implementation of the Arnoldi algorithm, show the following points of detail:
  \begin{enumerate}[label=(\alph*)]
    \item $Q_{j+1}$ is unitary and its inverse is $Q_{j+1}^{T}$.
    \item $Q_{j+1}^{T} = P_1 P_2 \cdots P_{j+1}$.
    \item $Q_{j+1}^{T} e_i = v_i$ for $i < j$.
    \item $Q_{j+1} A V_m = V_{m+1} [e_1, e_2, \ldots, e_{j+1}] \bar{H}_m$, where $e_i$ is the $i$-th column of the $n \times n$ identity matrix.
    \item The vectors $v_1, v_2, \ldots, v_j$ are orthonormal.
    \item The vectors $v_1, \ldots, v_j$ are equal to the Arnoldi vectors produced by the Gram-Schmidt version, except possibly for a scaling factor.
  \end{enumerate}
\end{problem}
\begin{solution}
  \begin{enumerate}
    \item 因为每个 $P_k$ 都是 Householder 反射矩阵,满足
      \[
        P_k^T P_k = I,
      \]
      即 $P_k$ 是正交矩阵。正交矩阵的乘积仍为正交矩阵,因此
      \[
        Q_{j+1} = P_{j+1} P_j \cdots P_1
      \]
      也是正交矩阵。对于实矩阵而言,\(\overline{H}=H \),故
      \[
        Q_{j+1}^{-1} = Q_{j+1}^T.
      \]
    \item 每个 $P_k$ 都满足 $P_k^T = P_k$,因此
      \[
        Q_{j+1}^T = (P_{j+1} \cdots P_1)^T = P_1 P_2 \cdots P_{j+1}.
      \]
    \item 对 $i < j$,证明 $Q_{j+1}^T e_i = v_i$。

      这里的 $e_i$ 是标准基向量。
      首先说明 $v_i$ 的意义:在 Householder 版本的 Arnoldi 过程中,
      逐步构造出向量序列 $\{v_i\}$,第 $i$ 步时通过前 $i$ 个反射矩阵将相应的向量
      对齐到第 $i$ 个标准方向。

      第一步中,构造 $P_1$ 使得 $P_1 x = \alpha e_1$,于是 $v_1 = P_1 e_1 = Q_{j+1}^T e_1$
      (当 $j+1 \ge 1$ 时一致)。

      假设对所有 $i \le t$,都有 $Q_{t+1}^T e_i = v_i$,则在第 $t+1$ 步,$P_{t+1}$ 只作用于第 $t+1$ 及之后的分量,不改变前 $t$ 个标准基向量,因此有
      \[
        Q_{j+1}^T e_i = P_1 P_2 \cdots P_{j+1} e_i = P_1 \cdots P_i e_i = v_i.
      \]
      因此命题对所有 $i < j$ 成立。
    \item 证明关系式
      \[
        Q_{j+1} A V_m = [e_1, e_2, \dots, e_{m+1}] \bar{H}_m.
      \]

      标准 Arnoldi 关系为
      \[
        A V_m = V_{m+1} \bar{H}_m,
      \]
      其中 $V_m \in \mathbb{R}^{n \times m}$ 为 Arnoldi 正交基,$V_{m+1} \in \mathbb{R}^{n \times (m+1)}$,而 $\bar{H}_m$ 为 $(m+1) \times m$ 上 Hessenberg 矩阵。
      由于在 (c) 中已知 $v_i = Q_{j+1}^T e_i$,可得
      \[
        V_{m+1} = Q_{j+1}^T [e_1, e_2, \dots, e_{m+1}].
      \]
      两边左乘 $Q_{j+1}$,得到
      \[
        Q_{j+1} A V_m = [e_1, e_2, \dots, e_{m+1}] \bar{H}_m.
      \]
      这就是所需的结果。
    \item 证明向量 $v_1, v_2, \dots, v_j$ 正交归一。
      由 (c) 有 $v_i = Q_{j+1}^T e_i$。由于 $Q_{j+1}^T$ 是正交矩阵,它保持内积不变,因此
      \[
        \langle v_i, v_\ell \rangle = \langle Q_{j+1}^T e_i, Q_{j+1}^T e_\ell \rangle
        = \langle e_i, e_\ell \rangle = \delta_{i\ell}.
      \]
      于是 $\{v_1, \dots, v_j\}$ 构成一组正交归一向量。
    \item 证明 $\{v_1, \dots, v_j\}$ 与 Gram–Schmidt 版本的 Arnoldi 向量一致,至多相差一个常数因子。
      Gram–Schmidt 版和 Householder 版 Arnoldi 都在同一 Krylov 子空间
      \[
        \mathcal{K}_m(A, b) = \operatorname{span}\{b, Ab, \dots, A^{m-1} b\}
      \]
      中构造正交基,且都满足 Arnoldi 关系 $A V_m = V_{m+1} \bar{H}_m$。
      因此两种方法得到的每一步新向量方向相同,仅可能因反射方向不同而相差一个符号(或归一化常数)。
      由于在 (e) 中已证明 Householder 构造的 $\{v_i\}$ 也是单位正交的,所以两者在数值上最多相差 $\pm 1$ 的符号因子。
      故结论成立。
  \end{enumerate}
\end{solution}

\begin{problem}
  To derive the basic version of GMRES, we use the standard formula
  \begin{equation} \label{eq:5.7}
    \tilde{x} = x_0 + V \left( W^{T} A V \right)^{-1} W^{T} r_0,
  \end{equation}
  where \( V = V_m \) and \( W = A V_m \).
\end{problem}
\begin{solution}
  GMRES 要求选择 \(y\) 使得残差的二范数最小,即求解
\[
y = \arg_{z\in\mathbb{R}^m}\min \|r_0 - A V z\|_2,
\]
其中 \(W \equiv A V\)。
\[
\min_{z\in\mathbb{R}^m} \|r_0 - W z\|_2.
\]

对平方范数对 \(z\) 求导并令梯度为零,
\[
W^T W \, y = W^T r_0,
\]
即
\[
( A V )^T ( A V ) \, y = ( A V )^T r_0.
\]
由于\(A \)为非奇异矩阵,则\(W^T W\)可逆,则正规方程的解为
\[
y = (W^T W)^{-1} W^T r_0 = \big( (AV)^T (AV) \big)^{-1} (AV)^T r_0.
\]

将 \(y\) 代回近似解的表达式 \( \tilde{x} = x_0 + V y \),得到
\[
\boxed{ \;
\tilde{x} = x_0 + V (W^T W)^{-1} W^T r_0
= x_0 + V \big( (AV)^T(AV) \big)^{-1} (AV)^T r_0 \;
}
\]
\end{solution}

\begin{problem}
  Let a matrix \( A \) have the form
  \[
    A =
    \begin{pmatrix}
      I & Y \\
      0 & I
    \end{pmatrix}.
  \]
  Assume that (full) GMRES is used to solve a linear system with the coefficient matrix \( A \).
  What is the maximum number of steps that GMRES would require to converge?
\end{problem}
\begin{solution}
  
\end{solution}

\begin{problem}
  Consider a matrix of the form
  \begin{equation}
    A = I + \alpha B
  \end{equation}
  where \( B \) is skew-symmetric (real), i.e., such that \( B^T = -B \).

  \begin{enumerate}
    \item Show that \( \dfrac{(A x, x)}{(x, x)} = 1 \) for all nonzero \( x \).
    \item Consider the Arnoldi process for \( A \). Show that the resulting Hessenberg matrix will have the following tridiagonal form
      \[
        H_m =
        \begin{pmatrix}
          1 & -\eta_2 &  &  &  \\
          \eta_2 & 1 & -\eta_3 &  &  \\
          & \eta_3 & 1 & \ddots &  \\
          &  & \ddots & \ddots & -\eta_m \\
          &  &  & \eta_m & 1
        \end{pmatrix}.
      \]
    \item Using the result of the previous question, explain why the CG algorithm applied as is to a linear system with the matrix \( A \), which is nonsymmetric, will still yield residual vectors that are orthogonal to each other.
  \end{enumerate}
\end{problem}
\begin{solution}
  
\end{solution}

\end{document}
