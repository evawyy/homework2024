%!Mode:: "TeX:UTF-8"
%!TEX encoding = UTF-8 Unicode
%!TeX language =    en
%arara: xelatex
\documentclass{ctexart}
\newif\ifpreface
%\prefacetrue
\input{../../../global/all}
\begin{document}
\large
\setlength{\baselineskip}{1.2em}
\ifpreface
\input{../../../global/preface}
\newgeometry{left=2cm,right=2cm,top=2cm,bottom=2cm}
\else
\newgeometry{left=2cm,right=2cm,top=2cm,bottom=2cm}
\maketitle
\fi
%from_here_to_type
\begin{problem}
  In the Householder implementation of the Arnoldi algorithm, show the following points of detail:
  \begin{enumerate}[label=(\alph*)]
    \item $Q_{j+1}$ is unitary and its inverse is $Q_{j+1}^{T}$.
    \item $Q_{j+1}^{T} = P_1 P_2 \cdots P_{j+1}$.
    \item $Q_{j+1}^{T} e_i = v_i$ for $i < j$.
    \item $Q_{j+1} A V_m = V_{m+1} [e_1, e_2, \ldots, e_{j+1}] \bar{H}_m$, where $e_i$ is the $i$-th column of the $n \times n$ identity matrix.
    \item The vectors $v_1, v_2, \ldots, v_j$ are orthonormal.
    \item The vectors $v_1, \ldots, v_j$ are equal to the Arnoldi vectors produced by the Gram-Schmidt version, except possibly for a scaling factor.
  \end{enumerate}
\end{problem}
\begin{solution}
  \ref{pro:}
  Problem \ref{pro}
  Problem \ref{pro}
  Problem \ref{pro}

\end{solution}

\begin{problem}
  To derive the basic version of GMRES, we use the standard formula
  \begin{equation} \label{eq:5.7}
    \tilde{x} = x_0 + V \left( W^{T} A V \right)^{-1} W^{T} r_0,
  \end{equation}
  where \( V = V_m \) and \( W = A V_m \).
\end{problem}

\begin{problem}
  Let a matrix \( A \) have the form
  \[
    A =
    \begin{pmatrix}
      I & Y \\
      0 & I
    \end{pmatrix}.
  \]
  Assume that (full) GMRES is used to solve a linear system with the coefficient matrix \( A \).
  What is the maximum number of steps that GMRES would require to converge?
\end{problem}
\begin{problem}
  Consider a matrix of the form
  \begin{equation}
    A = I + \alpha B
  \end{equation}
  where \( B \) is skew-symmetric (real), i.e., such that \( B^T = -B \).

  \begin{enumerate}
    \item Show that \( \dfrac{(A x, x)}{(x, x)} = 1 \) for all nonzero \( x \).
    \item Consider the Arnoldi process for \( A \). Show that the resulting Hessenberg matrix will have the following tridiagonal form
      \[
        H_m =
        \begin{pmatrix}
          1 & -\eta_2 &  &  &  \\
          \eta_2 & 1 & -\eta_3 &  &  \\
          & \eta_3 & 1 & \ddots &  \\
          &  & \ddots & \ddots & -\eta_m \\
          &  &  & \eta_m & 1
        \end{pmatrix}.
      \]
    \item Using the result of the previous question, explain why the CG algorithm applied as is to a linear system with the matrix \( A \), which is nonsymmetric, will still yield residual vectors that are orthogonal to each other.
  \end{enumerate}
\end{problem}

\end{document}
