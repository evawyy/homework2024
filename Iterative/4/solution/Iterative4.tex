%!Mode:: "TeX:UTF-8"
%!TEX encoding = UTF-8 Unicode
%arara: xelatex
\documentclass{ctexart}
\newif\ifpreface
%\prefacetrue
\input{../../../global/all}
\begin{document}
\large
\setlength{\baselineskip}{1.2em}
\ifpreface
    \input{../../../global/preface}
\newgeometry{left=2cm,right=2cm,top=2cm,bottom=2cm}
\else
\newgeometry{left=2cm,right=2cm,top=2cm,bottom=2cm}
\maketitle
\fi
%from_here_to_type
\begin{problem}\label{pro:1}
Using the notation of Section~7.1.2, prove that 
\[
q_{j+k}(t) = t^{k} p_{j}(t)
\]
is orthogonal to the polynomials 
\[
p_1, p_2, \ldots, p_{j-k},
\]
assuming that \( k \leq j \). 
Show that if \( q_{j+k} \) is orthogonalized against 
\( p_1, p_2, \ldots, p_{j-k} \), 
the result would be orthogonal to all polynomials of degree 
\( < j + k \). 
Derive a general \textbf{Look-Ahead non-Hermitian Lanczos procedure} based on this observation.
\end{problem}
\begin{solution}
  
\end{solution}

\begin{problem}\label{pro:2}
Consider the matrices 
\[
V_m = [v_1, v_2, \ldots, v_m], \qquad 
W_m = [w_1, w_2, \ldots, w_m],
\]
obtained from the Lanczos biorthogonalization algorithm.
\begin{enumerate}
  \item What are the matrix representations of the (oblique) projector onto 
  \(\mathcal{K}_m(A, v_1)\)
  orthogonal to the subspace 
  \(\mathcal{K}_m(A^{T}, w_1)\),
  and the projector onto 
  \(\mathcal{K}_m(A^{T}, w_1)\)
  orthogonal to the subspace 
  \(\mathcal{K}_m(A, v_1)\)?
  
  \item Express a general condition for the existence of an oblique projector onto a subspace 
  \(K\),
  orthogonal to another subspace 
  \(L\).
  
  \item How can this condition be interpreted using the Lanczos vectors and the Lanczos algorithm?
\end{enumerate}
\end{problem}
\begin{solution}
  
\end{solution}

\begin{problem}\label{pro:3}
Show a three-term recurrence satisfied by the residual vectors \(r_j\) of the BCG algorithm. 
Include the first two iterates to start the recurrence. 

Similarly, establish a three-term recurrence for the conjugate direction vectors \(p_j\) in BCG.
\end{problem}
\begin{solution}
  
\end{solution}


\end{document}
