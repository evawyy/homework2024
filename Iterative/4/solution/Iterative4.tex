%!Mode:: "TeX:UTF-8"
%!TEX encoding = UTF-8 Unicode
%arara: xelatex
\documentclass{ctexart}
\newif\ifpreface
%\prefacetrue
\input{../../../global/all}
\begin{document}
\large
\setlength{\baselineskip}{1.2em}
\ifpreface
\input{../../../global/preface}
\newgeometry{left=2cm,right=2cm,top=2cm,bottom=2cm}
\else
\newgeometry{left=2cm,right=2cm,top=2cm,bottom=2cm}
\maketitle
\fi
%from_here_to_type
\begin{problem}\label{pro:1}
  \begin{enumerate}
    \item   Using the notation of Section~7.1.2, prove that
      \[
        q_{j+k}(t) = t^{k} p_{j}(t)
      \]
      is orthogonal to the polynomials
      \[
        p_1, p_2, \ldots, p_{j-k},
      \]
      assuming that \( k \leq j \).
    \item  Show that if \( q_{j+k} \) is orthogonalized against
      \( p_1, p_2, \ldots, p_{j-k} \),
      the result would be orthogonal to all polynomials of degree
      \( < j + k \).
    \item   Derive a general \textbf{Look-Ahead non-Hermitian Lanczos procedure} based on this observation.
  \end{enumerate}
\end{problem}
\begin{solution}
  \begin{enumerate}
    \item 先证明\(q_{j + k} \) 与\(p_1,\cdots,p_{j-k},k \leq j \)正交。
      对任意固定的 $m$,$1\le m\le j-k$,
      \[
        \langle q_{j+k},p_m\rangle=\langle t^{k}p_j,p_m\rangle.
      \]
      利用内积对称性,
      \[
        \langle t^{k}p_j,p_m\rangle=\langle p_j,t^{k}p_m\rangle.
      \]
      由于 $\deg p_m=m-1$,因此
      \[
        \deg\big(t^{k}p_m\big)=(m-1)+k \le (j-k-1)+k = j-1=\deg p_j.
      \]
      根据正交多项式的构造,$p_j$ 与所有次数 $\le j-2$ 的多项式正交;
      若 $ \deg(t^{k}p_m)\le j-2$ 则内积为零,即\( \langle q_{j + k}, p_m\rangle =0\).
      若 $\deg (t^k p_m)=j-1$,即\(m=j-k \)也可分解并用标准 Gram-Schmidt 的线性代数事实论证:
      $p_j$ 在构造满足\(\forall f \) $\deg f <j-1$ \(\langle f,p_j\rangle =0 \)。由此得 $\langle p_j,t^{k}p_m\rangle=0$,从而
      \[
        \langle q_{j+k},p_m\rangle=0,
      \]
      对所有 $m=1,\dots,j-k$ 成立,证毕。
    \item 把 $q_{j+k}=t^{k}p_j$ 对 $\{p_1,\dots,p_{j-k}\}$ 做 Gram-Schmidt 正交化,设正交化后的多项式为 $\widetilde q_{j+k}$。
      % 则 $\widetilde q_{j+k}$ 被构造成与 $\operatorname{span}\{p_1,\dots,p_{j-k}\}$ 中的任意多项式正交;这一子空间包含所有次数 $\le j-k-1$ 的多项式。
      任取任意多项式 $r(t)$,若 $\deg r<j+k$,则可以把 $r$ 分解为
      \[
        r = r_{\text{low}} + r_{\text{rem}},
      \]
      其中 $r_{\text{low}}\in\operatorname{span}\{p_1,\dots,p_{j-k}\}$(包含所有足够低次成分),
      故\(\deg (r_{\text{low}}) \leq j-k\),
      令 $r_{\text{rem}}:=s(t)$,其中 $\deg s<k$。
      \[
        \langle \widetilde q_{j+k}, r\rangle
        = \langle \widetilde q_{j+k}, r_{\text{low}}\rangle + \langle \widetilde q_{j+k}, t^{k}s\rangle.
      \]
      由正交化定义故\(\langle \widetilde q_{j+k}, r_{\text{low}}\rangle =0\)。
      设\(g \) 为\(q_{j + k} \)在 \(span \{p_1,\dots,p_{j-k}\} \)的投影,故\(g + \widetilde q_{j + k} =q_{j + k} \),
      且\(\langle g, t^ks\rangle =0 \).

      %%%todu:gap
      而且 $\langle \widetilde q_{j+k}, s\rangle= \langle q_{j + k} , s \rangle - \langle g, s \rangle = \langle t^kp_j,s\rangle $。
      而由于 $\deg s<j$,上式也为 $0$。因此 $\widetilde q_{j+k}$ 与任意 $\deg r<j+k$ 的多项式正交。

    \item   设在第 $j$ 步我们按常规范处得到向量$u = A v_j - \sum_{i=1}^j \alpha_{i} v_i$。
      若 $u\neq 0$,则可归一化为 $v_{j+1}$ 并类似地得到 $w_{j+1}$。若 $u\approx 0$(breakdown),设常规方法不能继续,此时执行 Look-Ahead:
      \begin{enumerate}
        \item 尝试 $u^{(1)}=A^1 v_j$,对其对 \(\{w_1,\dots,w_{j-1}\}\) 做双正交化;若得到非零向量则用之继续;
        \item 否则尝试 $u^{(2)}=A^2 v_j$,逐步增加幂数 $k$,直至找到非零向量 $u^{(k)}$,然后把该 $u^{(k)}$ 对 $w_1,\dots,w_{j-k}$ 正交化,归一化并作为新的 $v_{j+k}$。
      \end{enumerate}
      对偶方向即$w$ 基,需对称地执行 look-ahead。

      结果:生成的 $V$与 $W$基仍然是双正交的,但投影矩阵在 $V$ 基下变为宽带,而不是纯三对角。
  \end{enumerate}
  % 下面给出高层次的数学思想及可实施的伪代码。我们用向量/矩阵表述(Krylov 空间生成),对应多项式表述中的 $t$ 对应矩阵乘 $A$。
  %
  % 目标:为非厄米矩阵 $A\in\mathbb{C}^{n\times n}$ 构造双正交基 $V=[v_1,\dots,v_\ell]$ 与 $W=[w_1,\dots,w_\ell]$,满足 $W^T V = I_\ell$,并使 $A V$ 在 $V$ 基下对应一个窄带(带状)矩阵。标准非厄米 Lanczos(Bi-Lanczos)在某一步可能出现 breakdown(例如新产生的向量与已有的 $V$ 空间线性相关或投影系数为零),Look-Ahead 的思想是:当出现准分解/零投影时,不马上终止,而尝试通过构造 $A^k v_j$(对应多项式乘 $t^k p_j$)并对早期基作正交化来“跳过”该步,从而继续生成可用的基向量。
  %
  % \subsection{关键步骤与代数关系}
  % \subsection{伪代码(高层次)}
  % \caption{Look-Ahead 非厄米 Lanczos(高层次)}
  % \begin{algorithmic}[1]
  % \REQUIRE $A\in\mathbb{C}^{n\times n}$, 初始向量 $v_1$ 与对偶向量 $w_1$ 使 $w_1^T v_1=1$, 最大步数 $m$, 容差 $\varepsilon$.
  % \STATE $V=[v_1],\; W=[w_1],\; j=1$.
  % \WHILE{$j < m$}
  %   \STATE $u \leftarrow A v_j$
  %   \FOR{$i=1,\dots,j$} \COMMENT{对 $u$ 做 $W$-投影消去}
  %     \STATE $\alpha_i \leftarrow w_i^T u$
  %     \STATE $u \leftarrow u - \alpha_i v_i$
  %   \ENDFOR
  %   \IF{$\|u\|>\varepsilon$}
  %     \STATE $v_{j+1}\leftarrow u/\|u\|$,在对偶方向用 $A^T$ 做对称更新以获得 $w_{j+1}$,并令 $j\leftarrow j+1$。
  %   \ELSE
  %     \STATE // 启动 Look-Ahead
  %     \STATE $k\leftarrow 1$
  %     \REPEAT
  %       \STATE $u^{(k)} \leftarrow A^{k} v_j$  \COMMENT{数值上每乘一次 $A$ 之后立即对 $W$ 做投影消去以保持稳定}
  %       \FOR{$i=1,\dots,j-k$}
  %         \STATE $\alpha^{(k)}_i \leftarrow w_i^T u^{(k)}$
  %         \STATE $u^{(k)} \leftarrow u^{(k)} - \alpha^{(k)}_i v_i$
  %       \ENDFOR
  %       \IF{$\|u^{(k)}\|>\varepsilon$}
  %         \STATE 令 $v_{j+k}\leftarrow u^{(k)}/\|u^{(k)}\|$,并对偶方向做对称的 look-ahead 更新得到 $w_{j+k}$。
  %         \STATE $j\leftarrow j+k$,跳出 repeat。
  %       \ELSE
  %         \STATE $k\leftarrow k+1$
  %       \ENDIF
  %     \UNTIL{$k$ 超过上限或找到非零向量}
  %     \IF{未找到可用向量}
  %       \STATE \textbf{终止:} 无法继续扩展基。
  %     \ENDIF
  %   \ENDIF
  % \ENDWHILE
  % \RETURN $V,W$ 及记录的投影系数(构成带状投影矩阵)。
  % \end{algorithmic}
  % \end{algorithm}
  %
  % \subsection{实施注意事项与数值稳定性}
  % \begin{itemize}
  %   \item \textbf{每次乘 $A$ 后就做正交化}:在计算 $A^k v_j$ 时,逐次乘法并在每次乘之后对已有 $W$ 做投影/正交化,能显著降低数值线性相关的累积误差。
  %   \item \textbf{对偶方向同步 look-ahead}:当在 $V$ 方向做 $A^k v_j$ 的 look-ahead 时,需要在 $W$ 方向做对称操作(用 $A^T$ 生成相应的对偶向量并正交化),以保证 $W^T V=I$。否则双正交关系会被破坏。
  %   \item \textbf{带宽和矩阵存储}:Look-Ahead 会使投影矩阵从三对角扩展为更宽的带状矩阵;应按最大 look-ahead 长度分配相应的存储并正确记录系数位置。
  %   \item \textbf{终止条件}:若连续增大 $k$ 直到超过给定上限仍未找到可用向量,则可判断为真正的 breakdown(例如矩阵的 Krylov 子空间已被穷尽或数值问题阻止继续)。
  % \end{itemize}
  %
  % \section{结论}
  % 基于多项式层面的简单观察($q_{j+k}=t^{k}p_j$ 与前 $p_1,\dots,p_{j-k}$ 正交),可以在出现标准 Bi-Lanczos breakdown 时,通过构造 $A^k v_j$ 并对早期基向量正交化来跳过中间步骤,从而继续构造双正交基。该方法在实际数值实现中要求在 $A$ 的乘法与正交化步骤上格外小心以保证数值稳定,并在对偶方向上同时执行对称的 look-ahead。
\end{solution}

\begin{problem}\label{pro:2}
  Consider the matrices
  \[
    V_m = [v_1, v_2, \ldots, v_m], \qquad
    W_m = [w_1, w_2, \ldots, w_m],
  \]
  obtained from the Lanczos biorthogonalization algorithm.
  \begin{enumerate}
    \item What are the matrix representations of the (oblique) projector onto
      \(\mathcal{K}_m(A, v_1)\)
      orthogonal to the subspace
      \(\mathcal{K}_m(A^{T}, w_1)\),
      and the projector onto
      \(\mathcal{K}_m(A^{T}, w_1)\)
      orthogonal to the subspace
      \(\mathcal{K}_m(A, v_1)\)?

    \item Express a general condition for the existence of an oblique projector onto a subspace
      \(K\),
      orthogonal to another subspace
      \(L\).

    \item How can this condition be interpreted using the Lanczos vectors and the Lanczos algorithm?
  \end{enumerate}
\end{problem}
\begin{solution}
  设
  \[
    K = \mathcal{K}_m(A, v_1) ,\qquad
    L = \mathcal{K}_m(A^T, w_1)。
  \]
  那么\[
    K = \mathcal{K}_m(A, v_1) = \operatorname{range}(V_m), \qquad
    L = \mathcal{K}_m(A^T, w_1) = \operatorname{range}(W_m)。
  \]
  \begin{enumerate}
    \item 设\(y \)投影到\(K \)投影向量为 $x \in K$,写作 $x = V_m z$,则斜投影存在则
      \[
        W_m^T (y - V_m z) = 0.
      \]
      从而,
      \[
        W_m^T V_m \, z = W_m^T y.
      \]
      若矩阵 $W_m^T V_m$ 可逆,则唯一解为
      \[
        z = (W_m^T V_m)^{-1} W_m^T y,
      \]
      对应的斜投影矩阵为
      \[
        P_{K,L} = V_m (W_m^T V_m)^{-1} W_m^T。
      \]
      同理,将向量投影到 $L$ 且误差与 $K$ 正交的投影矩阵为
      \[
        P_{L, K} = W_m (V_m^T W_m)^{-1} V_m^T。
      \]
      Lanczos 双正交规范化,满足 $w_i^T v_j = \delta_{ij}$,则矩阵 $W_m^T V_m = I$,从而
      \[
        P_{K,L} = V_m W_m^T, \qquad P_{L,K} = W_m V_m^T。
      \]
    \item 给定两个列满秩矩阵 $V$ 和 $W$,分别生成子空间 $K$ 和 $L$,要求斜投影 $P$ 存在,
      即对任意 $y$ 存在唯一 $x \in K$ 使得 $y-x \perp L$,充要条件是
      \[
        W^T V \text{ 可逆}。
      \]
      设要对任意给定的 \(y\in\mathbb{R}^n\) 寻找投影 \(x\in K\) 满足 \(y-x\perp L\),即
      \[
        x = V z,\qquad z\in\mathbb{R}^m.
      \]
      正交条件 \(y - x \perp L\) 等价于
      \[
        W^T (y - V z) = 0.
      \]
      因此,
      \[
        W^T V\, z = W^T y
      \]
      若 \(W^T V\) 可逆,则方程有唯一解
      \[
        z = (W^T V)^{-1} W^T y,
      \]
      从而唯一确定 \(x=Vz\)。

      反之,若对任意 \(y\) 方程\(\exists x = Vz\in K \),使得\(y -x \perp L \),即 \(W^TVz=W^Ty \)都有唯一解。
      由于\(W \)列满秩,故\(W^T \)可逆。
      考虑线性映射\(\phi: \mathbb{R}^m \to \mathbb{R}^m \),\(z \to W^TVz \)为满射。
      又由\(z \)的唯一性,可知\(\phi \)为单射。
      则取 \(y\) 使得 \(W^T y\) 遍历整个 \(\mathbb{R}^m\),
      可得线性映射 \(\phi \) 是双射,因而 \(W^T V\) 可逆。
    \item   在 Lanczos 双正交化过程中,生成的向量序列满足
      \[
        w_i^T v_j = 0 \ (i \neq j), \qquad w_i^T v_i = \gamma_i。
      \]
      因此
      \[
        W_m^T V_m = \operatorname{diag}(\gamma_1, \ldots, \gamma_m)。
      \]
      若所有 $\gamma_i \neq 0$,矩阵 $W_m^T V_m$ 可逆,斜投影存在。
      若某个 $w_i^T v_i = 0$,则 $W_m^T V_m$ 奇异,无法定义斜投影。
  \end{enumerate}

\end{solution}

\begin{problem}\label{pro:3}
  \begin{enumerate}
    \item Show a three-term recurrence satisfied by the residual vectors \(r_j\) of the BCG algorithm.
      Include the first two iterates to start the recurrence.
    \item Similarly, establish a three-term recurrence for the conjugate direction vectors \(p_j\) in BCG.
  \end{enumerate}
\end{problem}
\begin{solution}
  残量向量 $r_j$ 和共轭方向向量 $p_j$ 的三项递推关系。算法的基本更新为:
  \[
    \begin{aligned}
      x_{j+1} &= x_j + \alpha_j p_j, \\
      r_{j+1} &= r_j - \alpha_j A p_j, \\
      p_{j+1} &= r_{j+1} + \beta_j p_j,
    \end{aligned}
  \]
  其中 $\alpha_j, \beta_j$ 为算法计算得到的标量系数,$p_0=r_0$。
  \begin{enumerate}
    \item
      \begin{equation} \label{eq:r-update}
        r_{j+1} = r_j - \alpha_j A p_j, \quad
        p_j = r_j + \beta_{j-1} p_{j-1}.
      \end{equation}

      将 $p_j$ 代入 $r_{j+1}$:
      \[
        r_{j+1} = r_j - \alpha_j A (r_j + \beta_{j-1} p_{j-1})
        = r_j - \alpha_j A r_j - \alpha_j \beta_{j-1} A p_{j-1}.
      \]

      又由上一轮残量更新:
      \[
        A p_{j-1} = \frac{r_{j-1} - r_j}{\alpha_{j-1}},
      \]
      代入上式得到
      \[
        r_{j+1} = r_j - \alpha_j A r_j - \alpha_j \beta_{j-1} \frac{r_{j-1}-r_j}{\alpha_{j-1}}
        = \left(1 + \frac{\alpha_j \beta_{j-1}}{\alpha_{j-1}}\right) r_j - \alpha_j A r_j - \frac{\alpha_j \beta_{j-1}}{\alpha_{j-1}} r_{j-1}.
      \]
      于是残量向量的三项递推为:
      \begin{equation} \label{eq:r3term}
        r_{j+1} = -\alpha_j A r_j + \left(1 + \frac{\alpha_j \beta_{j-1}}{\alpha_{j-1}}\right) r_j - \frac{\alpha_j \beta_{j-1}}{\alpha_{j-1}} r_{j-1}.
      \end{equation}
      起始两步迭代:
      \[
        r_1 = r_0 - \alpha_0 A p_0, \qquad
        r_2 = r_1 - \alpha_1 A p_1.
      \]
    \item 根据基本公式:
      \[
        p_{j+1} = r_{j+1} + \beta_j p_j.
      \]

      利用 $r_j = p_j - \beta_{j-1} p_{j-1}$ 代入:
      \[
        p_{j+1} = (r_j - \alpha_j A p_j) + \beta_j p_j = (p_j - \beta_{j-1} p_{j-1} - \alpha_j A p_j) + \beta_j p_j
        = -\alpha_j A p_j + (1+\beta_j) p_j - \beta_{j-1} p_{j-1}.
      \]
      因此得到共轭方向向量的三项递推:
      \begin{equation} \label{eq:p3term}
        p_{j+1} = -\alpha_j A p_j + (1+\beta_j) p_j - \beta_{j-1} p_{j-1}.
      \end{equation}
      起始两步迭代:
      \[
        p_0 \text{ 已知}, \qquad
        p_1 = r_1 + \beta_0 p_0 = r_0 - \alpha_0 A p_0 + \beta_0 p_0.
      \]
  \end{enumerate}
  % \subsection*{说明}
  %
  % 1. 上述三项递推公式中,$r_{j+1}$ 和 $p_{j+1}$ 都由前两步及当前矩阵-向量乘积表示。
  % 2. 由于 $A$ 一般非对称,递推中无法仅用过去两个向量的线性组合表示 $A r_j$ 或 $A p_j$,所以显式出现矩阵-向量乘积。
  % 3. 系数 $\alpha_j, \beta_j$ 可通过 BiCG 算法中双向正交性条件计算得到。
  % 4. 起始两步用于启动递推,之后可按公式 \eqref{eq:r3term} 和 \eqref{eq:p3term} 计算后续迭代。
\end{solution}
\end{document}
