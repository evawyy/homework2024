%!Mode:: "TeX:UTF-8"
%!TEX encoding = UTF-8 Unicode
%arara: xelatex
\documentclass{ctexart}
\usepackage{algorithm}
\usepackage{algpseudocode}
\newif\ifpreface
%\prefacetrue
\input{../../../global/all}

\begin{document}
\large
\setlength{\baselineskip}{1.2em}
\ifpreface
\input{../../../global/preface}
\newgeometry{left=2cm,right=2cm,top=2cm,bottom=2cm}
\else
\newgeometry{left=2cm,right=2cm,top=2cm,bottom=2cm}
\maketitle
\fi
%from_here_to_type

\begin{problem}\label{pro:1}
  Let a matrix $A$ and its preconditioner $M$ be SPD. Observing that
  $M^{-1}A$ is self-adjoint with respect to the $A$ inner product,
  write an algorithm similar to Algorithm~9.1 for solving the preconditioned
  linear system
  \[
    M^{-1} A x = M^{-1} b,
  \]
  using the $A$-inner product.
  The algorithm should employ only one matrix-by-vector product per CG step.
\end{problem}
\begin{solution}
  \(M^{-1} A \) is self-adjiont with \(A \) inner product:
  \((M^{-1} A x, y)_{A} =(A M^{-1} A x, y)=y^T A M^{-1}Ax=y^T(A^T (M^{-1})^T) Ax =(AM^{-1}y)^T Ax=(Ax,M^{-1}Ay)=(x,M^{-1}A)_A \)

  \begin{algorithm}[H]
    \caption{Preconditioned Conjugate Gradient using the $A$-inner product}
    \begin{algorithmic}[1]
      \State \textbf{Given:} SPD matrices $A$ and $M$, initial guess $x_0$
      \State Compute $r_0 := b - A x_0$
      \State Compute $z_0 := M^{-1} r_0$
      \State Set $p_0 := z_0$
      \For{$j = 0, 1, 2, \dots$ until convergence}
      \State Compute $q_j := A p_j$
    \State $\alpha_j := \dfrac{(A z_j, z_j)}{(M^{-1} q_j, q_j)} \)
    \State $x_{j+1} := x_j + \alpha_j p_j$
    \State $r_{j+1} := r_j - \alpha_j q_j$
    \State $z_{j+1} := M^{-1} r_{j+1}$
  \State $\beta_j := \dfrac{(A z_{j+1}, z_{j+1})}{(A z_j, z_j)} \)
  \State $p_{j+1} := z_{j+1} + \beta_j p_j$
  \EndFor
\end{algorithmic}
\end{algorithm}

\end{solution}

\begin{problem}\label{pro:2}
In order to save operations, the two matrices $D^{-1}E$ and $D^{-1}E^{T}$
must be stored when computing $\widehat{A}v$ by Algorithm~9.3.
This exercise considers alternatives.

\begin{enumerate}
\item Consider the matrix $B \equiv D \widehat{A} D$.
  Show how to implement an algorithm similar to Algorithm~9.3
  for multiplying a vector $v$ by $B$.
  The requirement is that only $E D^{-1}$ must be stored.

\item The matrix $B$ in the previous question is not the proper
  preconditioned version of $A$ by the preconditioning~(9.6).
  CG is used on an equivalent system involving $B$ but a further
  preconditioning by a diagonal must be applied.
  Which one?
  How does the resulting algorithm compare in terms of cost and storage
  with an algorithm based on 9.3?

\item It was mentioned in Section~9.2.2 that $\widehat{A}$ needed to be
  further preconditioned by $D^{-1}$.
  Consider the split-preconditioning option:
  CG is to be applied to the preconditioned system associated with
  \[
    C = D^{1/2} \widehat{A} D^{1/2}.
  \]
  Defining $\widehat{E} = D^{-1/2} E D^{-1/2}$, show that
  \[
    C = (I - \widehat{E})^{-1} D_{2} (I - \widehat{E})^{-T}
    + (I - \widehat{E})^{-1} + (I - \widehat{E})^{-T},
  \]
  where $D_{2}$ is a certain matrix to be determined.
  Then write an analogue of Algorithm~9.3 using this formulation.
  How does the operation count compare with that of Algorithm~9.3?
\end{enumerate}
\end{problem}
\begin{solution}
\begin{enumerate}
\item 首先把你给出的式子按规范写出并说明记号。设
  \[
    A = D_0 - E - E^T,\qquad \widehat{A}=(D-E)^{-1}A(D-E^T)^{-1}.
  \]
  则有
  \begin{align}
    \widehat{A}
    &= (D - E)^{-1} \bigl(D_0 - E - E^T\bigr) (D - E^T)^{-1} \nonumber\\
    &= (D - E)^{-1} \bigl[\,D_0 - 2D + (D - E) + (D - E^T)\,\bigr] (D - E^T)^{-1} \nonumber\\
    &\equiv (D - E)^{-1} D_1 (D - E^T)^{-1}
    + (D - E)^{-1} + (D - E^T)^{-1},\label{eq:Ahat-decomp}
  \end{align}
  其中我们定义
  \[
    D_1 \equiv D_0 - 2D.
  \]
  式\eqref{eq:Ahat-decomp} 正是将 $\widehat{A}$ 展开成三项之和的形式:中间项带有两边的三角因子,而另外两项则是单边的逆因子。

  定义
  \[
    B \equiv D \widehat{A} D.
  \]
  把 \eqref{eq:Ahat-decomp} 左右各乘以 $D$,得到
  \[
    B = D (D - E)^{-1} D_1 (D - E^T)^{-1} D
    + D (D - E)^{-1} D + D (D - E^T)^{-1} D.
  \]
  令
  \[
    F \equiv E D^{-1},\qquad I - F = I - E D^{-1}.
  \]
  注意到
  \[
    D(D-E)^{-1} = (I - F)^{-1},\qquad (D-E^T)^{-1}D = (I - F^T)^{-1},
  \]
  于是
  \begin{equation}\label{eq:B-decomp}
    B = (I-F)^{-1} D_1 (I-F^T)^{-1} + (I-F)^{-1} D + D (I-F^T)^{-1}.
  \end{equation}
  由此可设计只需存储 $F=ED^{-1}$ 的乘法算法.

  伪码(对任意向量 $v$ 计算 $w=Bv$):
  \begin{algorithm}[H]
    \caption{用 $F=ED^{-1}$ 计算 $w=Bv$}
    \begin{algorithmic}[1]
      \State $\widehat{v} := D v$
      \State $z := (I - F^T)^{-1} \widehat{v}$
      \State $w := (I - F)^{-1} \bigl( \widehat{v} + D_{1}D^{-1} z \bigr)$
      \State $w := w + z$
    \end{algorithmic}
  \end{algorithm}

  以上步骤只用到了矩阵与向量的三种操作:对 $(I-F)$ 或 $(I-F^T)$ 的三角解、对角矩阵乘法($D$ 与 $D_1$ 作用向量)以及向量加法。
  因此只需存储 $F=ED^{-1}$以及 $D$ 和 $D_1$ 的对角条目.
\item 下证明证明存在可逆矩阵 $P$ 使得
  \[
    D^{-1}B = P^{-1}(M^{-1}A)P,
  \]
  从而说明 $D^{-1}B$ 与 $M^{-1}A$ 相似,具有相同的特征值。

  取
  \[
    P = (D - E^{T})^{-1}D,
    \qquad
    P^{-1} = D^{-1}(D - E^{T}).
  \]

  由定义可知
  \[
    M^{-1}A = (D - E^{T})^{-1} D (D - E)^{-1} A.
  \]

  计算
  \[
    \begin{aligned}
      P^{-1}(M^{-1}A)P
      &=
      \bigl[D^{-1}(D - E^{T})\bigr]
      \bigl[(D - E^{T})^{-1} D (D - E)^{-1} A \bigr]
      \bigl[(D - E^{T})^{-1}D\bigr] \\[4pt]
      &= D^{-1} D (D - E)^{-1} A (D - E^{T})^{-1} D \\[4pt]
      &= (D - E)^{-1} A (D - E^{T})^{-1} D.
    \end{aligned}
  \]
  上式第二步使用了 $(D - E^{T})(D - E^{T})^{-1} = I$ 以及 $D^{-1}D = I$。

  由 $B = D \, A^{\wedge} D = D (D - E)^{-1} A (D - E^{T})^{-1} D$,可得
  \[
    D^{-1}B = (D - E)^{-1} A (D - E^{T})^{-1} D.
  \]
  因此
  \[
    D^{-1}B = P^{-1}(M^{-1}A)P.
  \]

  因此,
  \[
    D^{-1}B = P^{-1}(M^{-1}A)P, \qquad
    P = (D - E^{T})^{-1}D.
  \]
  由于相似矩阵具有相同的谱(特征值集合),因此 $D^{-1}B$ 与 $M^{-1}A$ 在谱性质和收敛行为(例如预条件共轭梯度法的收敛性)上是等价的。

  下比较基于 \(D^{-1}B\) 的实现与 Algorithm 9.3 的成本与存储:
  令 \(n\) 为未知量维数,记
  \[
    m \equiv \mathrm{nnz}(E)
  \]
  为矩阵 \(E\) 中的非零元个数(严格下三角部分)。
  在稀疏三角解中主要成本与 \(m\) 成正比;对角缩放与向量加法的成本为 \(O(n)\)。
  用常数 \(c\) 表示每个非零在三角求解中造成的平均乘加次数(常数级数)。

  下面给出两种算法计算一次矩阵–向量乘(Algorithm 9.3:计算 \(\widehat A v\),以及基于 \(D^{-1}B\) 的实现:计算 \(y=D^{-1}Bv\))的典型步骤与成本估计。
  Algorithm 9.3:计算 \(w=\widehat A v\):
  \begin{enumerate}
    \item \(\widehat v \leftarrow D^{-1} v\). \quad \(O(n)\)(对角缩放)
    \item \(z \leftarrow (I - D^{-1}E^{T})^{-1} \widehat v\). \quad 三角解,约 \(c\,m\)
    \item \(t \leftarrow \widehat v + D_{1} z\). \quad \(O(n)\)(对角乘 + 向量加)
    \item \(w \leftarrow (I - D^{-1}E)^{-1} t\). \quad 三角解,约 \(c\,m\)
    \item \(w \leftarrow w + z\). \quad \(O(n)\)
  \end{enumerate}
  因此主项 FLOP 约为
  \[
    \mathrm{FLOP}_{9.3} \approx 2c\,m + O(n).
  \]
  基于 \(D^{-1}B\)(存 \(F=ED^{-1}\))的实现,计算 \(y=D^{-1}Bv\):
  典型实现可为:
  \begin{enumerate}
    \item  对角缩放 $\widehat{v} := Dv$,成本 $O(n)$;
    \item  上三角系统 $(I - F^T)^{-1}\widehat{v}$,成本 $\approx c\,\mathrm{nnz}(E)$;
    \item  对角操作 $D_1 D^{-1} z$ 与向量加法,成本 $O(n)$;
    \item  下三角系统 $(I - F)^{-1}(\cdot)$,成本 $\approx c\,\mathrm{nnz}(E)$;
    \item  向量加法 $w := w + z$,成本 $O(n)$。
    \item   $w := D^{-1} w $,成本 $O(n)$。
  \end{enumerate}
  因此总 FLOP 约为
  \[
    \mathrm{FLOP}(Bv) \approx 2c\,\mathrm{nnz}(E) + O(n),
  \]
  与 Algorithm 9.3 相同阶。唯一的差别在于前后的对角缩放次序,可能多或少一次 $O(n)$ 操作,对总体复杂度影响可忽略。

  存储比较(稀疏矩阵与对角向量):
  两种方法所需的主要存储项为稀疏矩阵数据(value + 索引)与若干长度为 \(n\) 的对角向量。
  若同时显式存储矩阵与其转置(以便高效做三角解与乘法),两种方法的稀疏存储量如下:
  \begin{itemize}
    \item Algorithm 9.3:存储 \(D^{-1}E\) 与 \((D^{-1}E)^T\),非零数目约为 \(2m\)(数值)以及相应的索引;另外存对角向量 \(D\)、\(D_1\)(各 \(n\) 个元素)。
    \item 基于 \(D^{-1}B\):存储 \(F=ED^{-1}\) 与 \(F^T\),非零数目同样约为 \(2m\);另外存对角向量 \(D\)、\(D_1\)。
  \end{itemize}
  因此以非零元计的稀疏存储量在两种实现间为同阶:
  \[
    \text{稀疏存储} = O(m),
    \qquad \text{额外对角向量} = O(n).
  \]
  因此,基于 \(D^{-1}B\)的实现与书中 Algorithm 9.3 在主计算量和存储量上是同阶的:两者的主项均为两次稀疏三角解,
  复杂度约为 \(O(m)\)(取 \(m=\mathrm{nnz}(E)\)),并各自需要若干 \(O(n)\) 的对角向量存储與操作。
  两者的差别仅在常数因子(是否多出一次对角缩放、是否显式存转置等),因此在实际工程中可根据存储与实现便利性选择任一方案,而不必担心阶上性能劣化。
\item  已知
  \[
    \widehat{A}=(D-E)^{-1}D_{1}(D-E^{T})^{-1}+(D-E)^{-1}+(D-E^{T})^{-1},
  \]
  其中 \(D_{1}\) 是由 \(A\) 的对角或题中给出的定义得到的矩阵(例如 \(D_{1}=D_{0}-2D\))。
  \[
    C \equiv D^{1/2}\widehat{A}D^{1/2}.
  \]
  令
  \[
    \widehat{E} \equiv D^{-1/2} E D^{-1/2}.
  \]
  则有缩放恒等式
  \[
    D - E = D^{1/2}(I - \widehat{E})D^{1/2},
    \qquad
    (D - E)^{-1} = D^{-1/2}(I - \widehat{E})^{-1}D^{-1/2},
  \]
  以及类似地
  \[
    (D - E^{T})^{-1} = D^{-1/2}(I - \widehat{E}^{T})^{-1}D^{-1/2}.
  \]

  将上述三式代入 \(\widehat{A}\) 并左右乘以 \(D^{1/2}\),得到
  \[
    \begin{aligned}
      C
      &= D^{1/2}\Big[(D-E)^{-1}D_{1}(D-E^{T})^{-1}+(D-E)^{-1}+(D-E^{T})^{-1}\Big]D^{1/2} \\
      &= \bigl[D^{1/2}(D-E)^{-1}\bigr] D_{1} \bigl[(D-E^{T})^{-1}D^{1/2}\bigr]
      + D^{1/2}(D-E)^{-1}D^{1/2} + D^{1/2}(D-E^{T})^{-1}D^{1/2} \\
      &= (I-\widehat{E})^{-1}\,D^{-1/2}D_{1}D^{-1/2}\,(I-\widehat{E})^{-T}
      + (I-\widehat{E})^{-1} + (I-\widehat{E})^{-T}.
    \end{aligned}
  \]
  于是令
  \[
    \,D_{2} \equiv D^{-1/2}D_{1}D^{-1/2}\,
  \]
  可得所需分解
  \[
    \,C = (I - \widehat{E})^{-1} D_{2} (I - \widehat{E})^{-T}
    + (I - \widehat{E})^{-1} + (I - \widehat{E})^{-T}\,
  \]
  基于上面的分解,对任意向量 \(v\) 计算 \(y=Cv\) 的步骤与原 Algorithm 9.3 完全类似,
  只是把 \(D^{-1}E\) 换成了 \(\widehat{E}\)、把 \(D_{1}\) 换成了 \(D_{2}\),并在必要时做 \(D^{\pm1/2}\) 的缩放。一个直接的伪码如下:

  \begin{algorithm}[H]
    \caption{用 \(\widehat{E}=D^{-1/2}ED^{-1/2}\) 计算 \(y=Cv\)}
    \begin{algorithmic}[1]
      \State $z \gets (I - \widehat{E}^{T})^{-1} v$
      \State $t \gets D_{2}\, z + v$
      \State $u \gets (I - \widehat{E})^{-1} t$
      \State $w \gets u +z$
    \end{algorithmic}
  \end{algorithm}
  运算量与原 Algorithm 9.3比较:
  \begin{itemize}
    \item \textbf{主要开销:} 两者的主要开销均来自对形如 \((I - \cdot)^{-1}\) 的稀疏三角求解(前代/后代)。
      在 split 形式中需求解的是 \((I - \widehat{E})\) 与 \((I - \widehat{E}^{T})\),其稀疏图与原来 \((I - D^{-1}E)\)、\((I - D^{-1}E^{T})\) 相同,
      因此每次三角求解的代价仍约为 \(O(c\cdot\mathrm{nnz}(E))\),其中 \(c\) 为每个非零对应的平均乘加常数。
    \item \textbf{对角乘与缩放:} 中间需要进行对角乘 \(D_{2} z\)($O(n)$),以及可能的 \(D^{\pm1/2}\) 缩放($O(n)$);这些为低阶项。
    \item \textbf{总 FLOP:} 阶上与 Algorithm~9.3 相同,均为
      \[
        \mathrm{FLOP} \approx 2c\,\mathrm{nnz}(E) + O(n),
      \]
      即两次稀疏三角解加若干 \(O(n)\) 的对角操作与向量加法。
    \item \textbf{预处理一次性成本:} 构造 \(\widehat{E}=D^{-1/2}ED^{-1/2}\) 与 \(D_{2}=D^{-1/2}D_{1}D^{-1/2}\) ,成本为一次性 \(O(\mathrm{nnz}(E)+n)\)。
  \end{itemize}

  \paragraph{存储比较}
  \begin{itemize}
    \item 需存储稀疏矩阵 \(\widehat{E}\),其非零数与 \(E\) 相同,故稀疏存储为 \(O(\mathrm{nnz}(E))\);还需存对角向量 \(D^{1/2}\) 及 \(D_{2}\)($O(n)$)。
    \item 与 Algorithm 9.3存储 \(D^{-1}E\) 或 \(ED^{-1}\)相比,存储阶相同;
      区别仅为是否显式存放 \(D^{\pm1/2}\) 与是否预先计算并存储 \(\widehat{E}\),这些差别为常数项。
  \end{itemize}
  综上,采用 \(C=D^{1/2}\widehat{A}D^{1/2}\) 并将 \(E\) 缩放为 \(\widehat{E}\)后,
  得到的矩阵-向量乘法形式与 Algorithm9.3 在结构上完全相似:主要仍为两次稀疏三角求解加若干对角乘与向量相加。
  故在渐近运算量与存储量上,与 Algorithm 9.3 同阶,差别仅为常数因子与一次性预处理(计算并存储 \(D^{\pm1/2}\)、\(\widehat{E}\)、\(D_{2}\) 等)。
\end{enumerate}

\end{solution}

\begin{problem}\label{pro:3}
Let $M = LU$ be a preconditioner for a matrix $A$.
Show that the left, right, and split preconditioned matrices all have the same eigenvalues.
Does this mean that the corresponding preconditioned iterations will converge in
\begin{enumerate}
\item exactly the same number of steps?
\item roughly the same number of steps for any matrix?
\item roughly the same number of steps, except for ill-conditioned matrices?
\end{enumerate}
\end{problem}
\begin{solution}
% \section*{题目}
% 设 \(M=LU\) 为矩阵 \(A\) 的一个分解预条件子(\(L\) 可逆、\(U\) 可逆)。证明左预处理、右预处理和分裂(split)预处理得到的矩阵具有相同的特征值。并回答:
% \begin{enumerate}
% \item 这是否意味着相应的预处理迭代在精确算术下会在\emph{完全相同}的步数内收敛?
% \item 是否会对任意矩阵大致在相同步数收敛?
% \item 是否对除了病态矩阵以外的大多数矩阵大致相同?
% \end{enumerate}
%
先证明特征值相同:
令三种预处理后的矩阵为
\[
\text{左预处理:}\quad M^{-1}A,
\qquad
\text{右预处理:}\quad AM^{-1},
\qquad
\text{分裂(split)预处理:}\quad L^{-1} A U^{-1}.
\]
注意到 \(M=LU\) 可逆,且
\[
M^{-1}=U^{-1}L^{-1}.
\]

首先证明 \(M^{-1}A\) 与 \(AM^{-1}\) 相似:
\[
AM^{-1} = A(U^{-1}L^{-1}) = (LU)(U^{-1}L^{-1}A) = L\big(UU^{-1}L^{-1}A\big)
= L\big(M^{-1}A\big)L^{-1}.
\]
因此
\[
AM^{-1} = L\,(M^{-1}A)\,L^{-1},
\]
即 \(AM^{-1}\) 与 \(M^{-1}A\) 相似,从而它们具有相同的特征值。

再证明分裂预处理与 \(M^{-1}A\) 相似。由 \(M^{-1}=U^{-1}L^{-1}\) 得
\[
M^{-1}A = U^{-1}L^{-1}A.
\]
两边左乘 \(U\) 并右乘 \(U^{-1}\):
\[
U(M^{-1}A)U^{-1} = U(U^{-1}L^{-1}A)U^{-1} = L^{-1}A U^{-1},
\]
即
\[
L^{-1} A U^{-1} = U\,(M^{-1}A)\,U^{-1}.
\]
因此 \(L^{-1} A U^{-1}\) 与 \(M^{-1}A\) 相似,故它也具有与前两者相同的特征值。

综上,三种预处理矩阵互为相似变换,因此具有完全相同的特征值集合(代数重数也相同)。

\begin{enumerate}
\item 不成立。反例: 取
  \[
    A =
    \begin{pmatrix} 1 & 100 \\ 0 & 1
    \end{pmatrix}, \quad M = I.
  \]
  左预处理 \(M^{-1}A = A\) 与右预处理 \(AM^{-1}=A\) 的特征值都是 \(1,1\)。
  但 \(A\) 是非正规矩阵,GMRES 迭代步数依赖非对角元素大小。
  左/右预处理对应的残差方向不同,因此迭代步数可以完全不同。
  \(\Rightarrow\) 特征值相同不保证步数完全相同。
\item 不成立。反例:对于任意非正规或病态矩阵 \(A\),即使左/右/分裂预处理矩阵特征值相同,迭代收敛速度依然可能差别很大。
  例如上例矩阵 \(A=
    \begin{pmatrix}1 & 100\\0 & 1
  \end{pmatrix}\) 已经说明,对非正规矩阵,步数不一定大致相同。
\item 成立(在 SPD 或接近正规矩阵的情况下)。
  假设 \(A\) 为对称正定(SPD)矩阵,\(M\) 也是 SPD。
  左预处理 \(M^{-1}A\)、右预处理 \(AM^{-1}\)、分裂预处理 \(L^{-1}AU^{-1}\) 都是对称正定或相似于 SPD 矩阵。
  对 SPD 矩阵的 CG 迭代收敛步数仅依赖条件数 \(\kappa\):
  \[
    \|x_k - x^*\|_A \le 2 \left(\frac{\sqrt{\kappa}-1}{\sqrt{\kappa}+1}\right)^k \|x_0 - x^*\|_A
  \]
  由于三种预处理矩阵谱相同,条件数相同,迭代步数也大致相同。
\end{enumerate}

\end{solution}

\end{document}
